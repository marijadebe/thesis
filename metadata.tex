%%% Please fill in basic information on your thesis, which will be automatically
%%% inserted at the right places. You need to replace \xxx{...} by real data.

% Type of your thesis:
%	"bc" for Bachelor's
%	"mgr" for Master's
%	"phd" for PhD
%	"rig" for rigorosum
\def\ThesisType{bc}

% Language of your study programme:
%	"cs" for Czech
%	"en" for English
\def\StudyLanguage{cs}

% Thesis title in English (exactly as in the official assignment)
% (Note: \xxx is a "ToDo label" which makes the unfilled visible. Remove it.)
\def\ThesisTitle{Category of acts of monoids on sets}

% Author of the thesis (you)
\def\ThesisAuthor{Tomáš Buriánek}

% Year when the thesis is submitted
\def\YearSubmitted{2025}

% Name of the department or institute, where the work was officially assigned
% (according to the Organizational Structure of MFF UK in English,
% see https://www.mff.cuni.cz/en/faculty/organizational-structure,
% or a full name of a department outside MFF)
\def\Department{Department of Algebra}

% Is it a department (katedra), or an institute (ústav)?
\def\DeptType{Department}

% Thesis supervisor: name, surname and titles
\def\Supervisor{doc. Mgr. Jan Žemlička, Ph.D.}

% Supervisor's department (again according to Organizational structure of MFF)
\def\SupervisorsDepartment{Department of Algebra}

% Study programme (does not apply to rigorosum theses)
\def\StudyProgramme{General Mathematics}

% An optional dedication: you can thank whomever you wish (your supervisor,
% consultant, who provided you with tea and pizza, etc.)
\def\Dedication{A thank you belongs to both of my parents and the rest of my family for their support during my studies. I would also like to express gratitude to my supervisor, doc. Žemlička, for his guidance, patience and constructive feedback.}

% Abstract (recommended length around 80-200 words; this is not a copy of your thesis assignment!)
\def\Abstract{The thesis topic is the theory of monoidal actions, or acts. First we define the localisation of pointed acts, Rees simple acts and prove some of their basic properties. 
In the second part of the thesis we study the relevant analogy of the Abelian concept of exactness, i.e. Rees exactness of pointed acts, we define chain complexes of pointed acts and prove an analogue of the four, five and the nine lemma, the snake lemma
and the zig-zag lemma for chain complexes of pointed acts. The four lemma and the five lemma for pointed acts was already proved by M. Jafari, et al. in 2019, though we supply a different proof.}


% 3 to 5 keywords (recommended) separated by \sep
% Keywords are useful for indexing and searching for the theses by topic.
\def\ThesisKeywords{monoid\sep pointed act\sep complex of acts\sep snake lemma}

% If any of your metadata strings contains TeX macros, you need to provide
% a plain-text version for use in XMP metadata embedded in the output PDF file.
% If you are not sure, check the generated thesis.xmpdata file.
\def\ThesisAuthorXMP{\ThesisAuthor}
\def\ThesisTitleXMP{\ThesisTitle}
\def\ThesisKeywordsXMP{\ThesisKeywords}
\def\AbstractXMP{\Abstract}

% If your abstracts are long and do not fit in the infopage, you can make the
% fonts a bit smaller by this setting. (Also, you should try to compress your abstract more.)
\def\InfoPageFont{}
%\def\InfoPageFont{\small}  % uncomment to decrease font size

% If you are studing in a Czech programme, you also need to provide metadata in Czech:
% (in English programmes, this is not used anywhere)

\def\ThesisTitleCS{Kategorie akcí monoidu na množinách}
\def\DepartmentCS{Katedra algebry}
\def\DeptTypeCS{Katedra}
\def\SupervisorsDepartmentCS{Katedra algebry}
\def\StudyProgrammeCS{Obecná matematika}

\def\ThesisKeywordsCS{monoid\sep akt s nulou\sep komplexy aktů\sep hadí lemma}

\def\AbstractCS{Práce se věnuje teorii monoidových působení neboli aktů. Definujeme funktor lokalizace aktů s nulou, vlastnost Reesovské jednoduchosti a studujeme dokazujeme jejich základní vlastnosti. 
V další části textu se věnujeme relevantní analogii Abelovského konceptu exaktnosti, tj. Reesovská exaktnost, v kategorii aktů s nulou, definujeme řetěžové komplexy aktů s nulou a dokazujeme obdobu lemmatu čtyř, pěti a devíti,
hadího lemmatu a cikcak lemmatu pro řetěžové komplexy aktů s nulou. Lemma čtyř a pěti pro akty s nulou již dokázal M. Jafari, et al. v roce 2019, v práci dodáváme jiný důkaz.}