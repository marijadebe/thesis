%%% This file contains definitions of various useful macros and environments %%%
%%% Please add more macros here instead of cluttering other files with them. %%%

%%% Switches based on thesis type

\def\TypeBc{bc}
\def\TypeMgr{mgr}
\def\TypePhD{phd}
\def\TypeRig{rig}

\ifx\ThesisType\TypeBc
\def\ThesisTypeName{bachelor}
\def\ThesisTypeTitle{BACHELOR THESIS}
\fi

\ifx\ThesisType\TypeMgr
\def\ThesisTypeName{master}
\def\ThesisTypeTitle{MASTER THESIS}
\fi

\ifx\ThesisType\TypePhD
\def\ThesisTypeName{doctoral}
\def\ThesisTypeTitle{DOCTORAL THESIS}
\fi

\ifx\ThesisType\TypeRig
\def\ThesisTypeName{rigorosum}
\def\ThesisTypeTitle{RIGOROSUM THESIS}
\fi

\ifx\ThesisTypeName\undefined
\PackageError{thesis}{Unknown thesis type.}{Please check the definition of ThesisType in metadata.tex.}
\fi

%%% Switches based on study program language

\def\LangCS{cs}
\def\LangEN{en}

\ifx\StudyLanguage\LangCS
\else\ifx\StudyLanguage\LangEN
\else\PackageError{thesis}{Unknown study language.}{Please check the definition of StudyLanguage in metadata.tex.}
\fi\fi

%%% Minor tweaks of style

% These macros employ a little dirty trick to convince LaTeX to typeset
% chapter headings sanely, without lots of empty space above them.
% Feel free to ignore.
\makeatletter
\def\@makechapterhead#1{
  {\parindent \z@ \raggedright \normalfont
   \Huge\bfseries \thechapter\quad #1
   \par\nobreak
   \vskip 20\p@
}}
\def\@makeschapterhead#1{
  {\parindent \z@ \raggedright \normalfont
   \Huge\bfseries #1
   \par\nobreak
   \vskip 20\p@
}}
\makeatother

% This macro defines a chapter, which is not numbered, but is included
% in the table of contents.
\def\chapwithtoc#1{
\chapter*{#1}
\addcontentsline{toc}{chapter}{#1}
}

% Draw black "slugs" whenever a line overflows, so that we can spot it easily.
\overfullrule=1mm

%%% Macros for definitions, theorems, claims, examples, ... (requires amsthm package)

\theoremstyle{plain}
\newtheorem{theorem}{Theorem}
\newtheorem{proposition}[theorem]{Proposition}
\newtheorem{corollary}[theorem]{Corollary}
\newtheorem{lemma}[theorem]{Lemma}
\newtheorem{claim}[theorem]{Claim}

\theoremstyle{definition}
\newtheorem{definition}[theorem]{Definition}

\theoremstyle{remark}
\newtheorem*{remark}{Remark}
\newtheorem*{example}{Example}

%%% Style of captions of floating objects (figures etc.)

\ifcsname DeclareCaptionStyle\endcsname
\DeclareCaptionStyle{thesis}{style=base,font=small,labelfont=bf,labelsep=quad}
\captionsetup{style=thesis}
\captionsetup[algorithm]{style=thesis,singlelinecheck=off}
\captionsetup[listing]{style=thesis,singlelinecheck=off}
\fi

%%% An environment for typesetting of program code and input/output
%%% of programs.
\DefineVerbatimEnvironment{code}{Verbatim}{fontsize=\small, frame=single}

% Settings for lstlisting -- program listing with syntax highlighting
\ifcsname lstset\endcsname
\lstset{
  language=C++,
  tabsize=2,
  showstringspaces=false,
  basicstyle=\footnotesize\tt\color{black!75},
  identifierstyle=\bfseries\color{black},
  commentstyle=\color{green!50!black},
  stringstyle=\color{red!50!black},
  keywordstyle=\color{blue!75!black}}
\fi

% Floating listings, used in the same way as the figure environment
\ifcsname DeclareNewFloatType\endcsname
\DeclareNewFloatType{listing}{}
\floatsetup[listing]{style=ruled}
\floatname{listing}{Program}
\fi


\newcommand*{\funcat}[2]{\left[#1,#2 \right]}
\newcommand*{\cat}[1]{\mathrm{\textbf{#1}}}
\newcommand*{\catc}{\mathcal{C}}
\newcommand*{\catd}{\mathcal{D}}
\newcommand*{\cate}{\mathcal{E}}
\newcommand*{\catj}{\mathcal{J}}
\newcommand*{\catm}{\mathcal{M}}
\newcommand*{\ob}[1]{\text{ob}(#1)}
\newcommand{\opp}[1]{#1^{\text{op}}}
\newcommand*{\circop}{\circ_{\text{op}}}
\newcommand*{\homc}[3]{#1(#2,#3)}
\newcommand*{\ins}[1]{\iota_{#1}}
\newcommand*{\im}{\text{im}}
\newcommand*{\act}[2]{_{\mathcal{#1}}#2}
\newcommand*{\actcat}[1]{#1\mathrm{-Act}_0}
\newcommand*{\spec}[1]{\mathrm{Spec}(#1)}
\newcommand{\con}[1]{\mathrm{Con}(#1)}
\renewcommand{\ker}[1]{\mathrm{ker}(#1)}
\newcommand*{\Kernel}[1]{\mathrm{Ker}(#1)}
\newcommand*{\Cokernel}[1]{\mathrm{Coker}(#1)}
\newcommand*{\Image}[1]{\mathrm{Im}(#1)}

%%% TODO items: remove before submitting :)
\newcommand{\xxx}[1]{\textcolor{red!}{#1}}

%%% Detailed settings of bibliography

\ifx\citet\undefined\else

% Maximum number of authors of a single work. If exceeded, "et al." is used.
%\ExecuteBibliographyOptions{maxnames=2}
% The same setting specific to citations using \citet{...}
\ExecuteBibliographyOptions{maxcitenames=2}
% The same settings specific to the list of literature
%\ExecuteBibliographyOptions{maxbibnames=2}

% Shortening first names of authors: "E. A. Poe" instead of "Edgar Allan Poe"
%\ExecuteBibliographyOptions{giveninits}
% The same without dots ("EA Poe")
%\ExecuteBibliographyOptions{terseinits}

% If your bibliography entries are hard to break into lines, try this mode:
%\ExecuteBibliographyOptions{block=ragged}

% Possibly reverse the names of the authors with the non-ISO styles:
%\DeclareNameAlias{default}{family-given}

% Use caps-and-small-caps for family names in ISO 690 style.
\let\familynameformat=\textsc

% We want to separate multiple authors in citations by commas
% (while we use semicolons in the bibliography as per the ISO standard)
\DeclareDelimFormat[textcite]{multinamedelim}{\addcomma\space}
\DeclareDelimFormat[textcite]{finalnamedelim}{\space and~}

\fi
