\chapter{Pointed acts}
\section{Basic notions}
\begin{definition}
    A pointed monoid is a $4$-tuple $(S,\ast,1,0)$, where $0,1\in S$ and $\ast: S\times S \to S$ a map, that satisfies the following
    \begin{enumerate}
        \item $\forall x,y,z\in S: (x\ast y)\ast z = x\ast(y\ast z)$
        \item $\forall x\in S: x\ast 1 = x = 1\ast x$
        \item $\forall x\in S: x\ast 0 = 0 = 0\ast x$
    \end{enumerate}
    we will call $1$ the \textbf{unit} of $(S,\ast,1,0)$ and $0$ the \textbf{zero} of $(S,\ast,1,0)$. We will often identify
    such a $4$-tuple with its underlying set $S$, if the structure is generic or clear from context. Often we will also write
    $xy$ instead of $x\ast y$ and omit unnecessary brackets.
\end{definition}
\begin{example}
    The most natural examples are $(\mathbb{N}_0, \cdot, 1,0)$ and $(\mathbb{N}_0\cup\{\infty\}, +,0,\infty)$. Given any ring $(R,+,-,0,\cdot,1)$ 
    we can also forget some of the structure and get a pointed monoid $(R,\cdot,1,0)$.  
\end{example}
\begin{example}
    In general given any monoid $(M,\cdot,1)$, we can consider the disjoint union $\{\bullet\}\times\{0\}\cup M\times\{1\}$ with any one element set
    $\{\bullet\}$ and extend the operation $\cdot$ as
    \[
        (m,1)\cdot(n,1) = (m\cdot n,1)
    \]
    \[
        (m,1)\cdot(\bullet, 0) = (\bullet, 0) = (\bullet,0)\cdot(m,1)
    \]
    \[
        (\bullet,0)\cdot(\bullet,0) = (\bullet,0)
    \]
    which gives us a pointed monoid, with the further property that $ab=0 \implies a=0 \lor b=0$. It can be shown that if pointed monoid 
    has this property, then it is isomorphic to this construction.
\end{example}

\begin{definition}
    Let $S$ be a pointed monoid. A (left) pointed $S$-act is a $3$-tuple $(A,\cdot, 0)$, where $0\in A$ and $\cdot: S\times A \to A$ 
    a map, that satisfies the following
    \begin{enumerate}
        \item $\forall r,s\in S\, \forall x\in A : (r\ast s)\cdot x = r\cdot(s\cdot x)$
        \item $\forall x\in A : 1\cdot x = x$
        \item $\forall x\in A : 0\cdot x = 0$
    \end{enumerate}
    it then follows that $r\cdot 0 = r\cdot (0\cdot 0) = (r\ast 0)\cdot 0 = 0\cdot 0 = 0$ and that 
    such an element $0$ satisfying $2.$ is necessarily unique. We will call the element $0\in A$ the 
    \textbf{sink} of $A$.
\end{definition}
\begin{example}
    Any pointed monoid can be considered as a pointed act over itself, where the action $\cdot$ is just the multiplication in the monoid. 
    Given a module $(M,+,-,0,\cdot)$ over a ring $(R,+,-,0,\cdot,1)$ we can view it as a pointed act $(M,\cdot,0)$ over the pointed monoid $(R,\cdot,1,0)$.
\end{example}
\begin{example}
    We can consider any pointed set $(X,x_0)$ (that is $X$ a set and $x_0\in X$) to be pointed $S$-act for any pointed monoid $S$,
    by letting $s\cdot x = x_0$ for all $x\in X$.
\end{example}
\begin{definition}
    Let $A$ be a pointed $S$-act. A \textbf{subact} of $A$ is a subset $B\subseteq A$ such that $0\in B$ and $B$ is closed under the action of $S$. 
    A \textbf{homomorphism} of pointed acts $f: A\to B$ is a map, such that $f(0) = 0$ and $f(r\cdot x) = r\cdot f(x)$ for any $r\in S$ and $x\in A$.
    A \textbf{congruence} on a pointed act $A$ is an equivalence relation $\rho$ on $A$ such that 
    \[
        \forall x,y\in A \,\forall r \in S: (x,y)\in\rho \implies (r\cdot x, r\cdot y)\in\rho
    \]
    A \textbf{factor} of a pointed act $A$ by a congruence $\rho$ is the set of equivalence classes $A/\rho$ with the action defined as $r \cdot [a]_\rho = [r \cdot a]_\rho$ and the zero element $[0]_\rho$. 
    This construction yields a pointed act.
\end{definition}
\begin{example}
    A kernel congruence of a homomorphism $f: A\to B$ is a congruence on $A$ defined as $\ker{f} = \{(x,y)\in A\times A \mid f(x) = f(y)\}$.
\end{example}
\begin{remark}
    The category of pointed acts over a pointed monoid $S$, denoted as $\actcat{S}$ consists of pointed acts as objects and their homomorphisms as morphisms.
    The composition of morphisms and the identity morphisms are defined as in the category of sets, clearly a composition of two homomorphisms yields a homomorphism
    and the identity is trivially a homomorphism.
\end{remark}
\begin{definition}
    Let $A\in S\mathrm{-Act}_0$ and $B$ a subact of $A$, then we let 
    \[
        \rho_B = \Delta_A \cup B\times B
    \]
    where $\Delta_A$ is the diagonal congruence on $A$. We call $\rho_B$ a \textbf{Rees congruence on $A$}. 
\end{definition}
\begin{proof}
    We will show, that $\rho_B$ is a congruence. \par 
    Reflexivity and symmetry of the relation is clear from definition. If $(x,y)\in\rho_B$ and $(y,z)\in\rho_B$, we either have 
    $x=y=z \implies (x,z)\in\rho_B$ or at least one of the pairs is in $B\times B$, if both are then again $(x,z)\in\rho_B$. If 
    say, WLOG, that $x=y$ and $y,z\in B$, then also $x\in B$ and therefore again $(x,z)\in\rho_B$, thus the relation is transitive. 
    If $(x,y)\in\rho_B$ and $r\in S$ we have either $x=y\implies rx=ry$ or $x,y\in B \implies rx,ry\in B$, so $\rho_B$ is a congruence on $A$.
\end{proof}
\begin{remark}
    For $B$ a subact of $A$ we will often denote the factor $A/\rho_B$ as $A/B$. 
\end{remark}

\section{Constructions}
\begin{proposition}
    Let $f,g: A \to B$ be homomorphisms of pointed acts. Let $\{f=g\} := \{x\in A \mid f(x) = g(x)\}$, then $\{f=g\}$ is a subact of $A$ and 
    the canonical inclusion $\iota: \{f=g\} \to A$, $x\mapsto x$ is an equaliser in $\actcat{S}$.
\end{proposition}
\begin{proof}
    We clearly have that $0\in\{f=g\}$ and $x\in\{f=g\}$ implies $f(x)=g(x)$, hence $f(rx)=rf(x)=rg(x)=g(rx) \implies rx\in\{f=g\}$. 
    Suppose $h: X \to A$ is such that $fh=gh$, then we have that $h(x)\in\{f=g\}$, so we simply let $\overline{h} = h: X\to\{f=g\}$, this
    map is unique. Say there were two maps $\iota\overline{h_1}=h=\iota\overline{h_2}$, then because $\iota$ is injective (so a monomorphism, 
    because we are in a concrete category) we have $\overline{h_1}=\overline{h_2}$.
\end{proof}
    We will denote the kernel as $\Kernel{f} = \{f=0\}$. 
\begin{proposition}
    Suppose $f,g: A\to B$ are homomorphisms, let $\sim$ be the smallest congruence on $B$ such that $\forall x\in A: f(x)\sim g(x)$, then 
    $B/\sim$ with the canonical projection $\pi: B\to B/\sim$, $x\mapsto [x]_\sim$ is a coequaliser in $\actcat{S}$. 
\end{proposition}
\begin{proof}
    Suppose that $h: B\to X$ is such that $hf=hg$, then we have that $\ker{h}$ is a congruence on $B$, satisfying 
    $\forall x\in A: (f(x),g(x))\in\ker{h}$, therefore $\sim\subseteq\ker{h}$. We need a morphism $\overline{h}: B/\sim \to X$, which satisfies 
    $h = \overline{h} \circ\pi$. The uniqueness is clear, if it exists, since then $\overline{h_1}\circ\pi = h = \overline{h_2}\circ\pi$, 
    which implies $\overline{h_1}=\overline{h_2}$, because $\pi$ is epi. We define $\overline{h}([x]_\sim) = h(x)$, this 
    is clearly a homomorphism of acts and it is well-defined because
    \[
        [x]_\sim = [y]_\sim\iff x\sim y \implies (x,y)\in\ker{h} \implies h(x)=h(y)
    \] 
\end{proof}
    In the case of cokernels, the congruence $\sim$ admits a much nicer description. 
\begin{proposition}
    Let $f: A \to B$ be a homomorphism of pointed acts, then the smallest congruence on $B$ such that $\forall x\in A :f(x)\sim 0$ is
    $\im(f):=\rho_{\Image{f}}$, where $\Image{f} = \{f(x) \mid x\in A\}$.
\end{proposition}
\begin{proof}
    We have that $(f(x),0)\in\Image{f}$, because $0\in\Image{f}$. If $\sim$ is any other such congruence, then for any $f(x),g(x)\in\Image{f}$
    it holds that $f(x)\sim 0 \sim g(x) \implies f(x)\sim g(x)$, therefore $\rho_{\Image{f}}=\Image{f}\times\Image{f}\cup\Delta_B\subseteq\sim$.
\end{proof}
We will now construct the products and coproducts in $\actcat{S}$. 
\begin{proposition}
    Let $I$ be a nonempty set, $(X_i)_{i\in I}$ a collection of pointed acts over a pointed monoid $S$, then we define 
    \[
        \prod_{i\in I}X_i = \{f: I \to\bigcup_{i\in I } X_i \mid\forall i\in I :  f(i)\in X_i \}
    \]
    then $\prod_{i\in I}X_i$ admits a structure of a pointed act and along with the canonical projections 
    $\pi_i : \prod_{i\in I }X_i$,$f\mapsto f(i)$ it is the product in the category $\actcat{S}$.
\end{proposition}
\begin{proof}
    We define the action of $S$ on the product componentwise as $(r\cdot f)(i) = r\cdot f(i)$. The zero in the product 
    is the map $i\mapsto 0$. It is routine to check that such a structure satisfies the 
    axioms of pointed acts. By definition of the action and zero we see that $\pi_i$ is a homomorphism. Suppose now that 
    object $U$ along with the maps $\mu_i : U \to X_i$ is a cone in $\actcat{S}$. Define a map $f: U \to\prod_{i\in I} X_i$ 
    as $f(u) = g$, where $\forall i\in I : g(i) := \mu_i(u)$. We have that 
    \[
        f(ru)(i) = \mu_i(ru) = r\mu_i(u) = r\cdot f(u)(i) = (r\cdot f(u))(i)
    \]  
    \[
        f(0)(i) = \mu_i(0) = 0 \implies f(0) = 0
    \]
    so it is a homomorphism of pointed acts that satisfies $\mu_i = \pi_i\circ f$. If $f'$ is any other such map we would have that 
    $\forall i\in I : \pi_i\circ f = \pi_i\circ g$, then fixing $u\in U$ and $i\in I$ we get $g(u)(i) = (\pi_i \circ g)(u) = (\pi_i\circ f)(u) = f(u)(i) \implies g(u)=f(u)$ 
    and $f=g$.
\end{proof}
\begin{proposition}
    Let $I$ be a nonempty set, $(X_i)_{i\in I}$ a collection of pointed acts over a pointed monoid $S$, then we define 
    \[
        \coprod_{i\in I } X_i = \{f\in\prod_{i\in I }X_i \mid \exists j\in I\,\forall i\in I : i\neq j \implies f(i) = 0\}
    \]
    then $\coprod_{i\in I} X_i$ is a subact of the product and along with the canonical inclusions $\iota_i : X_i \to \coprod_{i\in I} X_i$, $x\mapsto f$, where 
    $f(i) = x$ and $f(j)=0$ for $j\neq i$
    forms the coproduct in $\actcat{S}$.
\end{proposition}
\begin{proof}
    Clearly $0\in\coprod_{i\in I } X_i$ and given any $r\in S$ and $f\in\coprod_{i\in I } X_i$, where 
    $f(j)=0$ for $j\neq i$, then $r\cdot f(j)=0$, hence $r\cdot f \in\coprod_{i\in I} X_i$. It follows that 
    $\coprod_{i\in I} X_i$ is a subact of the product.\par 
    Let $U$ be a pointed act and $\nu_i : X_i \to U$, then let $h: \coprod_{i\in I} X_i \to U$ be defined as follows - 
    given any $f$ and $i\in I$, such that $f(j)=0$ for all $j\neq i$, define $h(f) = \nu_i(f(i))$ (Note that this is well-defined 
    even in the case of zero, because $\forall i\in I : \nu_i(0) = 0$). By definition we have that $h\circ \iota_i = \nu_i$. It is 
    indeed a homomorphism, since $r\cdot h(f) = r\cdot\nu_i(f(i))=\nu_i((r\cdot f)(i)) = h(r\cdot f)$. Let 
    $\overline{h} : \coprod_{i\in I } X_i \to U$ be any other morphism, such that $\forall i\in I :\overline{h}\circ\iota_i = \nu_i$ we see, given 
    any $f$, letting $x=f(i)$, where $i\in I$ is such that $\forall j\neq i : f(j)=0$, that 
    \[
        h(f) = \nu_i(x) = \overline{h}(\iota_i(x)) = \overline{h}(f) \implies h=\overline{h} 
    \]
\end{proof}
\begin{corollary}
    The category $\actcat{S}$ is complete and cocomplete.
\end{corollary}
    Our category $\actcat{S}$ equipped with the "forgetful" functor $U$, which maps objects to their underlying sets and morphisms 
    to their underlying set maps is a concrete category.
\begin{proposition}
    The monomorphisms in $\actcat{S}$ are precisely the injective homomorphisms and the epimorphisms in $\actcat{S}$ are precisely
    the surjective homomorphisms.
\end{proposition}
\begin{proof}
    See \cite[Proposition~6.15]{Kilp00}.
\end{proof}

\begin{proposition}
    Let $(S,\cdot,0,1)$ be a pointed commutative monoid and 
    $K\subseteq S$ a submonoid of $S$ such that $0\not\in K$. 
    We define $K^{-1}S = S\times K/\sim$, where 
    \[
        (s_1,k_1)\sim(s_2,k_2)\iff \exists v\in K : s_1k_2v = s_2k_1v 
    \]
    then $K^{-1}S$ is well-defined and admits the structure of a pointed commutative monoid.
\end{proposition}
\begin{proof}[Proof]
    Relation $\sim$ is reflexive, since $1\in K$. It is clearly symmetric. 
    Let $(s_1,k_1)\sim(s_2,k_2)\sim(s_3,k_3)$, so there are $v,w\in K$ such that 
    $s_1k_2v = s_2k_1v$ and $s_2k_3w = s_3k_2w$, then 
    \[
    (s_1k_3)k_2vw = s_1k_2vk_3w = s_2k_1vk_3w = s_2k_3wk_1v = s_3k_2wk_1v = (s_3k_1)k_2vw
    \]
    therefore $(s_1,k_1)\sim(s_3,k_3)$. We will denote the equivalence classes as
    $\frac{s}{k} = [s,k]_{\sim}$. Define 
    \[
        \frac{s_1}{k_1}\cdot\frac{s_2}{k_2} = \frac{s_1 s_2}{k_1 k_2}
    \]
    to prove well-definedness assume $(s_1,k_1)\sim(d_1,l_1)$ and $(s_2,k_2)\sim(d_2,l_2)$, then 
    we find $u,v\in K$ such that 
    $s_1l_1u=d_1k_1u$ and $s_2l_2v = d_2k_2v$, then 
    \[
    s_1s_2l_1l_2uv = (s_1l_1u)(s_2l_2)v = d_1k_1ud_2k_2v = d_1d_2k_1k_2uv 
    \]
    thus
    \[
        \frac{s_1}{k_1}\cdot\frac{s_2}{k_2} = \frac{s_1s_2}{k_1k_2} = \frac{d_1d_2}{l_1l_2} = \frac{d_1}{l_1}\cdot\frac{d_2}{l_2}
    \]
    We define the unit $1 = \frac{1}{1}$ and $0 = \frac{0}{1}$. The associativity, commutativity and unit axioms are clearly 
    satisfied. We have 
    \[
        0\cdot\frac{s}{k} = \frac{0}{k} = \frac{0}{1} = 0
    \]
    since $(0,k)\sim(0,1)$, because $0\cdot 1 = 0 = 0\cdot k$. 
\end{proof}
\begin{definition}
    We define the \textbf{localisation map} $\Lambda: S \to K^{-1}S$ by $\Lambda(x) = \frac{x}{1}$. It is a 
    homomorphism of pointed monoids.  
\end{definition}
\begin{proposition}
    Let $K\subseteq S$ be a submonoid of a commutative pointed monoid $S$ such that $0\not\in S$ and $A$ an $S$ act. We define 
    $K^{-1}A = A\times K/\sim$, where 
    \[
        (a_1,k_1)\sim(a_2,k_2) \iff \exists u\in K : uk_1 a_2 = uk_2 a_1
    \]
    then $K^{-1}A$ admits the structure of a pointed $K^{-1} S$ act.
\end{proposition}
\begin{proof}[Proof]
    The relation is clearly reflexive and symmetric. Assume 
    that $(k_1,a_1)\sim(k_2,a_2)\sim(k_3,a_3)$, so 
    $uk_1a_2 = uk_2a_1$ and $vk_2a_3 = vk_3a_2$ for some $u,v\in K$, then 
    \[
        uvk_2(k_1a_3) = uk_1vk_2a_3 = uk_1vk_3a_2 = vk_3uk_1a_2 = vk_3uk_2a_1 = uvk_2(k_3a_1)
    \] 
    thus the relation is an equivalence. Denote the equivalence classes as $\frac{a}{k} = [a,k]_\sim$. 
    Define the action of $K^{-1}S$ on $K^{-1}A$ by 
    \[
        \frac{s_1}{k_1}\cdot\frac{a}{k} = \frac{s_1 a}{k_1 k}
    \]
    to show well-definedness assume $\frac{s_1}{k_1}=\frac{s_2}{k_2}$ and $\frac{a_1}{l_1}=\frac{a_2}{l_2}$, then 
    we find $u,v\in K$ such that $s_1k_2u = s_2k_1u$ and $vl_1a_2 = vl_2a_1$, then we have 
    \[
        uvk_2l_2s_1a_1 = s_1k_2uvl_2a_1 = s_2k_1uvl_1a_2 = uvk_1l_1s_2a_2 \implies \frac{s_1a_1}{k_1l_1} = \frac{s_2a_2}{k_2l_2}
    \]
    so 
    \[
        \frac{s_1}{k_1}\cdot\frac{a_1}{l_1} = \frac{s_1a_1}{k_1l_1} = \frac{s_2a_2}{k_2l_2} = \frac{s_2}{k_2}\cdot\frac{a_2}{l_2}
    \]
    which proves that the map is well defined. We have to show that it is an action. 
    The associativity and unit axiom is trivial. Define $0 = \frac{0}{1}$, then 
    \[
        0\cdot\frac{a}{k} = \frac{0}{1}\cdot \frac{a}{k} = \frac{0}{k} = \frac{0}{1} = 0
    \]
    since $k\cdot 0 = 1\cdot 0$. Hence it is really a pointed act over $K^{-1}S$. 
\end{proof}
\begin{proposition}
    Let $f: A\to B$ be an act homomorphism, then we have an induced 
    map $K^{-1}f : K^{-1}A \to K^{-1}B$ defined as 
    \[
        K^{-1}f\left(\frac{a}{k}\right) = \frac{f(a)}{k}
    \] 
    it is a well-defined homomorphism of acts.
\end{proposition}
\begin{proof}[Proof]
    Assume $\frac{a_1}{k_1}=\frac{a_2}{k_2}$, so we find $u\in K$ such that 
    $uk_2a_1= uk_1a_2$, then 
    \[
        uk_2f(a_1) = f(uk_2a_1) = f(uk_1a_2) = uk_1f(a_2)\] 
        \[ \implies K^{-1}f\left(\frac{a_1}{k_1}\right)=\frac{f(a_1)}{k_1}=\frac{f(a_2)}{k_2} = K^{-1}f\left(\frac{a_2}{k_2}\right)
    \]
    and 
    \[
    \frac{s_1}{k_1}\cdot K^{-1}f\left(\frac{a}{k}\right) = \frac{s_1f(a)}{k_1k} = \frac{f(s_1a)}{k_1k} = 
    K^{-1}f\left(\frac{s_1a}{k_1k}\right) = K^{-1}f\left(\frac{s_1}{k_1}\cdot\frac{a}{k}\right)
    \]
\end{proof}
\begin{proposition}
    We have a localisation functor $K^{-1}:\actcat{S}\to\actcat{K^{-1}S}$.
\end{proposition}
\begin{proof}[Proof]
    Choose $\frac{a}{k}$, then
    \[
    K^{-1}(g\circ f)\left(\frac{a}{k}\right) = \frac{g(f(a))}{k} = K^{-1}g\left(\frac{f(a)}{k}\right) =\] \[=(K^{-1}g\circ K^{-1}f)(\frac{a}{k}) \implies K^{-1}(g\circ f) = K^{-1}g\circ K^{-1}f
    \]
    and 
    \[
        K^{-1}1_A\left(\frac{a}{k}\right) = \frac{a}{k} = 1_{K^{-1}A}(\frac{a}{k}) \implies K^{-1}1_A = 1_{K^{-1}A}
    \]
\end{proof}
\begin{remark}
    The localisation map $\Lambda$ induces a functor (by restriction of scalars) from $\actcat{K^{-1}S}$ to $\actcat{S}$, where given any 
    $A$ a $K^{-1}S$ act we define the action of $S$ on $A$ by $s\cdot a = \Lambda(s)\cdot a$. The functor maps the 
    homomorphisms to themselves. This way we can consider the 
    localisation to be an endofunctor by considering the functor $\Lambda\circ K^{-1}$ instead. 
\end{remark}
\section{Regularity and exactness}
    For the remaining chapters we fix the following notation, for a homomorphism $f$ by $\im(f)$ we mean the 
    Rees congruence $\rho_{\Image{f}}$. To remind the reader by $\ker{f}$ we mean the kernel congruence $\ker{f} = \{(x,y) \mid f(x)=f(y)\}$
    and $\Kernel{f} = \{x \mid f(x) = 0\}$, $\Image{f} = \{ y \mid \exists x : f(x) =y \}$.
\begin{proposition}
    In the category of pointed acts, every monomorphism is normal (i.e. a kernel of its cokernel). 
\end{proposition}
\begin{proof}[Proof]
    Let $f: K \to A$ be a monomorphism and $\pi : A \to A/\Image{f}$ be the canonical projection. We clearly have that 
    $\pi\circ f = 0$. Let $\hat{f} : S \to A$ be any other homomorphism, such that $\pi\circ\hat{f} = 0$, then 
    given $x\in S$ we have $[\hat{f}(x)]=0$ or equivalently that $\hat{f}(x)\in\Image{f}$, therefore there is a unique 
    $x'\in K$ so that $f(x')=\hat{f}(x)$. Define $\phi(x)=x'$ to be the map from $S$ to $K$. By definition we have 
    $f\circ\phi = \hat{f}$ and this relation defines the map uniquely, because for every element in the image of $\hat{f}$ there 
    is precisely one element in $K$ that can map onto it. It remains to show that $\phi$ is really a homomorphism, fixing the 
    notation as above, let $\phi(rx)=z'$ and $\phi(x)=z_0'$, then we have 
    \[
        f(\phi(rx))=f(z')=\hat{f}(rx)=r\hat{f}(x)=rf(z_0')=f(rz_0')=f(r\phi(x))
    \]
    thus $\phi(rx)=r\phi(x)$, since $f$ is a mono, which concludes the proof.
\end{proof}

\begin{definition}
    Let $A$ and $B$ be pointed acts over a pointed monoid $\mathcal{S}$, we say that 
    $f: A\to B$ is \textbf{Rees regular}, if 
    \[
        \rho_{\Kernel{f}} = \ker{f}
    \]
    In diagrams we will denote the fact that $f$ is Rees regular as 
    \[\begin{tikzcd}
	A & B
	\arrow["f", "\circ"{marking}, from=1-1, to=1-2]
\end{tikzcd}\]
\end{definition}
\begin{remark}
    Note that we always have $\rho_{\Kernel{f}}\subseteq\ker{f}$, since $x=y\lor f(x)=0=f(y)\implies f(x)=f(y)$.
\end{remark}
\begin{proposition}
    Let $f: A\to B$ be a homomorphism of pointed acts over a pointed monoid $\mathcal{S}$, then the following are equivalent
    \begin{enumerate}
        \item $\rho_{A'} = \ker{f}$ for some subact $A'\subseteq A$.
        \item $f$ is Rees regular.
        \item $f: A \to\Image{f}$ is a conormal epimorphism.
    \end{enumerate}
\end{proposition}
\begin{proof}
    $1.\implies 2.$ Let $x\in A'$, then $(x,0)\in\ker{f} \implies f(x)=0 \implies x \in\Kernel{f}$. If $f(x)=0$, then 
    $(x,0)\in\rho_A' \implies x\in A'$. \par
    $2.\implies 3.$ Assume $\rho_{\Kernel{f}} = \ker{f}$ and let $g: A\to X$ be a homomorphism such that $g\circ\iota = 0$, where
    $\iota: \Kernel{f}\to A$ is the canonical inclusion. We need to show that there exists a unique homomorphism $\phi: \Image{f}\to X$ such that
    the following diagram commutes 
    \[\begin{tikzcd}
        && X \\
        {\Kernel{f}} & A && {\Image{f}}
        \arrow["\iota", hook, from=2-1, to=2-2]
        \arrow["g", from=2-2, to=1-3]
        \arrow["f", from=2-2, to=2-4]
        \arrow["{\exists ! \phi}"', dashed, from=2-4, to=1-3]
    \end{tikzcd}\]
    Define $\phi(y) = g(x)$, where $x\in A$ is such that $f(x)=y$. This map is well-defined, because if $f(x)=y=f(x')$, then
    either $x=x'$ or $f(x)=0=f(x') \implies g(x)=0=g(x')$, since $g\circ\iota = 0$. Clearly $\phi\circ f = g$ by definition.
    It remains to verify uniqueness. Suppose that $\phi': \Image{f}\to X$ is another such map, then for any $y\in\Image{f}$ we 
    find a $x\in A$ so that $f(x) = y$, then $\phi'(y) = \phi'(f(x)) = g(x) = \phi(f(x)) = \phi(y)$. \par
    $3. \implies 1.$ Consider the canonical projection $\pi : A \to A/\Kernel{f}$, then $\pi\circ\iota = 0$, therefore 
    there exists a unique map $\phi: \Image{f}\to X$, such that $\phi\circ f = \pi$, then if $f(x)=f(y)$ we have 
    $\phi(f(x))=\phi(f(y)) \implies [x]_\Kernel{f}=[y]_\Kernel{f}$, hence $x=y \lor x,y\in\Kernel{f}$, so 
    $\ker{f}=\rho_{\Kernel{f}}$. In particular $1.$ holds.
\end{proof}
\begin{remark}
    Note that $3.$ in the above characterisation is the definition of admissible morphism in $\actcat{S}$, 
    which appears in \cite{Flores15}.
\end{remark}
\begin{proposition}
    Assume $f:A \to B$ is Rees regular and $\sim$ a congruence on $A$, the following holds
    \begin{enumerate}
        \item $f$ is a monomorphism, if and only if $\Kernel{f} = \{0\}$.
        \item If $K$ is a multiplicative subset of $\mathcal{S}$, then $K^{-1}f$ is Rees regular.
        \item If $g: B \to C$ is Rees regular, then $g\circ f$ is.
        \item The canonical projection $\pi : A \to A/\sim$, $x\mapsto [x]$ is Rees regular, if and only if 
        $\sim$ is a Rees congruence.
    \end{enumerate}
\end{proposition}
\begin{proof}[Proof]
    Let $f(x)=f(y)$, then $(x,y)\in\rho_{\Kernel{f}} \implies x=y \lor f(x)=0=f(y) \implies x=y \lor x=0=y$.\par 
    Assume that 
    \[
        K^{-1}f\left(\frac{a}{k_1}\right) = K^{-1}f\left(\frac{b}{k_2}\right)
    \]
    then $f(kk_2a)=kk_2f(a) = kk_1f(b)=f(kk_1b)$ for some $k\in K$, so $(kk_2 a,kk_1b)\in\rho_{\Kernel{f}}$, which implies 
    \[
        \frac{a}{k_1}=\frac{b}{k_2} \lor K^{-1}f\left(\frac{a}{k_1}\right)= \frac{f(a)}{k_1} = 0 = \frac{f(b)}{k_2} = K^{-1}f\left(\frac{b}{k_2}\right) 
    \]
    Assume that $g: B\to C$ is Rees regular and $g(f(x))=g(f(y))$, since $g$ is Rees regular we have 
    \[
        f(x)=f(y) \lor g(f(x))=0=g(f(y)) \implies 
        \]
        \[\implies x=y \lor f(x)=0=f(y) \lor g(f(x))=0=g(f(y)) \implies
    \]
    \[
        \implies x=y \lor g(f(x))=0=g(f(y))
    \]
    hence $\ker{g\circ f} = \rho_{\Kernel{g\circ f}}$.\par
    Suppose the canonical projection is Rees regular, if $x\sim y$, then $[x]=[y]$, hence by regularity 
    $x=y \lor [x]=[0]=[y]$ or equivalently $x=y\lor x,y\in [0]$. Note that $[0]\leq A$, since $x\sim 0 \implies rx\sim r0 = 0$ 
    therefore it is a subact of $A$. If $\sim$ is a Rees congruence we have $[x]=[y] \implies x=y \lor x,y\in A'$ for some $A'\leq A$,
    which means precisely that $x=y \lor [x]=[0]=[y]$, because $0\in A'$. 
\end{proof}
\begin{definition}
    A \textbf{chain complex} $(A_n,a_{n} : A_n\to A_{n-1})_{n\in\mathbb{Z}}$ is a sequence of acts and act morphisms
    \[\begin{tikzcd}
        \cdots & {A_{n+1}} & {A_n} & {A_{n-1}} & \cdots
        \arrow[from=1-1, to=1-2]
        \arrow["{a_{n+1}}", from=1-2, to=1-3]
        \arrow["{a_n}", from=1-3, to=1-4]
        \arrow[from=1-4, to=1-5]
    \end{tikzcd}\]
    such that for all $n\in\mathbb{Z}$ we have $\im(a_{n+1})\subseteq\ker{a_n}$. Chain complex is \textbf{Rees exact
    at $A_n$}, if $\im(a_{n+1})=\ker{a_n}$. Chain complex is \textbf{Rees exact}, if it is Rees exact at $A_n$ for each $n\in\mathbb{Z}$. 
    We say that a chain complex is \textbf{Rees regular}, if all $a_n$ are Rees regular homomorphisms.
\end{definition}
We will often write $(A,a)$ for short to mean the chain complex $(A_n,a_{n} : A_n\to A_{n-1})_{n\in\mathbb{Z}}$. \par
Given any finite sequence of acts $\begin{tikzcd}
	{A_0} & \cdots & {A_n}
	\arrow["{f_0}", from=1-1, to=1-2]
	\arrow["{f_{n-1}}", from=1-2, to=1-3]
\end{tikzcd}$ we say that the sequence is Rees exact, if $\forall n\in\{1,\dots,n-1\} : \ker{d_n} = \im(d_{n-1})$.
\begin{remark} 
    A sequence $(A_n,a_{n} : A_n\to A_{n-1})_{n\in\mathbb{Z}}$ of pointed acts and their morphisms is a chain complex, if and only if $a_{n+1}\circ a_n = 0$. If $(A,a)$ is a chain complex 
    we always have that $\Image{a_{n+1}}\subseteq\Kernel{a_n} \forall n\in\mathbb{Z}$.
\end{remark}
\begin{proof}[Proof]
    Choose $n\in\mathbb{Z}$. 
    If $\im(a_{n+1})\subseteq\ker{a_n}$ we have for $x\in A_{n+1}$ that
    $(a_{n+1}(x),0)\in\im(a_{n+1})\subseteq\ker{a_n}$, so $(a_n\circ a_{n+1})(x) = 0$. Now assume that $a_{n+1}\circ a_n = 0$, then 
    if $(x,y)\in\im(a_{n+1})$ we either have $x=y$, which trivially implies that $a_n(x)=a_n(y)$ or 
    $\exists x_0,y_0\in A_{n+1}$ such that $a_{n+1}(x_0)=x\land a_{n+1}(y_0)=y$, which implies 
    \[
        a_n(x) = a_n(a_{n+1}(x_0)) = 0 = a_n(a_{n+1}(y_0)) = a_n(y)
    \]
    so $(x,y)\in\ker{a_n}$.\par
    If $y\in A_n$ is such that $a_{n+1}(y_0)=y$ for some $y_0\in A_{n+1}$ we 
    have $a_n(y) = a_n(a_{n+1}(y_0)) = 0 \implies y\in\Kernel{a_n}$.
\end{proof}
\begin{definition}
    Let $(A,a)$ be a chain complex and $n\in\mathbb{Z}$ we define the $n$-th homology act $H_n(A)$ as 
    \[
        H_n(A) = \Kernel{a_n}/\im(a_{n+1})
    \]
\end{definition}
\begin{proposition}\label{regularityProp}
    Given a sequence of pointed acts 
    \[\begin{tikzcd}
        \cdots & {A_{n+1}} & {A_n} & {A_{n-1}} & \cdots
        \arrow[from=1-1, to=1-2]
        \arrow["{a_{n+1}}", from=1-2, to=1-3]
        \arrow["{a_n}", from=1-3, to=1-4]
        \arrow[from=1-4, to=1-5]
    \end{tikzcd}\]
    the following are equivalent for each $n\in\mathbb{Z}$
    \begin{enumerate}
        \item $\ker{a_n}=\im(a_{n+1})$
        \item $\Image{a_{n+1}}=\Kernel{a_n}$ and $a_n$ is Rees regular.
    \end{enumerate}
\end{proposition}
\begin{proof}[Proof]
    Assume exactness at $A_n$ and let $a_n(x)=0$, then $(x,0)\in\im(a_{n+1})$, if 
    $x=0$, then trivially $a_{n+1}(0)=x$, else we find $x_0\in A_{n+1},x_0\neq 0$ such that $a_{n+1}(x_0)=x$, thus
    $\Kernel{a_n}=\Image{a_{n+1}}$ by the previous remark. Let $a_n(x)=a_n(y)$,$x\neq y$, then $(x,y)\in\ker{a_n}=\im(a_{n+1})$, 
    so $a_{n+1}(x_0)=x$ and $a_{n+1}(y_0)=y$ for some $x_0,y_0\in A_{n+1}$, therefore 
    \[
        a_n(x)= a_n(a_{n+1}(x_0)) = 0 = a_n(a_{n+1}(y_0)) = a_n(y) \implies (x,y)\in\rho_\Kernel{a_n}
    \]
    Assume that the second condition holds. We have $\ker{a_n} = \rho_{\Kernel{a_n}} = \rho_{\Image{a_{n+1}}} = \im(a_{n+1})$, 
    which concludes the proof.
\end{proof}
\begin{remark}
    For a Rees regular chain complex $(A,a)$ we have that $(A,a)$ is Rees exact at $A_n$, if and only if $H_n(A)=0$.
\end{remark}
\begin{definition}
    Let $(A,a)$ and $(B,b)$ be two chain complexes. A \textbf{chain map} $f:(A,a)\to(B,b)$ is a sequence of act morphisms $(f_n: A_n\to B_n)$ 
    such that $\forall n\in\mathbb{Z}$ the following square diagram commutes
    \[\begin{tikzcd}
        {A_n} && {B_n} \\
        \\
        {A_{n-1}} && {B_{n-1}}
        \arrow["{f_n}"{description}, from=1-1, to=1-3]
        \arrow["{a_n}", from=1-1, to=3-1]
        \arrow["{b_n}", from=1-3, to=3-3]
        \arrow["{f_{n-1}}"{description}, from=3-1, to=3-3]
    \end{tikzcd}\]
\end{definition}
\begin{proposition}
    Chain complexes of acts along with chain maps form a category $\mathrm{Ch}(\actcat{S})$.
\end{proposition}
\begin{proposition}
    Let $f: (A,a) \to (B,b)$ be a chain map and $n\in\mathbb{Z}$, then we have an induced map $H_n(f) : H_n(A) \to H_n(B)$
    defined as $H_n(f)([x]) = [f_n(x)]$. This makes $H_n(A)$ into a functor from $\mathrm{Ch}(\actcat{S})$ to $\actcat{S}$.
\end{proposition}
\begin{proof}[Proof]
    We need to show well-definedness. Assume that $[x]=[y]$, if $x=y$ we are done, otherwise $x,y\in\Image{a_{n+1}}$ so we 
    find $x_0,y_0\in A_{n+1}$ such that $a_{n+1}(x_0) = x$,$a_{n+1}(y_0)=y$. By commutativity $f_n(x)=f_n(a_{n+1}(x_0))=b_{n+1}(f_{n+1}(x_0))$
    and $f_n(y) = f_n(a_{n+1}(y_0))=b_{n+1}(f_{n+1}(y_0))$, thus $[f_n(x)]=[0]=[f_n(y)]$. The fact that $H_n(f)$ is a homomorphism 
    is clear from the definition.\par 
    Next we compute
    \[
        H_n(1_{(A,a)})([x]) = [1_{A_n}(x)] = [x] \implies H_n(1_{(A,a)}) = 1_{H_n(A)}
    \]
    \[
        H_n(g\circ f)([x]) = [g_n(f_n(x))] = H_n(g)([f_n(x)]) =
    \]
    \[
         =(H_n(g)\circ H_n(f))([x]) \implies H_n(g\circ f) = H_n(g)\circ H_n(f)
    \]
\end{proof}
\begin{lemma}
    If $f: (A,a)\to (B,b)$ is a chain map and $f_n$ is Rees regular, then $H_n(f)$ is Rees regular.
\end{lemma}
\begin{proof}[Proof]
    Assume $[f_n(x)]=H_n(f)([x])=H_n(f)([y])=[f_n(y)]$, then $f_n(x)=f_n(y) \implies x=y \lor f_n(x)=0=f_n(y)$ or 
    $f_n(x),f_n(y)\in\Image{b_{n+1}}$. In either case we have $[x]=[y]$ or $H_n(f)([x])=[f_n(x)]=0=[f_n(y)]=H_n(f)([y])$.
\end{proof}
\begin{definition}
    We say that a chain complex $(A,a)$ of pointed acts is \textbf{bounded above}, if there is some $n\in\mathbb{Z}$ such that 
    $A_m=0$ for all $m\geq n$. Similarly we say that $(A,a)$ is \textbf{bounded below}, if there exists $n\in\mathbb{Z}$ such that 
    $A_m=0$ for all $m\leq n$. Chain complex $(A,a)$ is \textbf{bounded}, if it is bounded above and below.
\end{definition}
\begin{definition}
    A \textbf{Rees short exact sequence} is a bounded Rees exact chain complex of the form
    \[\begin{tikzcd}
        0 & A & B & C & 0
        \arrow[from=1-1, to=1-2]
        \arrow["f", from=1-2, to=1-3]
        \arrow["g", from=1-3, to=1-4]
        \arrow[from=1-4, to=1-5]
    \end{tikzcd}\]
\end{definition}
\begin{remark}
    By definition of exactness it follows that $\ker{f} = \{(0,0)\}\cup\Delta_A \land \im(g)= C\times C$, therefore 
    $f$ is injective and $g$ is surjective. Similarly injectivity of $f$ and surjecivity of $g$ imply exactness at $A$ 
    and $C$ respectively.
\end{remark}
\begin{example}
    Let $A$ and $B$ be acts, then we have the following Rees short exact sequence
    \[\begin{tikzcd}
        0 & A & {A\coprod B} & B & 0
        \arrow[from=1-1, to=1-2]
        \arrow["\iota", from=1-2, to=1-3]
        \arrow["\lambda", from=1-3, to=1-4]
        \arrow[from=1-4, to=1-5]
    \end{tikzcd}\]
    where $\iota: x \mapsto (x,0)$ and 
    \[
        \lambda(x,y) = 
        \begin{cases}
            y, & \text{if } x=0 \\
            0, & \text{otherwise}
        \end{cases}
    \]
    the exactness at $A$ and $B$ is easily seen. We will show exactness 
    at $A\coprod B$.\par
    Choose $((x_1,y_1),(x_2,y_2))\in\im(\iota)$, if $x_1=x_2$ and $y_1=y_2$ we are done, 
    otherwise we have $y_1=0=y_2$, therefore 
    \[
        \lambda(x_1,y_1)=0=\lambda(x_2,y_2)
    \]
    which implies $\im(\iota)\subseteq\ker{\lambda}$.
    Let $((x_1,y_1),(x_2,y_2))\in\ker{\lambda}$, if $x_1=x_2$ and $y_1=y_2$
    we are done, therefore assume that $x_1\neq x_2$ or $y_1\neq y_2$.\par
    Suppose $x_1=0$, then $x_2\neq 0$ (else $y_1=y_2$), this implies
    that $y_2=0$ and $y_1=\lambda(x_1,y_1)=\lambda(x_2,y_2)=0$
    so $(x_1,y_1)=(0,0)$ and $y_2=0$, then 
    $\iota(x_2)=(x_2,y_2)$ and $\iota(0)=(x_1,y_1)$.\par
    Suppose now that $x_1\neq 0$, then $y_1=0$ and 
    $\lambda(x_2,y_2)=\lambda(x_1,y_1)=0$, which means either 
    $(x_2,y_2)=(0,0)$ or that $x_2\neq 0$, in either case $y_2=0$,
    hence $\iota(x_1)=(x_1,y_1)$ and $\iota(x_2)=(x_2,y_2)$.
    In conclusion $\ker{\lambda}=\im(\iota)$ as required.
\end{example}
\begin{example}
    If $B$ is a subact of $A$, then the following sequence is Rees short exact 
    \[\begin{tikzcd}
        0 & B & A & {A/B} & 0
        \arrow[from=1-1, to=1-2]
        \arrow["\iota", from=1-2, to=1-3]
        \arrow["\pi", from=1-3, to=1-4]
        \arrow[from=1-4, to=1-5]
    \end{tikzcd}\]
    where $\iota$ is the set inclusion map and $\pi$ is the canonical projection. 
    The exactness at $A$ and $A/B$ is trivial. Choose $x\in B$, then we have 
    $\pi(\iota(x)) = \pi(x) = [x] = [0] \implies \im(\iota)\subseteq\ker{\pi}$. Choose 
    $(x,y)\in\ker{\pi}$, so $[x]=[y]$, if $x=y$, then it is trivially in $\im(\iota)$, else 
    we necessarily have $[x]=[0]=[y]$, so $x,y\in B$ as required.
\end{example}
\begin{definition}
    A sequence of chain complexes 
\[\begin{tikzcd}
	0 & {(A,a)} & {(B,b)} & {(C,c)} & 0
	\arrow[from=1-1, to=1-2]
	\arrow["f", from=1-2, to=1-3]
	\arrow["g", from=1-3, to=1-4]
	\arrow[from=1-4, to=1-5]
\end{tikzcd}\]
    (where by $0$ we mean $\cdots\to 0\to0\to0\to\cdots$) is exact 
    if for each $n\in\mathbb{Z}$ we have that 
    \[\begin{tikzcd}
	0 & {A_n} & {B_n} & {C_n} & 0
        \arrow[from=1-1, to=1-2]
        \arrow["{f_n}", from=1-2, to=1-3]
        \arrow["{g_n}", from=1-3, to=1-4]
        \arrow[from=1-4, to=1-5]
    \end{tikzcd}\]
    is a Rees short exact sequence.
\end{definition}
\begin{proposition}
    Assume that $\mathcal{S}$ is a commutative, pointed monoid and $K\subseteq S$ is a submonoid of $S$ such that $0\not\in K$, then
    given a Rees short exact sequence 
    \[\begin{tikzcd}
        0 & A & B & C & 0
        \arrow[from=1-1, to=1-2]
        \arrow[from=1-2, to=1-3]
        \arrow[from=1-3, to=1-4]
        \arrow[from=1-4, to=1-5]
    \end{tikzcd}\]
    the following sequence is Rees short exact
    \[\begin{tikzcd}
        0 & {K^{-1}A} & {K^{-1}B} & {K^{-1}C} & 0
        \arrow[from=1-1, to=1-2]
        \arrow[from=1-2, to=1-3]
        \arrow[from=1-3, to=1-4]
        \arrow[from=1-4, to=1-5]
    \end{tikzcd}\]
\end{proposition}
\begin{proof}[Proof]
    Let $\frac{a}{k}\in K^{-1}A$, then 
    \[
    (K^{-1}g\circ fK^{-1}f)\left(\frac{a}{k}\right) = \frac{g(f(a))}{k} = \frac{0}{k} =\frac{0}{1}=0 \implies \im(K^{-1}f)\subseteq\ker{K^{-1}g}
    \]
    Assume that $\frac{g(b_1)}{k_1}=\frac{g(b_2)}{k_2}$, so we find $u\in K$ such that 
    $g(uk_1b_2)=uk_1g(b_2)=uk_2g(b_1)=g(uk_2b_1)$, therefore 
    $(uk_1b_2,uk_2b_1)\in\im(f)$. If $uk_1b_2=uk_2b_1$ we are done, otherwise 
    we find $z_1,z_2\in A$ such that $f(z_1) = uk_1b_2$ and $f(z_2)=uk_2b_1$.
    then 
    \[
        K^{-1}f\left(\frac{z_1}{uk_1k_2}\right) = \frac{uk_1b_2}{uk_1k_2} = \frac{b_2}{k_2}
    \] 
    \[
        K^{-1}f\left(\frac{z_2}{uk_2k_1}\right) = \frac{uk_2b_1}{uk_2k_1} = \frac{b_1}{k_1}
    \]
    which proves exactness at $K^{-1}B$, we have to show that $K^{-1}$ preserves injections and surjections.
    Suppose $\frac{f(a_1)}{k_1}=\frac{f(a_2)}{k_2}$, then find $u\in K^{-1}$ such that 
    \[
        uk_1f(a_2) = uk_2f(a_1)\implies f(uk_1a_2)=f(uk_2a_1) \implies
    \] 
    \[
        \implies uk_1a_2=uk_2a_1 \implies\frac{a_1}{k_1}=\frac{a_2}{k_2}
    \]
    To prove surjectivity choose $\frac{z}{k}\in K^{-1}C$, we find $y\in B$ such that $g(y)=z$, then 
    \[
        K^{-1}g\left(\frac{y}{k}\right) = \frac{g(y)}{k} = \frac{z}{k}
    \]
    which completes the proof.
\end{proof}
\begin{corollary}
    Let $B$ be a subact of $A$, then $K^{-1}B$ is a subact of $K^{-1}A$ and  
    \[
        K^{-1}\left(\frac{A}{B}\right) \simeq \frac{K^{-1}A}{K^{-1}B}
    \]
\end{corollary}
\begin{proof}[Proof]
    Consider the Rees short exact sequence 
    \[\begin{tikzcd}
        0 & {K^{-1}B} & {K^{-1}A} & {K^{-1}(A/B)} & 0
        \arrow[from=1-1, to=1-2]
        \arrow["{K^{-1}\iota}", from=1-2, to=1-3]
        \arrow["{K^{-1}\pi}", from=1-3, to=1-4]
        \arrow[from=1-4, to=1-5]
    \end{tikzcd}\]
    $K^{-1}B$ is clearly a subset of $K^{-1}A$, it contains zero, since $B$ contains zero and 
    it is closed under the monoidal action, because $B$ is.
    First we need to show that $\im K^{-1}\iota = \rho_{K^{-1}B}$, we have 
    \[
    \left(\frac{b_1}{k_1},\frac{b_2}{k_2}\right)\in\im(S^{-1}\iota) \iff \frac{b_1}{k_1}=\frac{b_2}{k_2}\lor\exists\,\frac{a_1}{l_1},\frac{a_2}{l_1}\in K^{-1}B :
    \frac{a_1}{l_1}=\frac{b_1}{k_1}\land\frac{a_2}{l_2}=\frac{b_2}{k_2}
    \]
    \[
        \iff \frac{b_1}{k_1}=\frac{b_2}{k_2}\lor \frac{b_1}{k_1},\frac{b_2}{k_2}\in K^{-1}B \iff \left(\frac{a_1}{k_1},\frac{a_2}{k_2}\right)\in\rho_{K^{-1}B}
    \]
    then by the first isomorphism theorem for universal algebras we have 
    \[
    K^{-1}(A/B)\simeq K^{-1}(A)/\ker{K^{-1}\pi} = K^{-1}(A)/\im(K^{-1}\iota) = K^{-1}A/K^{-1}B
    \]
\end{proof}
