%%% The main file. It contains definitions of basic parameters and includes all other parts.

% Meta-data of your thesis (please edit)
\input metadata.tex

% Generate metadata in XMP format for use by the pdfx package
\input xmp.tex

%% Settings for single-side (simplex) printing
% Margins: left 40mm, right 25mm, top and bottom 25mm
% (but beware, LaTeX adds 1in implicitly)
\documentclass[12pt,a4paper]{report}
\setlength\textwidth{145mm}
\setlength\textheight{247mm}
\setlength\oddsidemargin{15mm}
\setlength\evensidemargin{15mm}
\setlength\topmargin{0mm}
\setlength\headsep{0mm}
\setlength\headheight{0mm}
% \openright makes the following text appear on a right-hand page
\let\openright=\clearpage

%% Settings for two-sided (duplex) printing
% \documentclass[12pt,a4paper,twoside,openright]{report}
% \setlength\textwidth{145mm}
% \setlength\textheight{247mm}
% \setlength\oddsidemargin{14.2mm}
% \setlength\evensidemargin{0mm}
% \setlength\topmargin{0mm}
% \setlength\headsep{0mm}
% \setlength\headheight{0mm}
% \let\openright=\cleardoublepage

%% If the thesis has no printed version, symmetric margins look better
% \documentclass[12pt,a4paper]{report}
% \setlength\textwidth{145mm}
% \setlength\textheight{247mm}
% \setlength\oddsidemargin{10mm}
% \setlength\evensidemargin{10mm}
% \setlength\topmargin{0mm}
% \setlength\headsep{0mm}
% \setlength\headheight{0mm}
% \let\openright=\clearpage

%% Generate PDF/A-2u
\usepackage[a-2u]{pdfx}

%% Prefer Latin Modern fonts
\usepackage{lmodern}
% If we are not using LuaTeX, we need to set up character encoding:
\usepackage{iftex}
\ifpdftex
\usepackage[utf8]{inputenc}
\usepackage[T1]{fontenc}
\usepackage{textcomp}
\fi

%% Further useful packages (included in most LaTeX distributions)
\usepackage{amsmath}        % extensions for typesetting of math
\usepackage{amsfonts}       % math fonts
\usepackage{amsthm}         % theorems, definitions, etc.
\usepackage{bm}             % boldface symbols (\bm)
\usepackage{booktabs}       % improved horizontal lines in tables
\usepackage{caption}        % custom captions of floating objects
\usepackage{dcolumn}        % improved alignment of table columns
\usepackage{floatrow}       % custom float environments
\usepackage{graphicx}       % embedding of pictures
\usepackage{indentfirst}    % indent the first paragraph of a chapter
\usepackage[nopatch=item]{microtype}   % micro-typographic refinement
\usepackage{paralist}       % improved enumerate and itemize
\usepackage[nottoc]{tocbibind} % makes sure that bibliography and the lists
			    % of figures/tables are included in the table
			    % of contents
\usepackage{xcolor}         % typesetting in color

% The hyperref package for clickable links in PDF and also for storing
% metadata to PDF (including the table of contents).
% Most settings are pre-set by the pdfx package.
\hypersetup{unicode}
\hypersetup{breaklinks=true}

% Packages for computer science theses
\usepackage{algpseudocode}  % part of algorithmicx package
\usepackage{algorithm}
\usepackage{fancyvrb}       % improved verbatim environment
\usepackage{listings}       % pretty-printer of source code
\usepackage{quiver}
% You might want to use cleveref for references
% \usepackage{cleveref}

% Set up formatting of bibliography (references to literature)
% Details can be adjusted in macros.tex.
%
% BEWARE: Different fields of research and different university departments
% have their own customs regarding bibliography. Consult the bibliography
% format with your supervisor.
%
% The basic format according to the ISO 690 standard with numbered references
\usepackage[natbib,style=iso-numeric,sorting=none]{biblatex}
% ISO 690 with alphanumeric references (abbreviations of authors' names)
%\usepackage[natbib,style=iso-alphabetic]{biblatex}
% ISO 690 with references Author (year)
%\usepackage[natbib,style=iso-authoryear]{biblatex}
%
% Some fields of research prefer a simple format with numbered references
% (sorting=none tells that bibliography should be listed in citation order)
%\usepackage[natbib,style=numeric,sorting=none]{biblatex}
% Numbered references, but [1,2,3,4,5] is compressed to [1-5]
%\usepackage[natbib,style=numeric-comp,sorting=none]{biblatex}
% A simple format with alphanumeric references:
%\usepackage[natbib,style=alphabetic]{biblatex}

% Load the file with bibliography entries
\addbibresource{bibliography.bib}

% Our definitions of macros (see description inside)
\input macros.tex

%%% Title page and various mandatory informational pages
\begin{document}
%%% Title page of the thesis and other mandatory pages

%%% Inscriptions at the opening page of the hard cover

% We usually do not typeset the hard cover, but if you want to do it, change \iffalse to \iftrue
\iffalse

\pagestyle{empty}
\hypersetup{pageanchor=false}
\begin{center}

\large
Charles University

\medskip

Faculty of Mathematics and Physics

\vfill

{\huge\bf\ThesisTypeTitle}

\vfill

{\huge\bf\ThesisTitle\par}

\vfill
\vfill

\hbox to \hsize{\YearSubmitted\hfil \ThesisAuthor}

\end{center}

\newpage\openright
\setcounter{page}{1}

\fi

%%% Title page of the thesis

\pagestyle{empty}
\hypersetup{pageanchor=false}
\begin{center}

\centerline{\mbox{\includegraphics[width=166mm]{img/logo-en.pdf}}}

\vspace{-8mm}
\vfill

{\bf\Large\ThesisTypeTitle}

\vfill

{\LARGE\ThesisAuthor}

\vspace{15mm}

{\LARGE\bfseries\ThesisTitle\par}

\vfill

\Department

\vfill

{
\centerline{\vbox{\halign{\hbox to 0.45\hsize{\hfil #}&\hskip 0.5em\parbox[t]{0.45\hsize}{\raggedright #}\cr
Supervisor of the \ThesisTypeName{} thesis:&\Supervisor \cr
\ifx\ThesisType\TypeRig\else
\noalign{\vspace{2mm}}
Study programme:&\StudyProgramme \cr
\fi
}}}}

\vfill

Prague \YearSubmitted

\end{center}

\newpage

%%% A page with a solemn declaration to the thesis

\openright
\hypersetup{pageanchor=true}
\vglue 0pt plus 1fill

\noindent
I declare that I carried out this \ThesisTypeName{} thesis on my own, and only with the cited
sources, literature and other professional sources.
I understand that my work relates to the rights and obligations under the Act No.~121/2000 Sb.,
the Copyright Act, as amended, in particular the fact that the Charles
University has the right to conclude a license agreement on the use of this
work as a school work pursuant to Section 60 subsection 1 of the Copyright~Act.

\vspace{10mm}

\hbox{\hbox to 0.5\hsize{%
In \hbox to 6em{\dotfill} date \hbox to 6em{\dotfill}
\hss}\hbox to 0.5\hsize{\dotfill\quad}}
\smallskip
\hbox{\hbox to 0.5\hsize{}\hbox to 0.5\hsize{\hfil Author's signature\hfil}}

\vspace{20mm}
\newpage

%%% Dedication

\openright

\noindent
\Dedication

\newpage

%%% Mandatory information page of the thesis

\openright
{\InfoPageFont

\vtop to 0.5\vsize{
\setlength\parindent{0mm}
\setlength\parskip{5mm}

Title:
\ThesisTitle

Author:
\ThesisAuthor

\DeptType:
\Department

Supervisor:
\Supervisor, \SupervisorsDepartment

Abstract:
\Abstract

Keywords:
{\def\sep{\unskip, }\ThesisKeywords}

\vfil
}

% In Czech study programmes, it is mandatory to include Czech meta-data:

\ifx\StudyLanguage\LangCS

\vtop to 0.49\vsize{
\setlength\parindent{0mm}
\setlength\parskip{5mm}

Název práce:
\ThesisTitleCS

Autor:
\ThesisAuthor

\DeptTypeCS:
\DepartmentCS

Vedoucí bakalářské práce:
\Supervisor, \SupervisorsDepartmentCS

Abstrakt:
\AbstractCS

Klíčová slova:
{\def\sep{\unskip, }\ThesisKeywordsCS}

\vfil
}

\fi

}

\newpage

%%% Further pages will be numbered
\pagestyle{plain}


%%% A page with automatically generated table of contents of the thesis

\tableofcontents

%%% Each chapter is kept in a separate file
\chapter*{Introduction}
\addcontentsline{toc}{chapter}{Introduction}

Monoids are algebraic structures with a single associative binary operation and an identity element. 
They appear naturally in various areas of mathematics and computer science, such as automata theory, formal languages, and the theory of computation. 
Understanding the structure and properties of monoids can provide insight into these fields.
In this thesis, we explore the concept of monoidal actions, also known as acts, which work as an analogue to modules over rings.\par 

In the first chapter we recall some basic definitions and results from category theory, which we will refer to throughout the thesis.\par  
In the first section of the second chapter we give a definition of a pointed act and recall some basic properties of pointed acts. In the second section we show that the category of pointed acts is complete and 
cocomplete, define localisation of pointed acts and show some properties the localisation has. In the last section of the second chapter we look at a 
notion of exactness of pointed acts, define a chain complex of pointed acts and look at a property analogous to Dedekind groups, i.e. Rees simple acts.\par 
In the third chapter we state and prove various diagram lemmata for pointed acts, such as the four, five and the nine lemma, snake lemma and the long homology sequence for
exact chain complexes.\\ \\
\textbf{Previous results}\par 
    In an article by Y. Chen \cite{Chen02} a notion of exactness for pointed acts is defined and conditions are given for when such 
    short exact sequences split. In a PhD thesis by Jaret Flores \cite{Flores15} admissible morphisms, i.e. morphisms that are injective up to kernel,
    and admissible short exact sequences are defined. We note that the two notions in fact coincide, which follows immediately from proposition \ref{regularityProp}.\par 
    Another article by M. Jafari, et al. \cite{Jafari19} from $2019$ gives a proof of the four and the five lemma and shows, how various classes of 
    pointed acts are inherited through Rees short exact sequences.
\chapter{Preliminaries}
    We recall some notions from the theory of categories. This chapters exposition will loosely follow \cite{Leinster14}.
\begin{definition}
    A \textbf{category} $\catc$ consists of
    \begin{enumerate}
        \item a class of objects $\ob{\catc}$
        \item for each $A,B\in\ob{\catc}$ a class of morphisms $\homc{\catc}{A}{B}$ called the hom-set of $A$ and $B$\\
        We will write $f : A\to B$ or
$\begin{tikzcd}
A & B
\arrow["f", from=1-1, to=1-2]
\end{tikzcd}$ for $f\in\homc{\catc}{A}{B}$
        \item for each $A,B,C\in\ob{\catc}$ a class map $\circ$ : $\homc{\catc}{B}{C}\times\homc{\catc}{A}{B}\to \homc{\catc}{A}{C}$ such that
    for morphisms $\begin{tikzcd}
        A & B & C & D
        \arrow["f", from=1-1, to=1-2]
        \arrow["g", from=1-2, to=1-3]
        \arrow["h", from=1-3, to=1-4]
    \end{tikzcd}$ we have $h\circ(g\circ f) = (h\circ g)\circ f$
        \item for each object $A\in\ob{\catc}$ we have a morphism $1_A : A\to A$ satisfying
         for all morphisms $\begin{tikzcd}
            X & A & Y
            \arrow["f", from=1-1, to=1-2]
            \arrow["g", from=1-2, to=1-3]
        \end{tikzcd}$ the identity
        \[1_A \circ f = f \hspace{20pt} g \circ 1_A = g \]
        which we will call the identity morphism. 
    \end{enumerate}
    A category $\catc$ is \textbf{small}, if all hom-sets are sets and $\ob{\catc}$ is a set. \par
    We define the \textbf{opposite category} $\opp{\catc}$ by
    \[\ob{\opp{\catc}} = \ob{\catc}\]
    \[\homc{\opp{\catc}}{A}{B} = \homc{\catc}{B}{A}\]
    For $A,B,C\in\ob{\opp{\catc}}$ and morphisms $\begin{tikzcd}
	A & B & C
	\arrow["f", from=1-1, to=1-2]
	\arrow["g", from=1-2, to=1-3]
\end{tikzcd}$ we'll let $g\circop f = f\circ g$ and define the identity morphisms of $\opp{\catc}$ as the identity morphisms of $\catc$.
\end{definition}
\begin{definition}
        Let $\catc$ and $\catd$ be categories. We will say that $\catc$ is a \textbf{subcategory} of $\catd$, if 
        $\ob{\catc}$ is a subclass of $\ob{\catd}$, for each $A,B\in\ob{\catc}$ the hom-set
        $\homc{\catc}{A}{B}$ is a subclass of $\homc{\catd}{A}{B}$, the identity morphisms in $\catc$ are 
        the identity morphisms in $\catd$ and the composition $\circ$ in $\catc$ is the restriction of composition in $\catd$.
        If $\forall A,B\in\ob{\catc} : \homc{\catc}{A}{B} = \homc{\catd}{A}{B}$
        then we call $\catc$ a \textbf{full subcategory} of $\catd$.    
\end{definition}
\begin{definition}
    Let $\catc$ and $\catd$ be categories. A \textbf{functor} $F: \catc \to \catd$ consists of 
    \begin{enumerate}
        \item a class map $\ob{\catc}\to\ob{\catd}$ where $A \mapsto F(A)$
        \item for each $A,B\in\ob{\catc}$ a class map $\homc{\catc}{A}{B}\to\homc{\catd}{F(A)}{F(B)}$ where $f \mapsto F(f)$
    \end{enumerate}
    such that it preserves composition $F(g\circ f) = F(g)\circ F(f)$ and the identity $F(1_A) = 1_{F(A)}$.\par
    Given a functor $F: \catc\to\catd$ we define the \textbf{opposite functor} $F^{op}: \opp{\catc}\to\opp{\catd}$ as 
    $F^{op}(A) = F(A)$ for $A\in\ob{\opp{\catc}}$ and $F^{op}(f)=F(f)$ for a morphism $f$.\par
    A functor $F:\catc\to\catd$ is \textbf{faithful}, if it is injective on hom-sets.
    More precisely for all $A,B\in\ob{\catc}$ and each pair of morphisms $\begin{tikzcd}
        A & B
        \arrow["g"', shift right, from=1-1, to=1-2]
        \arrow["f", shift left, from=1-1, to=1-2]
    \end{tikzcd}$ if $F(f)=F(g)$ then $f=g$.\\ 
\end{definition}
\begin{definition}
    Let $f: A\to B$ be a morphism in some category, then we call $f$
    \begin{enumerate}
        \item an \textbf{isomorphism}, if there exists $f^{-1} : B\to A$ such that $f\circ f^{-1} = 1_B$ and $f^{-1}\circ f = 1_A$.
        \item a \textbf{split monomorphism}, if there exists $g: B\to A$ such that $g\circ f= 1_A$.
        \item a \textbf{split epimorphism}, if it is a split monomorphism in the opposite category.
        \item a \textbf{monomorphism}, if for all objects $X$ and $g,h: X \to A$ we have $f\circ g =f\circ h \implies g=h$.
        \item an \textbf{epimorphism}, if it is a monomorphism in the opposite category.
        \item a \textbf{bimorphism}, if it is both a monomorphism and an epimorphism.
    \end{enumerate}
    We call a category \textbf{balanced}, if every bimorphism in it is an isomorphism.
\end{definition}
\begin{definition}
    A category $\catc$ is \textbf{pointed}, if there exists an object $0\in\ob{\catc}$ such that for all $A\in\ob{\catc}$ there are 
    unique morphisms $0 \to A$ and $A \to 0$, such an object is necessarily unique up to a unique isomorphism and we call it the \textbf{zero object of $A$}. For objects $A,B$ in a 
    pointed category we define the \textbf{zero morphism} $0: A\to B$ to be the unique morphism, that factors through the zero object.
\end{definition}
\begin{definition}
    Given a small category $\mathbf{J}$ and a category $\catc$ we call any functor $F:\mathbf{J}\to\catc$ a \textbf{diagram in $\catc$}. We call the tuple
    $(A,(f_j : A \to F(j))_{j\in\mathbf{\ob{J}}})$ where $A\in\ob{A}$ a \textbf{cone of the diagram $F$}, if for all 
    $j_1,j_2\in\ob{\mathbf{J}}$ and $g: j_1 \to j_2$ we have $F(g)\circ f_{j_1} = f_{j_2}$.\par
    Let $F: \mathbf{J} \to\catc$ be a diagram and $(U, (f_j : U \to F(j))_{j\in\mathbf{\ob{J}}})$ a cone over $F$. We say the cone $(U, (f_j : U \to F(j))_{j\in\mathbf{\ob{J}}})$
    is \textbf{universal} or a \textbf{limit} of the diagram $F$, if given any other cone $(A,(g_j: A \to F(j))_{j\in\mathbf{\ob{J}}})$ 
    there exists a unique morphism $g: A \to U$ such that $g\circ f_j = g_j$ for all $j\in\ob{J}$.\par
    Given a diagram $F: \mathbf{J}\to\catc$ we define a \textbf{colimit} of $F$ as a limit of $F^{op}$ in $\opp{\catc}$.
\end{definition}
\begin{remark}
    If $(U, (f_j : U \to F(j))_{j\in\mathbf{\ob{J}}})$ and $(V, (g_j : V \to F(j))_{j\in\mathbf{\ob{J}}})$ are universal cones, 
    there exists a unique isomorphism $\phi: U \to V$ such that $g_j\circ\phi=f_j$, we will choose one universal cone from this isomorphism class 
    and call it the limit of diagram $F$ and similarly we can speak of the colimit of a diagram. In most categories there is a very canonical way to make this choice. 
\end{remark}
\begin{definition}
    We call $\mathbf{J}$ a \textbf{discrete} category, if the only morphisms in $\mathbf{J}$ are the identity morphisms.
    Given a diagram $F:\mathbf{J}\to\catc$ of a discrete category, we call its (co)limit the \textbf{(co)product of} $(F(j))_{j\in\mathbf{\ob{J}}}$.
\end{definition}
\begin{definition}
    If $\mathbf{J}$ is a category with two objects and two morphisms
    of the form 
        \[\begin{tikzcd}
        \bullet & \bullet
        \arrow["{j_1}", shift left, from=1-1, to=1-2]
        \arrow["{j_2}"', shift right, from=1-1, to=1-2]
    \end{tikzcd}\]
    we call its (co)limit the \textbf{(co)equaliser of $F(j_1)$ and $F(j_2)$}.
\end{definition}
\begin{definition}
    A category is called (co)complete, if it has all (co)limits.
\end{definition}
\begin{theorem}
    A category is (co)complete, if and only if it has all (co)equalisers and (co)products.
\end{theorem}
\begin{proof}
    See \cite[Proposition 5.1.26]{Leinster14}.
\end{proof}
\begin{definition}
    If $f$ is a morphism in a pointed category, we will call the \textbf{(co)kernel of $f$} the (co)equaliser of $f$ and the zero morphism, if it exists.
\end{definition}
\begin{definition}
    A monomorphism $f$ in a pointed category $\catc$ with kernels and cokernels is called normal, if $f$ is a kernel of its cokernel. An 
    epimorphism $f$ is called conormal, if it is a cokernel of its kernel. The category $\catc$ is then called 
    (co)normal, if every (epi)monomorphism 
    in it is (co)normal.
\end{definition}
\begin{definition}
    A tuple $(\catc,U)$, where $\catc$ is a category and $U:\catc\to\mathbf{Set}$ is a functor will be called a \textbf{concrete category}, if $U$ is faithful.
\end{definition}
    In a concrete category we say that a morphism $f$ is injective/surjective, if $U(f)$ is injective/surjective. It is easy to see 
    that any injective/surjective morphism in a concrete category is a mono/epimorphism.
\iffalse
\section{Algebras}
\begin{definition}
    Let $\Omega$ be a nonempty set and $\tau: \Omega\to\mathbb{N}_0$ a map, which we will call the \textbf{signature of $\Omega$}.
    We will call the tuple $(\Omega,\tau)$ an \textbf{algebra type}.
\end{definition}
\begin{definition}
    Let $(\Omega, \tau)$ be an algebra type, we call the tuple $(A,(F_\omega)_{\omega\in\Omega})$ an \textbf{algebra of type $(\Omega,\tau)$}, if
    $A$ is a set and for each $\omega\in\Omega$ we have $F_\omega : A^{\tau(\omega)}\to A$, we then say that $A$ is a universe of the algebra 
    and $F_{\omega}$ an operation of arity $\tau(\omega)$ on $A$. We will also, for readibility, denote $F_\omega$ as $\omega$ (keeping in mind, that we cannot exactly 
    identify $F_\omega$ with $\omega$).
\end{definition}
\begin{remark}
    We identify $A^0$ with the set $\{\emptyset\}$ and the map $e: A^0 \to A$ with its image $e(\emptyset)$. In case $\Omega$ is finite of cardinality $n$, we will denote 
    it as an ordered $n$-tuple $\Omega=(\omega_1,\dots,\omega_n)$ such that $\tau(\omega_1)\geq\dots\geq\tau(\omega_n)$ and we'll denote its signature as 
    $(\tau(\omega_1),\dots,\tau(\omega_n))$.
\end{remark}
\begin{definition}
    Let $(A,(F_\omega)_{\omega\in\Omega}),(B,(G_\omega)_{\omega\in\Omega})$ be two algebras of type $(\Omega,\tau)$ we say 
    $f:A \to B$ is a \textbf{homomorphism}, if 
    \[
        \forall\omega\in\Omega\,\forall x_1,\dots,x_{\tau(\omega)} : f(F_\omega(x_1,\dots,x_{\tau(\omega)})) = G_\omega(f(x_1),\dots,f(x_{\tau(\omega)})) 
    \]
\end{definition}
\begin{remark}
    Algebras of type $(\Omega,\tau)$ along with their homomorphisms form a category $\mathrm{Alg}_{(\Omega,\tau)}$.
\end{remark}
\begin{definition}
    Let $(A,(F_\omega)_{\omega\in\Omega})$ be an algebra of type $(\Omega,\tau)$ and equivalence relation $\sim$ on $A$ is a \textbf{congruence}
    on $A$, if for each $\omega\in\Omega$ and sequences $x_1,\dots,x_{\tau(\omega)},y_1,\dots,y_{\tau(\omega)}\in A$
    \[
        x_1\sim y_1,\dots, x_{\tau(\omega)}\sim y_{\tau(\omega)} \implies F_\omega(x_1,\dots,x_{\tau(\omega)}) \sim F_\omega(y_1,\dots,y_{\tau(\omega)})
    \]
\end{definition}
    If $f: A \to B$ is a homomorphism of algebras, we define a relation $\ker{f} = \{(x,y)\in A^2 : f(x)=f(y)\}$, such a relation is
    always a congruence on the algebra $A$ and we will call it the kernel congruence of $f$. On every algebra we also have the 
    smallest and the largest congruence $\Delta_A = \{(x,x)\in A^2 : x\in A\}$ (which we will call the diagonal relation) and $A\times A$ respectively. \par 
    The congruences on $A$ form a lattice, whose meet is defined to be the intersection and whose join is the intersection of all congruences 
    that contain the union of a given set of congruences (the least congruence generated by the union).
\begin{definition}
    Let $(A,(F_\omega)_{\omega\in\Omega})$ be an algebra of type $(\Omega,\tau)$ and $\sim$ a congruence on $A$. We define a 
    \textbf{factor algebra} $(A/\sim,(G_\omega)_{\omega\in\Omega})$, where we let
    \[
        G_\omega([x_1]_\sim,\dots,[x_\tau(\omega)]_\sim) = [F_\omega(x_1,\dots,x_{\tau(\omega)})]_\sim
    \] 
    these operations are well defined, since $\sim$ is a congruence on $A$.
\end{definition}
\begin{definition}
    Let $(A,(F_\omega)_{\omega\in\Omega})$,$(B,(G_\omega)_{\omega\in\Omega})$ be algebras of type $(\Omega,\tau)$. We call $B$ a \textbf{subalgebra 
    of $A$}, if $B\subseteq A$ and for each $\omega\in\Omega$ we have $F_\omega\restriction_B = G_\omega$.
\end{definition}
\fi
\chapter{Pointed acts}
\section{Basic notions}
\begin{definition}
    A pointed monoid is a $4$-tuple $(S,\ast,1,0)$, where $0,1\in S$ and $\ast: S\times S \to S$ a map, that satisfies the following
    \begin{enumerate}
        \item $\forall x,y,z\in S: (x\ast y)\ast z = x\ast(y\ast z)$
        \item $\forall x\in S: x\ast 1 = x = 1\ast x$
        \item $\forall x\in S: x\ast 0 = 0 = 0\ast x$
    \end{enumerate}
    we will call $1$ the \textbf{unit} of $(S,\ast,1,0)$ and $0$ the \textbf{zero} of $(S,\ast,1,0)$. We will often identify
    such a $4$-tuple with its underlying set $S$, if the structure is generic or clear from context. Often we will also write
    $xy$ instead of $x\ast y$ and omit unnecessary brackets.
\end{definition}
\begin{example}
    The most natural examples are $(\mathbb{N}_0, \cdot, 1,0)$ and $(\mathbb{N}_0\cup\{\infty\}, +,0,\infty)$. Given any ring $(R,+,-,0,\cdot,1)$ 
    we can also forget some of the structure and get a pointed monoid $(R,\cdot,1,0)$.  
\end{example}
\begin{example}
    In general given any monoid $(M,\cdot,1)$, we can consider the disjoint union $\{\bullet\}\times\{0\}\cup M\times\{1\}$ with any one element set
    $\{\bullet\}$ and extend the operation $\cdot$ as
    \[
        (m,1)\cdot(n,1) = (m\cdot n,1)
    \]
    \[
        (m,1)\cdot(\bullet, 0) = (\bullet, 0) = (\bullet,0)\cdot(m,1)
    \]
    \[
        (\bullet,0)\cdot(\bullet,0) = (\bullet,0)
    \]
    which gives us a pointed monoid, with the further property that $ab=0 \implies a=0 \lor b=0$. It can be shown that if pointed monoid 
    has this property, then it is isomorphic to this construction.
\end{example}

\begin{definition}
    Let $S$ be a pointed monoid. A (left) pointed $S$-act is a $3$-tuple $(A,\cdot, 0)$, where $0\in A$ and $\cdot: S\times A \to A$ 
    a map, that satisfies the following
    \begin{enumerate}
        \item $\forall r,s\in S\, \forall x\in A : (r\ast s)\cdot x = r\cdot(s\cdot x)$
        \item $\forall x\in A : 1\cdot x = x$
        \item $\forall x\in A : 0\cdot x = 0$
    \end{enumerate}
    it then follows that $r\cdot 0 = r\cdot (0\cdot 0) = (r\ast 0)\cdot 0 = 0\cdot 0 = 0$ and that 
    such an element $0$ satisfying $3.$ is necessarily unique. We will call the element $0\in A$ the 
    \textbf{sink} of $A$.
\end{definition}
\begin{example}
    Any pointed monoid can be considered as a pointed act over itself, where the action $\cdot$ is just the multiplication in the monoid. 
    Given a module $(M,+,-,0,\cdot)$ over a ring $(R,+,-,0,\cdot,1)$ we can view it as a pointed act $(M,\cdot,0)$ over the pointed monoid $(R,\cdot,1,0)$.
\end{example}
\begin{example}
    We can consider any pointed set $(X,x_0)$ (that is $X$ a set and $x_0\in X$) to be pointed $S$-act for any pointed monoid $S$,
    by letting $s\cdot x = x$ for $s\neq 0$ and $s\cdot x = x_0$ for $s=0$.
\end{example}
\begin{definition}
    Let $A$ be a pointed $S$-act. A \textbf{subact} of $A$ is a subset $B\subseteq A$ such that $0\in B$ and $B$ is closed under the action of $S$. 
    A \textbf{homomorphism} of pointed acts $f: A\to B$ is a map, such that $f(0) = 0$ and $f(r\cdot x) = r\cdot f(x)$ for all $r\in S$ and $x\in A$.
    A \textbf{congruence} on a pointed act $A$ is an equivalence relation $\rho$ on $A$ such that 
    \[
        \forall x,y\in A \,\forall r \in S: (x,y)\in\rho \implies (r\cdot x, r\cdot y)\in\rho
    \]
    A \textbf{factor} of a pointed act $A$ by a congruence $\rho$ is the set of equivalence classes $A/\rho$ with the action defined as $r \cdot [a]_\rho = [r \cdot a]_\rho$ and the zero element $[0]_\rho$. 
    This construction yields a pointed act.
\end{definition}
\begin{example}
    The kernel congruence of a homomorphism $f: A\to B$ is a congruence on $A$ defined as $\ker{f} = \{(x,y)\in A\times A \mid f(x) = f(y)\}$.
\end{example}
\begin{remark}
    The category of pointed acts over a pointed monoid $S$, denoted as $\actcat{S}$ consists of pointed acts as objects and their homomorphisms as morphisms.
    The composition of morphisms and the identity morphisms are defined as in the category of sets, clearly a composition of two homomorphisms yields a homomorphism
    and the identity is trivially a homomorphism.
\end{remark}
\begin{definition}
    Let $A\in S\mathrm{-Act}_0$ and $B$ a subact of $A$, then we let 
    \[
        \rho_B = \Delta_A \cup B\times B
    \]
    where $\Delta_A= \{(x,x): x\in A \}$ is the diagonal relation on $A$. We call $\rho_B$ a \textbf{Rees congruence on $A$}. 
\end{definition}
\begin{proof}
    We will show, that $\rho_B$ is a congruence. \par 
    Reflexivity and symmetry of the relation is clear from definition. If $(x,y)\in\rho_B$ and $(y,z)\in\rho_B$, we either have 
    $x=y=z \implies (x,z)\in\rho_B$ or at least one of the pairs is in $B\times B$, if both are then again $(x,z)\in\rho_B$. If 
    say, WLOG, that $x=y$ and $y,z\in B$, then also $x\in B$ and therefore again $(x,z)\in\rho_B$, thus the relation is transitive. 
    If $(x,y)\in\rho_B$ and $r\in S$ we have either $x=y\implies rx=ry$ or $x,y\in B \implies rx,ry\in B$, so $\rho_B$ is a congruence on $A$.
\end{proof}
\begin{remark}
    For $B$ a subact of $A$ we will often denote the factor $A/\rho_B$ as $A/B$. 
\end{remark}

\section{Constructions}
\begin{proposition}
    Let $f,g: A \to B$ be homomorphisms of pointed acts. Let $\{f=g\} = \{x\in A \mid f(x) = g(x)\}$, then $\{f=g\}$ is a subact of $A$ and 
    the canonical inclusion $\iota: \{f=g\} \to A$, $x\mapsto x$ is an equaliser in $\actcat{S}$.
\end{proposition}
\begin{proof}
    We clearly have that $0\in\{f=g\}$ and $x\in\{f=g\}$ implies $f(x)=g(x)$, hence $f(rx)=rf(x)=rg(x)=g(rx) \implies rx\in\{f=g\}$. 
    Suppose $h: X \to A$ is such that $fh=gh$, then we have that $h(x)\in\{f=g\}$, so we simply let $\overline{h} = h: X\to\{f=g\}$, this
    map is unique. Say there were two maps $\iota\overline{h_1}=h=\iota\overline{h_2}$, then because $\iota$ is injective (so a monomorphism, 
    because we are in a concrete category) we have $\overline{h_1}=\overline{h_2}$.
\end{proof}
    We will denote the equalizer $\{f=0\}$ as $\Kernel{f}$. 
\begin{proposition}
    Suppose $f,g: A\to B$ are homomorphisms, let $\sim$ be the smallest congruence on $B$ such that $\forall x\in A: f(x)\sim g(x)$, then 
    $B/\sim$ with the canonical projection $\pi: B\to B/\sim$, $x\mapsto [x]_\sim$ is a coequaliser in $\actcat{S}$. 
\end{proposition}
\begin{proof}
    Suppose that $h: B\to X$ is such that $hf=hg$, then we have that $\ker{h}$ is a congruence on $B$, satisfying 
    $\forall x\in A: (f(x),g(x))\in\ker{h}$, therefore $\sim\subseteq\ker{h}$. We need a morphism $\overline{h}: B/\sim \to X$, which satisfies 
    $h = \overline{h} \circ\pi$. The uniqueness is clear, if it exists, since then $\overline{h_1}\circ\pi = h = \overline{h_2}\circ\pi$, 
    which implies $\overline{h_1}=\overline{h_2}$, because $\pi$ is epi. We define $\overline{h}([x]_\sim) = h(x)$, this 
    is clearly a homomorphism of acts and it is well-defined because
    \[
        [x]_\sim = [y]_\sim\iff x\sim y \implies (x,y)\in\ker{h} \implies h(x)=h(y)
    \] 
\end{proof}
    In the case of cokernels, the congruence $\sim$ admits a much nicer description. 
\begin{proposition}
    Let $f: A \to B$ be a homomorphism of pointed acts, then the smallest congruence on $B$ such that $\forall x\in A :f(x)\sim 0$ is
    $\im(f)=\rho_{\Image{f}}$, where $\Image{f} = \{f(x) \mid x\in A\}$.
\end{proposition}
\begin{proof}
    We have that $(f(x),0)\in\Image{f}$, because $0\in\Image{f}$. If $\sim$ is any other such congruence, then for any $f(x),g(x)\in\Image{f}$
    it holds that $f(x)\sim 0 \sim g(x) \implies f(x)\sim g(x)$, therefore $\rho_{\Image{f}}=\Image{f}\times\Image{f}\cup\Delta_B\subseteq\sim$.
\end{proof}
We will now construct the products and coproducts in $\actcat{S}$. 
\begin{proposition}
    Let $I$ be a nonempty set, $(X_i)_{i\in I}$ a collection of pointed acts over a pointed monoid $S$, then we define 
    \[
        \prod_{i\in I}X_i = \{f: I \to\bigcup_{i\in I } X_i \mid\forall i\in I :  f(i)\in X_i \}
    \]
    then $\prod_{i\in I}X_i$ admits a structure of a pointed act and along with the canonical projections 
    $\pi_i : \prod_{i\in I }X_i$,$f\mapsto f(i)$ it is the product in the category $\actcat{S}$.
\end{proposition}
\begin{proof}
    We define the action of $S$ on the product componentwise as $(r\cdot f)(i) = r\cdot f(i)$. The zero in the product 
    is the map $i\mapsto 0$. It is routine to check that such a structure satisfies the 
    axioms of pointed acts. By definition of the action and zero we see that $\pi_i$ is a homomorphism. Suppose now that 
    object $U$ along with the maps $\mu_i : U \to X_i$ is a cone in $\actcat{S}$. Define a map $f: U \to\prod_{i\in I} X_i$ 
    as $f(u) = g$, where $\forall i\in I : g(i) = \mu_i(u)$. We have that 
    \[
        f(ru)(i) = \mu_i(ru) = r\mu_i(u) = r\cdot f(u)(i) = (r\cdot f(u))(i)
    \]  
    \[
        f(0)(i) = \mu_i(0) = 0 \implies f(0) = 0
    \]
    so it is a homomorphism of pointed acts that satisfies $\mu_i = \pi_i\circ f$. If $f'$ is any other such map we would have that 
    $\forall i\in I : \pi_i\circ f = \pi_i\circ g$, then fixing $u\in U$ and $i\in I$ we get $g(u)(i) = (\pi_i \circ g)(u) = (\pi_i\circ f)(u) = f(u)(i) \implies g(u)=f(u)$ 
    thus $f=g$.
\end{proof}
\begin{proposition}
    Let $I$ be a nonempty set, $(X_i)_{i\in I}$ a collection of pointed acts over a pointed monoid $S$, we define 
    \[
        \coprod_{i\in I } X_i = \{f\in\prod_{i\in I }X_i \mid \exists j\in I\,\forall i\in I : i\neq j \implies f(i) = 0\}
    \]
    then $\coprod_{i\in I} X_i$ is a subact of the product and along with the canonical inclusions $\iota_i : X_i \to \coprod_{i\in I} X_i$, $x\mapsto f$, where 
    $f(i) = x$ and $f(j)=0$ for $j\neq i$
    forms the coproduct in $\actcat{S}$.
\end{proposition}
\begin{proof}
    Clearly $0\in\coprod_{i\in I } X_i$ and given any $r\in S$ and $f\in\coprod_{i\in I } X_i$, where 
    $f(j)=0$ for $j\neq i$, then $r\cdot f(j)=0$, hence $r\cdot f \in\coprod_{i\in I} X_i$. It follows that 
    $\coprod_{i\in I} X_i$ is a subact of the product.\par 
    Let $U$ be a pointed act and $\nu_i : X_i \to U$, then let $h: \coprod_{i\in I} X_i \to U$ be defined as follows - 
    given any $f$ and $i\in I$, such that $f(j)=0$ for all $j\neq i$, define $h(f) = \nu_i(f(i))$ (Note that this is well-defined 
    even in the case of zero, because $\forall i\in I : \nu_i(0) = 0$). By definition we have that $h\circ \iota_i = \nu_i$. It is 
    indeed a homomorphism, since $r\cdot h(f) = r\cdot\nu_i(f(i))=\nu_i((r\cdot f)(i)) = h(r\cdot f)$. Let 
    $\overline{h} : \coprod_{i\in I } X_i \to U$ be any other morphism, such that $\forall i\in I :\overline{h}\circ\iota_i = \nu_i$ we see, given 
    any $f$, letting $x=f(i)$, where $i\in I$ is such that $\forall j\neq i : f(j)=0$, that 
    \[
        h(f) = \nu_i(x) = \overline{h}(\iota_i(x)) = \overline{h}(f) \implies h=\overline{h} 
    \]
\end{proof}
\begin{corollary}
    The category $\actcat{S}$ is complete and cocomplete.
\end{corollary}
    Our category $\actcat{S}$ equipped with the "forgetful" functor $U$, which maps objects to their underlying sets and morphisms 
    to their underlying set maps is a concrete category.
\begin{proposition}
    The monomorphisms in $\actcat{S}$ are precisely the injective homomorphisms and the epimorphisms in $\actcat{S}$ are precisely
    the surjective homomorphisms.
\end{proposition}
\begin{proof}
    See \cite[Proposition~6.15]{Kilp00}.
\end{proof}

\begin{proposition}
    Let $(S,\cdot,0,1)$ be a pointed commutative monoid and 
    $K\subseteq S$ a submonoid of $S$ such that $0\not\in K$. 
    We define $K^{-1}S = S\times K/\sim$, where 
    \[
        (s_1,k_1)\sim(s_2,k_2)\iff \exists v\in K : s_1k_2v = s_2k_1v 
    \]
    then $K^{-1}S$ is well-defined and admits the structure of a pointed commutative monoid.
\end{proposition}
\begin{proof}[Proof]
    Relation $\sim$ is reflexive, since $1\in K$. It is clearly symmetric. 
    Let $(s_1,k_1)\sim(s_2,k_2)\sim(s_3,k_3)$, so there are $v,w\in K$ such that 
    $s_1k_2v = s_2k_1v$ and $s_2k_3w = s_3k_2w$, then 
    \[
    (s_1k_3)k_2vw = s_1k_2vk_3w = s_2k_1vk_3w = s_2k_3wk_1v = s_3k_2wk_1v = (s_3k_1)k_2vw
    \]
    therefore $(s_1,k_1)\sim(s_3,k_3)$. We will denote the equivalence classes as
    $\frac{s}{k} = [s,k]_{\sim}$. Define 
    \[
        \frac{s_1}{k_1}\cdot\frac{s_2}{k_2} = \frac{s_1 s_2}{k_1 k_2}
    \]
    to prove well-definedness assume $(s_1,k_1)\sim(d_1,l_1)$ and $(s_2,k_2)\sim(d_2,l_2)$, then 
    we find $u,v\in K$ such that 
    $s_1l_1u=d_1k_1u$ and $s_2l_2v = d_2k_2v$, then 
    \[
    s_1s_2l_1l_2uv = (s_1l_1u)(s_2l_2)v = d_1k_1ud_2k_2v = d_1d_2k_1k_2uv 
    \]
    thus
    \[
        \frac{s_1}{k_1}\cdot\frac{s_2}{k_2} = \frac{s_1s_2}{k_1k_2} = \frac{d_1d_2}{l_1l_2} = \frac{d_1}{l_1}\cdot\frac{d_2}{l_2}
    \]
    We define the unit $1 = \frac{1}{1}$ and $0 = \frac{0}{1}$. The associativity, commutativity and unit axioms are clearly 
    satisfied. We have 
    \[
        0\cdot\frac{s}{k} = \frac{0}{k} = \frac{0}{1} = 0
    \]
    since $(0,k)\sim(0,1)$, because $0\cdot 1 = 0 = 0\cdot k$. 
\end{proof}
\begin{definition}
    We define the \textbf{localisation map} $\Lambda: S \to K^{-1}S$ by $\Lambda(x) = \frac{x}{1}$. It is a 
    homomorphism of pointed monoids.  
\end{definition}
\begin{proposition}
    Let $K\subseteq S$ be a submonoid of a commutative pointed monoid $S$ such that $0\not\in S$ and $A$ an $S$ act. We define 
    $K^{-1}A = A\times K/\sim$, where 
    \[
        (a_1,k_1)\sim(a_2,k_2) \iff \exists u\in K : uk_1 a_2 = uk_2 a_1
    \]
    then $K^{-1}A$ admits the structure of a pointed $K^{-1} S$ act.
\end{proposition}
\begin{proof}[Proof]
    The relation is clearly reflexive and symmetric. Assume 
    that $(k_1,a_1)\sim(k_2,a_2)\sim(k_3,a_3)$, so 
    $uk_1a_2 = uk_2a_1$ and $vk_2a_3 = vk_3a_2$ for some $u,v\in K$, then 
    \[
        uvk_2(k_1a_3) = uk_1vk_2a_3 = uk_1vk_3a_2 = vk_3uk_1a_2 = vk_3uk_2a_1 = uvk_2(k_3a_1)
    \] 
    thus the relation is an equivalence. Denote the equivalence classes as $\frac{a}{k} = [a,k]_\sim$. 
    Define the action of $K^{-1}S$ on $K^{-1}A$ by 
    \[
        \frac{s_1}{k_1}\cdot\frac{a}{k} = \frac{s_1 a}{k_1 k}
    \]
    to show well-definedness assume $\frac{s_1}{k_1}=\frac{s_2}{k_2}$ and $\frac{a_1}{l_1}=\frac{a_2}{l_2}$, then 
    we find $u,v\in K$ such that $s_1k_2u = s_2k_1u$ and $vl_1a_2 = vl_2a_1$, then we have 
    \[
        uvk_2l_2s_1a_1 = s_1k_2uvl_2a_1 = s_2k_1uvl_1a_2 = uvk_1l_1s_2a_2 \implies \frac{s_1a_1}{k_1l_1} = \frac{s_2a_2}{k_2l_2}
    \]
    so 
    \[
        \frac{s_1}{k_1}\cdot\frac{a_1}{l_1} = \frac{s_1a_1}{k_1l_1} = \frac{s_2a_2}{k_2l_2} = \frac{s_2}{k_2}\cdot\frac{a_2}{l_2}
    \]
    which proves that the map is well defined. We have to show that it is an action. 
    The associativity and unit axiom is trivial. Define $0 = \frac{0}{1}$, then 
    \[
        0\cdot\frac{a}{k} = \frac{0}{1}\cdot \frac{a}{k} = \frac{0}{k} = \frac{0}{1} = 0
    \]
    since $k\cdot 0 = 1\cdot 0$. Hence it is really a pointed act over $K^{-1}S$. 
\end{proof}
\begin{proposition}
    Let $f: A\to B$ be an act homomorphism, then we have an induced 
    map $K^{-1}f : K^{-1}A \to K^{-1}B$ defined as 
    \[
        K^{-1}f\left(\frac{a}{k}\right) = \frac{f(a)}{k}
    \] 
    it is a well-defined homomorphism of acts.
\end{proposition}
\begin{proof}[Proof]
    Assume $\frac{a_1}{k_1}=\frac{a_2}{k_2}$, so we find $u\in K$ such that 
    $uk_2a_1= uk_1a_2$, then 
    \[
        uk_2f(a_1) = f(uk_2a_1) = f(uk_1a_2) = uk_1f(a_2)\] 
        \[ \implies K^{-1}f\left(\frac{a_1}{k_1}\right)=\frac{f(a_1)}{k_1}=\frac{f(a_2)}{k_2} = K^{-1}f\left(\frac{a_2}{k_2}\right)
    \]
    and 
    \[
    \frac{s_1}{k_1}\cdot K^{-1}f\left(\frac{a}{k}\right) = \frac{s_1f(a)}{k_1k} = \frac{f(s_1a)}{k_1k} = 
    K^{-1}f\left(\frac{s_1a}{k_1k}\right) = K^{-1}f\left(\frac{s_1}{k_1}\cdot\frac{a}{k}\right)
    \]
\end{proof}
\begin{proposition}
    We have a localisation functor $K^{-1}:\actcat{S}\to\actcat{K^{-1}S}$.
\end{proposition}
\begin{proof}[Proof]
    Choose $\frac{a}{k}$, then
    \[
    K^{-1}(g\circ f)\left(\frac{a}{k}\right) = \frac{g(f(a))}{k} = K^{-1}g\left(\frac{f(a)}{k}\right) =\] \[=(K^{-1}g\circ K^{-1}f)(\frac{a}{k}) \implies K^{-1}(g\circ f) = K^{-1}g\circ K^{-1}f
    \]
    and 
    \[
        K^{-1}1_A\left(\frac{a}{k}\right) = \frac{a}{k} = 1_{K^{-1}A}(\frac{a}{k}) \implies K^{-1}1_A = 1_{K^{-1}A}
    \]
\end{proof}
\begin{remark}
    The localisation map $\Lambda$ induces a functor (by restriction of scalars) from $\actcat{K^{-1}S}$ to $\actcat{S}$, where given any 
    $A$ a $K^{-1}S$ act we define the action of $S$ on $A$ by $s\cdot a = \Lambda(s)\cdot a$. The functor maps the 
    homomorphisms to themselves. This way we can consider the 
    localisation to be an endofunctor by considering the functor $\Lambda\circ K^{-1}$ instead. 
\end{remark}
\begin{remark}
    If $B\subseteq A$ is a subact of $A$, then $K^{-1} B = \{ x/k : x\in B \}$ is a subact of $K^{-1} A$. Indeed, $0\in K^{-1} B$ and if 
    $x/k\in K^{-1} B$, then for any $r/s\in K^{-1} S$ we have $(rx)/(sk)\in K^{-1} B$, because $rx\in B$.
\end{remark}
\begin{definition}
    If $A$ is a pointed act over a pointed monoid $S$ and there exists a subset $\{x_1,\dots,x_n\}\subseteq A$ such that 
    $\forall x\in A \exists i\in\{1,\dots,n\} \exists r\in S : x = r\cdot x_i$ we say that $A$ is \textbf{finitely generated} 
    and that the set $\{x_1,\dots,x_n\}$ is a \textbf{generating set} of $A$.
\end{definition}
\begin{proposition}
    Let $S$ be a commutative, pointed monoid and $K\subseteq S$ is a submonoid of $S$ such that $0\not\in K$.
    If $A$ is a finitely generated pointed act over $S$, then $K^{-1} A$ is.
\end{proposition}
\begin{proof}
    Let $\{x_1,\dots,x_n\}$ be a generating set of $A$. Let $\frac{y}{k}\in K^{-1} A$. We have $y=r\cdot x_i$ for some $i\in\{1,\dots, n\}$,
    then $\frac{y}{k} = \frac{r}{k}\cdot\frac{x_i}{1}$, therefore 
    \[
        \left\{\frac{x_1}{1},\dots,\frac{x_n}{1}\right\}
    \]
    is a generating set of $K^{-1} A$ a finitely generated over $K^{-1}S$.
\end{proof}
\section{Regularity and exactness}
    For the remaining chapters we fix the following notation, for a homomorphism $f$ by $\im(f)$ we mean the 
    Rees congruence $\rho_{\Image{f}}$. To remind the reader by $\ker{f}$ we mean the kernel congruence $\ker{f} = \{(x,y) \mid f(x)=f(y)\}$
    and $\Kernel{f} = \{x \mid f(x) = 0\}$, $\Image{f} = \{ y \mid \exists x : f(x) =y \}$.
\begin{proposition}
    In the category of pointed acts, every monomorphism is normal (i.e. a kernel of its cokernel). 
\end{proposition}
\begin{proof}[Proof]
    Let $f: K \to A$ be a monomorphism and $\pi : A \to A/\Image{f}$ be the canonical projection. We clearly have that 
    $\pi\circ f = 0$. Let $\hat{f} : S \to A$ be any other homomorphism, such that $\pi\circ\hat{f} = 0$, then 
    given $x\in S$ we have $[\hat{f}(x)]=0$ or equivalently that $\hat{f}(x)\in\Image{f}$, therefore there is a unique 
    $x'\in K$ so that $f(x')=\hat{f}(x)$. Define $\phi(x)=x'$ to be the map from $S$ to $K$. By definition we have 
    $f\circ\phi = \hat{f}$ and this relation defines the map uniquely, because for every element in the image of $\hat{f}$ there 
    is precisely one element in $K$ that can map onto it. It remains to show that $\phi$ is really a homomorphism, fixing the 
    notation as above, let $\phi(rx)=z'$ and $\phi(x)=z_0'$, then we have 
    \[
        f(\phi(rx))=f(z')=\hat{f}(rx)=r\hat{f}(x)=rf(z_0')=f(rz_0')=f(r\phi(x))
    \]
    thus $\phi(rx)=r\phi(x)$, since $f$ is a mono, which concludes the proof.
\end{proof}

\begin{definition}
    Let $A$ and $B$ be pointed acts over a pointed monoid $S$, we say that 
    $f: A\to B$ is \textbf{Rees regular}, if 
    \[
        \rho_{\Kernel{f}} = \ker{f}
    \]
    In diagrams we will denote the fact that $f$ is Rees regular as 
    \[\begin{tikzcd}
	A & B
	\arrow["f", "\circ"{marking}, from=1-1, to=1-2]
\end{tikzcd}\]
\end{definition}
\begin{remark}
    Note that we always have $\rho_{\Kernel{f}}\subseteq\ker{f}$, since $x=y\lor f(x)=0=f(y)\implies f(x)=f(y)$.
\end{remark}
\begin{proposition}\label{regChar}
    Let $f: A\to B$ be a homomorphism of pointed acts over a pointed monoid $S$, then the following are equivalent
    \begin{enumerate}
        \item $\rho_{A'} = \ker{f}$ for some subact $A'\subseteq A$.
        \item $f$ is Rees regular.
        \item $f: A \to\Image{f}$ is a conormal epimorphism.
    \end{enumerate}
\end{proposition}
\begin{proof}
    $1.\implies 2.$ Let $x\in A'$, then $(x,0)\in\ker{f} \implies f(x)=0 \implies x \in\Kernel{f}$. If $f(x)=0$, then 
    $(x,0)\in\rho_A' \implies x\in A'$. \par
    $2.\implies 3.$ Assume $\rho_{\Kernel{f}} = \ker{f}$ and let $g: A\to X$ be a homomorphism such that $g\circ\iota = 0$, where
    $\iota: \Kernel{f}\to A$ is the canonical inclusion. We need to show that there exists a unique homomorphism $\phi: \Image{f}\to X$ such that
    the following diagram commutes 
    \[\begin{tikzcd}
        && X \\
        {\Kernel{f}} & A && {\Image{f}}
        \arrow["\iota", hook, from=2-1, to=2-2]
        \arrow["g", from=2-2, to=1-3]
        \arrow["f", from=2-2, to=2-4]
        \arrow["{\exists ! \phi}"', dashed, from=2-4, to=1-3]
    \end{tikzcd}\]
    Define $\phi(y) = g(x)$, where $x\in A$ is such that $f(x)=y$. This map is well-defined, because if $f(x)=y=f(x')$, then
    either $x=x'$ or $f(x)=0=f(x') \implies g(x)=0=g(x')$, since $g\circ\iota = 0$. Clearly $\phi\circ f = g$ by definition.
    It remains to verify uniqueness. Suppose that $\phi': \Image{f}\to X$ is another such map, then for any $y\in\Image{f}$ we 
    find a $x\in A$ so that $f(x) = y$, then $\phi'(y) = \phi'(f(x)) = g(x) = \phi(f(x)) = \phi(y)$. \par
    $3. \implies 1.$ Consider the canonical projection $\pi : A \to A/\Kernel{f}$, then $\pi\circ\iota = 0$, therefore 
    there exists a unique map $\phi: \Image{f}\to X$, such that $\phi\circ f = \pi$, then if $f(x)=f(y)$ we have 
    $\phi(f(x))=\phi(f(y)) \implies [x]_\Kernel{f}=[y]_\Kernel{f}$, hence $x=y \lor x,y\in\Kernel{f}$, so 
    $\ker{f}=\rho_{\Kernel{f}}$. In particular $1.$ holds.
\end{proof}
\begin{remark}
    Note that $3.$ in the above characterisation is the definition of admissible morphism in $\actcat{S}$, 
    which appears in \cite{Flores15}.
\end{remark}
\begin{proposition}\label{regPropChar}
    Assume $f:A \to B$ is Rees regular and $\sim$ a congruence on $A$, the following holds
    \begin{enumerate}
        \item $f$ is a monomorphism, if and only if $\Kernel{f} = \{0\}$.
        \item If $K$ is a multiplicative subset of $S$, then $K^{-1}f$ is Rees regular.
        \item If $g: B \to C$ is Rees regular, then $g\circ f$ is.
        \item The canonical projection $\pi : A \to A/\sim$, $x\mapsto [x]$ is Rees regular, if and only if 
        $\sim$ is a Rees congruence.
    \end{enumerate}
\end{proposition}
\begin{proof}[Proof]
    Let $f(x)=f(y)$, then $(x,y)\in\rho_{\Kernel{f}} \implies x=y \lor f(x)=0=f(y) \implies x=y \lor x=0=y$.\par 
    Assume that 
    \[
        K^{-1}f\left(\frac{a}{k_1}\right) = K^{-1}f\left(\frac{b}{k_2}\right)
    \]
    then $f(kk_2a)=kk_2f(a) = kk_1f(b)=f(kk_1b)$ for some $k\in K$, so $(kk_2 a,kk_1b)\in\rho_{\Kernel{f}}$, which implies 
    \[
        \frac{a}{k_1}=\frac{b}{k_2} \lor K^{-1}f\left(\frac{a}{k_1}\right)= \frac{f(a)}{k_1} = 0 = \frac{f(b)}{k_2} = K^{-1}f\left(\frac{b}{k_2}\right) 
    \]
    Assume that $g: B\to C$ is Rees regular and $g(f(x))=g(f(y))$, since $g$ is Rees regular we have 
    \[
        f(x)=f(y) \lor g(f(x))=0=g(f(y)) \implies 
        \]
        \[\implies x=y \lor f(x)=0=f(y) \lor g(f(x))=0=g(f(y)) \implies
    \]
    \[
        \implies x=y \lor g(f(x))=0=g(f(y))
    \]
    hence $\ker{g\circ f} = \rho_{\Kernel{g\circ f}}$.\par
    Suppose the canonical projection is Rees regular, if $x\sim y$, then $[x]=[y]$, hence by regularity 
    $x=y \lor [x]=[0]=[y]$ or equivalently $x=y\lor x,y\in [0]$. Note that $[0]\leq A$, since $x\sim 0 \implies rx\sim r0 = 0$ 
    therefore it is a subact of $A$. If $\sim$ is a Rees congruence we have $[x]=[y] \implies x=y \lor x,y\in A'$ for some $A'\leq A$,
    which means precisely that $x=y \lor [x]=[0]=[y]$, because $0\in A'$. 
\end{proof}
\begin{definition}
    A \textbf{chain complex} $(A_n,a_{n} : A_n\to A_{n-1})_{n\in\mathbb{Z}}$ is a sequence of acts and act morphisms
    \[\begin{tikzcd}
        \cdots & {A_{n+1}} & {A_n} & {A_{n-1}} & \cdots
        \arrow[from=1-1, to=1-2]
        \arrow["{a_{n+1}}", from=1-2, to=1-3]
        \arrow["{a_n}", from=1-3, to=1-4]
        \arrow[from=1-4, to=1-5]
    \end{tikzcd}\]
    such that for all $n\in\mathbb{Z}$ we have $\im(a_{n+1})\subseteq\ker{a_n}$. Chain complex is \textbf{Rees exact
    at $A_n$}, if $\im(a_{n+1})=\ker{a_n}$. Chain complex is \textbf{Rees exact}, if it is Rees exact at $A_n$ for each $n\in\mathbb{Z}$. 
    We say that a chain complex is \textbf{Rees regular}, if all $a_n$ are Rees regular homomorphisms.
\end{definition}
We will often write $(A,a)$ for short to mean the chain complex $(A_n,a_{n} : A_n\to A_{n-1})_{n\in\mathbb{Z}}$. \par
Given any finite sequence of acts $\begin{tikzcd}
	{A_0} & \cdots & {A_n}
	\arrow["{f_0}", from=1-1, to=1-2]
	\arrow["{f_{n-1}}", from=1-2, to=1-3]
\end{tikzcd}$ we say that the sequence is Rees exact, if $\forall n\in\{1,\dots,n-1\} : \ker{d_n} = \im(d_{n-1})$.
\begin{remark} 
    A sequence $(A_n,a_{n} : A_n\to A_{n-1})_{n\in\mathbb{Z}}$ of pointed acts and their morphisms is a chain complex, if and only if $a_{n+1}\circ a_n = 0$. If $(A,a)$ is a chain complex 
    we always have that $\Image{a_{n+1}}\subseteq\Kernel{a_n} \forall n\in\mathbb{Z}$.
\end{remark}
\begin{proof}[Proof]
    Choose $n\in\mathbb{Z}$. 
    If $\im(a_{n+1})\subseteq\ker{a_n}$ we have for $x\in A_{n+1}$ that
    $(a_{n+1}(x),0)\in\im(a_{n+1})\subseteq\ker{a_n}$, so $(a_n\circ a_{n+1})(x) = 0$. Now assume that $a_{n+1}\circ a_n = 0$, then 
    if $(x,y)\in\im(a_{n+1})$ we either have $x=y$, which trivially implies that $a_n(x)=a_n(y)$ or 
    $\exists x_0,y_0\in A_{n+1}$ such that $a_{n+1}(x_0)=x\land a_{n+1}(y_0)=y$, which implies 
    \[
        a_n(x) = a_n(a_{n+1}(x_0)) = 0 = a_n(a_{n+1}(y_0)) = a_n(y)
    \]
    so $(x,y)\in\ker{a_n}$.\par
    If $y\in A_n$ is such that $a_{n+1}(y_0)=y$ for some $y_0\in A_{n+1}$ we 
    have $a_n(y) = a_n(a_{n+1}(y_0)) = 0 \implies y\in\Kernel{a_n}$.
\end{proof}
\begin{definition}
    Let $(A,a)$ be a chain complex and $n\in\mathbb{Z}$ we define the $n$-th homology act $H_n(A)$ as 
    \[
        H_n(A) = \Kernel{a_n}/\im(a_{n+1})
    \]
\end{definition}
\begin{proposition}\label{regularityProp}
    Given a sequence of pointed acts 
    \[\begin{tikzcd}
        \cdots & {A_{n+1}} & {A_n} & {A_{n-1}} & \cdots
        \arrow[from=1-1, to=1-2]
        \arrow["{a_{n+1}}", from=1-2, to=1-3]
        \arrow["{a_n}", from=1-3, to=1-4]
        \arrow[from=1-4, to=1-5]
    \end{tikzcd}\]
    the following are equivalent for each $n\in\mathbb{Z}$
    \begin{enumerate}
        \item $\im(a_{n+1})=\ker{a_n}$
        \item $\Image{a_{n+1}}=\Kernel{a_n}$ and $a_n$ is Rees regular.
    \end{enumerate}
\end{proposition}
\begin{proof}[Proof]
    Assume exactness at $A_n$ and let $a_n(x)=0$, then $(x,0)\in\im(a_{n+1})$, if 
    $x=0$, then trivially $a_{n+1}(0)=x$, else we find $x_0\in A_{n+1},x_0\neq 0$ such that $a_{n+1}(x_0)=x$, thus
    $\Kernel{a_n}=\Image{a_{n+1}}$ by the previous remark. Let $a_n(x)=a_n(y)$,$x\neq y$, then $(x,y)\in\ker{a_n}=\im(a_{n+1})$, 
    so $a_{n+1}(x_0)=x$ and $a_{n+1}(y_0)=y$ for some $x_0,y_0\in A_{n+1}$, therefore 
    \[
        a_n(x)= a_n(a_{n+1}(x_0)) = 0 = a_n(a_{n+1}(y_0)) = a_n(y) \implies (x,y)\in\rho_\Kernel{a_n}
    \]
    Assume that the second condition holds. We have $\ker{a_n} = \rho_{\Kernel{a_n}} = \rho_{\Image{a_{n+1}}} = \im(a_{n+1})$, 
    which concludes the proof.
\end{proof}
\begin{remark}
    For a Rees regular chain complex $(A,a)$ we have that $(A,a)$ is Rees exact at $A_n$, if and only if $H_n(A)=0$.
\end{remark}
\begin{definition}
    Let $(A,a)$ and $(B,b)$ be two chain complexes. A \textbf{chain map} $f:(A,a)\to(B,b)$ is a sequence of act morphisms $(f_n: A_n\to B_n)$ 
    such that $\forall n\in\mathbb{Z}$ the following square diagram commutes
    \[\begin{tikzcd}
        {A_n} && {B_n} \\
        \\
        {A_{n-1}} && {B_{n-1}}
        \arrow["{f_n}"{description}, from=1-1, to=1-3]
        \arrow["{a_n}", from=1-1, to=3-1]
        \arrow["{b_n}", from=1-3, to=3-3]
        \arrow["{f_{n-1}}"{description}, from=3-1, to=3-3]
    \end{tikzcd}\]
\end{definition}
\begin{proposition}
    Chain complexes of acts along with chain maps form a category $\mathrm{Ch}(\actcat{S})$.
\end{proposition}
\begin{proof}
    Indeed, for $f:(A,a)\to (B,b)$ and $g:(B,b)\to (C,c)$ define $g\circ f$ componentwise as 
    $(g\circ f)_n = g_n \circ f_n$. For a chain complex $(A,a)$ we let the identity $1_{(A,a)}$ 
    be defined as $(1_{(A,a)})_n = 1_{A_n}$. The composition is associative, because it is associative 
    componentwise and the identity acts as the identity on components.
\end{proof}
\begin{proposition}
    Let $f: (A,a) \to (B,b)$ be a chain map and $n\in\mathbb{Z}$, then we have an induced map $H_n(f) : H_n(A) \to H_n(B)$
    defined as $H_n(f)([x]) = [f_n(x)]$. This makes $H_n(A)$ into a functor from $\mathrm{Ch}(\actcat{S})$ to $\actcat{S}$.
\end{proposition}
\begin{proof}[Proof]
    We need to show well-definedness. Assume that $[x]=[y]$, if $x=y$ we are done, otherwise $x,y\in\Image{a_{n+1}}$ so we 
    find $x_0,y_0\in A_{n+1}$ such that $a_{n+1}(x_0) = x$,$a_{n+1}(y_0)=y$. By commutativity $f_n(x)=f_n(a_{n+1}(x_0))=b_{n+1}(f_{n+1}(x_0))$
    and $f_n(y) = f_n(a_{n+1}(y_0))=b_{n+1}(f_{n+1}(y_0))$, thus $[f_n(x)]=[0]=[f_n(y)]$. The fact that $H_n(f)$ is a homomorphism 
    is clear from the definition.\par 
    Next we compute
    \[
        H_n(1_{(A,a)})([x]) = [1_{A_n}(x)] = [x] \implies H_n(1_{(A,a)}) = 1_{H_n(A)}
    \]
    \[
        H_n(g\circ f)([x]) = [g_n(f_n(x))] = H_n(g)([f_n(x)]) =
    \]
    \[
         =(H_n(g)\circ H_n(f))([x]) \implies H_n(g\circ f) = H_n(g)\circ H_n(f)
    \]
\end{proof}
\begin{lemma}
    If $f: (A,a)\to (B,b)$ is a chain map and $f_n$ is Rees regular, then $H_n(f)$ is Rees regular.
\end{lemma}
\begin{proof}[Proof]
    Assume $[f_n(x)]=H_n(f)([x])=H_n(f)([y])=[f_n(y)]$, then $f_n(x)=f_n(y) \implies x=y \lor f_n(x)=0=f_n(y)$ or 
    $f_n(x),f_n(y)\in\Image{b_{n+1}}$. In either case we have $[x]=[y]$ or $H_n(f)([x])=[f_n(x)]=0=[f_n(y)]=H_n(f)([y])$.
\end{proof}
\begin{definition}
    We say that a chain complex $(A,a)$ of pointed acts is \textbf{bounded above}, if there is some $n\in\mathbb{Z}$ such that 
    $A_m=0$ for all $m\geq n$. Similarly we say that $(A,a)$ is \textbf{bounded below}, if there exists $n\in\mathbb{Z}$ such that 
    $A_m=0$ for all $m\leq n$. Chain complex $(A,a)$ is \textbf{bounded}, if it is bounded above and below.
\end{definition}
\begin{definition}
    A \textbf{Rees short exact sequence} is a bounded Rees exact chain complex of the form
    \[\begin{tikzcd}
        0 & A & B & C & 0
        \arrow[from=1-1, to=1-2]
        \arrow["f", from=1-2, to=1-3]
        \arrow["g", from=1-3, to=1-4]
        \arrow[from=1-4, to=1-5]
    \end{tikzcd}\]
\end{definition}
\begin{remark}
    By definition of exactness it follows that $\ker{f} = \{(0,0)\}\cup\Delta_A \land \im(g)= C\times C$, therefore 
    $f$ is injective and $g$ is surjective. Similarly injectivity of $f$ and surjecivity of $g$ imply exactness at $A$ 
    and $C$ respectively.
\end{remark}
\begin{example}
    Let $A$ and $B$ be acts, then we have the following Rees short exact sequence
    \[\begin{tikzcd}
        0 & A & {A\coprod B} & B & 0
        \arrow[from=1-1, to=1-2]
        \arrow["\iota", from=1-2, to=1-3]
        \arrow["\lambda", from=1-3, to=1-4]
        \arrow[from=1-4, to=1-5]
    \end{tikzcd}\]
    where $\iota: x \mapsto (x,0)$ and $\lambda(x,y) = y$.
    the exactness at $A$ and $B$ is easily seen. We will show exactness 
    at $A\coprod B$.\par
    Choose $((x_1,y_1),(x_2,y_2))\in\im(\iota)$, if $x_1=x_2$ and $y_1=y_2$ we are done, 
    otherwise we have $y_1=0=y_2$, therefore 
    \[
        \lambda(x_1,y_1)=0=\lambda(x_2,y_2)
    \]
    which implies $\im(\iota)\subseteq\ker{\lambda}$.
    Let $((x_1,y_1),(x_2,y_2))\in\ker{\lambda}$, if $x_1=x_2$ and $y_1=y_2$
    we are done, therefore assume that $x_1\neq x_2$ or $y_1\neq y_2$.\par
    Suppose $x_1=0$, then $x_2\neq 0$ (else $y_1=y_2$), this implies
    that $y_2=0$ and $y_1=\lambda(x_1,y_1)=\lambda(x_2,y_2)=0$
    so $(x_1,y_1)=(0,0)$ and $y_2=0$, then 
    $\iota(x_2)=(x_2,y_2)$ and $\iota(0)=(x_1,y_1)$.\par
    Suppose now that $x_1\neq 0$, then $y_1=0$ and 
    $\lambda(x_2,y_2)=\lambda(x_1,y_1)=0$, which means either 
    $(x_2,y_2)=(0,0)$ or that $x_2\neq 0$, in either case $y_2=0$,
    hence $\iota(x_1)=(x_1,y_1)$ and $\iota(x_2)=(x_2,y_2)$.
    In conclusion $\ker{\lambda}=\im(\iota)$ as required.
\end{example}
\begin{example}
    If $B$ is a subact of $A$, then the following sequence is Rees short exact 
    \[\begin{tikzcd}
        0 & B & A & {A/B} & 0
        \arrow[from=1-1, to=1-2]
        \arrow["\iota", from=1-2, to=1-3]
        \arrow["\pi", from=1-3, to=1-4]
        \arrow[from=1-4, to=1-5]
    \end{tikzcd}\]
    where $\iota$ is the set inclusion map and $\pi$ is the canonical projection. 
    The exactness at $A$ and $A/B$ is trivial. Choose $x\in B$, then we have 
    $\pi(\iota(x)) = \pi(x) = [x] = [0] \implies \im(\iota)\subseteq\ker{\pi}$. Choose 
    $(x,y)\in\ker{\pi}$, so $[x]=[y]$, if $x=y$, then it is trivially in $\im(\iota)$, else 
    we necessarily have $[x]=[0]=[y]$, so $x,y\in B$ as required.
\end{example}
In contrast with the category of modules, the 
internal hom functor isn't left exact. We give a counterexample, a different 
counterexample already appeas in \cite{Chen02}.
\begin{example}
    Recall that we can consider pointed sets to be $S$ acts over any pointed monoid $S$. Consider a 
    short exact sequence of pointed sets (where we consider $0$ to be the base point of our pointed sets and by $\mathbb{Z}_n$ we mean the set $\{0,\dots,n-1\})$
    \[\begin{tikzcd}
        0 & {\mathbb{Z}_2} & {\mathbb{Z}_3} & {\mathbb{Z}_3/\mathbb{Z}_2} & 0
        \arrow[from=1-1, to=1-2]
        \arrow["\iota", hook, from=1-2, to=1-3]
        \arrow["\pi", two heads, from=1-3, to=1-4]
        \arrow[from=1-4, to=1-5]
    \end{tikzcd}\]
    where $\iota$ is the inclusion map and $\pi$ is the canonical projection.
    We will show that the following sequence isn't exact at $\text{Hom}(\mathbb{Z}_3,\mathbb{Z}_3)$
    \[\begin{tikzcd}
        0 & {\text{Hom}(\mathbb{Z}_3,\mathbb{Z}_2)} & {\text{Hom}(\mathbb{Z}_3,\mathbb{Z}_3)} & {\text{Hom}(\mathbb{Z}_3,\mathbb{Z}_3/\mathbb{Z}_2)}
        \arrow[from=1-1, to=1-2]
        \arrow["{\iota_*}", from=1-2, to=1-3]
        \arrow["{\pi_*}", from=1-3, to=1-4]
    \end{tikzcd}\]
    Note that $\left|\text{Hom}(\mathbb{Z}_3,\mathbb{Z}_2)\right| = 2^{3-1}$, since $0\mapsto 0$, similarly we have 
    $\left|\text{Hom}(\mathbb{Z}_3,\mathbb{Z}_3) \right| =9$ and $\left|\text{Hom}(\mathbb{Z}_3,\mathbb{Z}_3/\mathbb{Z}_2)\right| = 4$.
    Clearly we then have $\left|\Image{\iota_*}\right|\leq 4$. If $\pi_*$ were to be Rees regular, then necessarily 
    $\left|\Kernel{\pi_*}\right| \geq 6$, so we cannot have that $\Image{\iota_*}=\Kernel{\pi_*}$.
\end{example}
\begin{proposition}
    Consider the following Rees short exact sequence
    \[\begin{tikzcd}
        0 & A & B & C & 0
        \arrow[from=1-1, to=1-2]
        \arrow[from=1-2, to=1-3]
        \arrow[from=1-3, to=1-4]
        \arrow[from=1-4, to=1-5]
    \end{tikzcd}\]
    then if $A$ and $C$ are finitely generated, then so is $B$, and if $B$ is finitely generated, then so is $C$.
\end{proposition}
\begin{proof}
    Assume that $\{x_1,\dots,x_n\}$ generates $A$ and $\{z_1,\dots,z_m\}$ generated $C$. By surjectivity of $g$ we find 
    for each $i\in\{1,\dots,m\}$ an element $y_i\in B$ such that $g(y_i)=z_i$. We will show that the set 
    \[
        \{f(x_1),\dots,f(x_n),y_1,\dots,y_m\}
    \]
    generates $B$. Choose any $y\in B$, then if $y\in\Kernel{g}$ we find $x\in A$ such that $f(x)=y$, but then 
    $x=r\cdot x_i$ for some $r\in S$ and $i\in\{1,\dots,n\}$, which implies that $y=r\cdot f(x_i)$. Assume now that $y\not\in\Kernel{g}$,
    then $f(y) = r\cdot z_i = r\cdot f(y_i) = f(r\cdot y_i)$ for some $i\in\{1,\dots,m\}$, hence $y=r\cdot y_i$, since $g$ is Rees regular, therefore 
    injective up to Kernel. This proves that the set $\{f(x_1),\dots,f(x_n),y_1,\dots,y_m\}$ generates B. \par 
    Suppose that $\{y_1,\dots,y_n\}$ generates $B$. Choose $z\in C$, then we find some $y\in B$ such that $g(y)=z$, then 
    $y=r\cdot y_i$ for some $i\in\{1,\dots,n\}$, which implies $z=r\cdot g(y_i)$, therefore $\{g(y_1),\dots,g(y_m)\}$ generates $C$.
\end{proof}
\begin{definition}
    A sequence of chain complexes 
\[\begin{tikzcd}
	0 & {(A,a)} & {(B,b)} & {(C,c)} & 0
	\arrow[from=1-1, to=1-2]
	\arrow["f", from=1-2, to=1-3]
	\arrow["g", from=1-3, to=1-4]
	\arrow[from=1-4, to=1-5]
\end{tikzcd}\]
    (where by $0$ we mean $\cdots\to 0\to0\to0\to\cdots$) is exact 
    if for each $n\in\mathbb{Z}$ we have that 
    \[\begin{tikzcd}
	0 & {A_n} & {B_n} & {C_n} & 0
        \arrow[from=1-1, to=1-2]
        \arrow["{f_n}", from=1-2, to=1-3]
        \arrow["{g_n}", from=1-3, to=1-4]
        \arrow[from=1-4, to=1-5]
    \end{tikzcd}\]
    is a Rees short exact sequence.
\end{definition}
\begin{proposition}
    Assume that $S$ is a commutative, pointed monoid and $K\subseteq S$ is a submonoid of $S$ such that $0\not\in K$, then
    given a Rees short exact sequence 
    \[\begin{tikzcd}
        0 & A & B & C & 0
        \arrow[from=1-1, to=1-2]
        \arrow[from=1-2, to=1-3]
        \arrow[from=1-3, to=1-4]
        \arrow[from=1-4, to=1-5]
    \end{tikzcd}\]
    the following sequence is Rees short exact
    \[\begin{tikzcd}
        0 & {K^{-1}A} & {K^{-1}B} & {K^{-1}C} & 0
        \arrow[from=1-1, to=1-2]
        \arrow[from=1-2, to=1-3]
        \arrow[from=1-3, to=1-4]
        \arrow[from=1-4, to=1-5]
    \end{tikzcd}\]
\end{proposition}
\begin{proof}[Proof]
    Let $\frac{a}{k}\in K^{-1}A$, then 
    \[
    (K^{-1}g\circ K^{-1}f)\left(\frac{a}{k}\right) = \frac{g(f(a))}{k} = \frac{0}{k} =\frac{0}{1}=0 \implies \im(K^{-1}f)\subseteq\ker{K^{-1}g}
    \]
    Assume that $\frac{g(b_1)}{k_1}=\frac{g(b_2)}{k_2}$, so we find $u\in K$ such that 
    $g(uk_1b_2)=uk_1g(b_2)=uk_2g(b_1)=g(uk_2b_1)$, therefore 
    $(uk_1b_2,uk_2b_1)\in\im(f)$. If $uk_1b_2=uk_2b_1$ we are done, otherwise 
    we find $z_1,z_2\in A$ such that $f(z_1) = uk_1b_2$ and $f(z_2)=uk_2b_1$.
    then 
    \[
        K^{-1}f\left(\frac{z_1}{uk_1k_2}\right) = \frac{uk_1b_2}{uk_1k_2} = \frac{b_2}{k_2}
    \] 
    \[
        K^{-1}f\left(\frac{z_2}{uk_2k_1}\right) = \frac{uk_2b_1}{uk_2k_1} = \frac{b_1}{k_1}
    \]
    which proves exactness at $K^{-1}B$, we have to show that $K^{-1}$ preserves injections and surjections.
    Suppose $\frac{f(a_1)}{k_1}=\frac{f(a_2)}{k_2}$, then find $u\in K^{-1}$ such that 
    \[
        uk_1f(a_2) = uk_2f(a_1)\implies f(uk_1a_2)=f(uk_2a_1) \implies
    \] 
    \[
        \implies uk_1a_2=uk_2a_1 \implies\frac{a_1}{k_1}=\frac{a_2}{k_2}
    \]
    To prove surjectivity choose $\frac{z}{k}\in K^{-1}C$, we find $y\in B$ such that $g(y)=z$, then 
    \[
        K^{-1}g\left(\frac{y}{k}\right) = \frac{g(y)}{k} = \frac{z}{k}
    \]
    which completes the proof.
\end{proof}
\begin{corollary}
    Let $B$ be a subact of $A$, then $K^{-1}B$ is a subact of $K^{-1}A$ and  
    \[
        K^{-1}\left(\frac{A}{B}\right) \simeq \frac{K^{-1}A}{K^{-1}B}
    \]
\end{corollary}
\begin{proof}[Proof]
    Consider the Rees short exact sequence 
    \[\begin{tikzcd}
        0 & {K^{-1}B} & {K^{-1}A} & {K^{-1}(A/B)} & 0
        \arrow[from=1-1, to=1-2]
        \arrow["{K^{-1}\iota}", from=1-2, to=1-3]
        \arrow["{K^{-1}\pi}", from=1-3, to=1-4]
        \arrow[from=1-4, to=1-5]
    \end{tikzcd}\]
    $K^{-1}B$ is clearly a subset of $K^{-1}A$, it contains zero, since $B$ contains zero and 
    it is closed under the monoidal action, because $B$ is.
    First we need to show that $\im K^{-1}\iota = \rho_{K^{-1}B}$, we have 
    \[
    \left(\frac{b_1}{k_1},\frac{b_2}{k_2}\right)\in\im(S^{-1}\iota) \iff \frac{b_1}{k_1}=\frac{b_2}{k_2}\lor\exists\,\frac{a_1}{l_1},\frac{a_2}{l_1}\in K^{-1}B :
    \frac{a_1}{l_1}=\frac{b_1}{k_1}\land\frac{a_2}{l_2}=\frac{b_2}{k_2}
    \]
    \[
        \iff \frac{b_1}{k_1}=\frac{b_2}{k_2}\lor \frac{b_1}{k_1},\frac{b_2}{k_2}\in K^{-1}B \iff \left(\frac{a_1}{k_1},\frac{a_2}{k_2}\right)\in\rho_{K^{-1}B}
    \]
    then by the first isomorphism theorem for acts/universal algebras \cite[Theorem~4.21]{Kilp00} we have 
    \[
    K^{-1}(A/B)\simeq K^{-1}(A)/\ker{K^{-1}\pi} = K^{-1}(A)/\im(K^{-1}\iota) = K^{-1}A/K^{-1}B
    \]
\end{proof}
\section{Rees simple acts}
\begin{definition}
    Let $S$ be a pointed monoid and $A$ be a pointed act over $S$, then we call $A$
    \textbf{Rees simple}, if every congruence in $A$ is Rees (i.e. of the form $\rho_B$ for some $B\subseteq A$ a subact of $A$).
\end{definition}
\begin{example}
    Let $(\mathbb{N}_0, +, 0, \infty)$ be a pointed monoid. We can consider it as an act over itself. For $n\in\mathbb{N}_0$ let 
    $n_> = \{x\in\mathbb{N}_0 \mid x > n\}$, then $n_>$ is a subact of $\mathbb{N}_0$, indeed $\infty\in n_>$ and if $a > n$, then 
    surely $r+a > n$ for any $r \geq 0$, then $\mathbb{N}_0/n_>$ is a pointed act over $\mathbb{N}_0$. Let $\sim$ be a congruence on 
    $\mathbb{N}_0/n_>$, suppose it isn't the diagonal congruence, which is trivially Rees, then there is $m\in\mathbb{N}_0$ and $k>0$ such that 
    $m\sim k+m$, but then, since $\rho$ is a congruence we have that $m\sim k+m\sim 2k+m \sim\dots \sim lk+m \sim \infty$, where $l$
    is large enough, such that $lk+m > n$, hence $x\sim y \iff x=y \lor x,y > n$ and any congruence is Rees. 
\end{example}
\begin{proposition}
    Let $A$ be a pointed act over a pointed monoid $S$, then TFAE
    \begin{enumerate}
        \item Every morphism $f: A \to X$ for all $X\in\actcat{S}$ is Rees regular.
        \item $A$ is Rees simple.
    \end{enumerate}
\end{proposition}
\begin{proof}[Proof]
    $1. \implies 2.$\par 
    Let $\kappa$ be a congruence in $A$, then $\kappa=\ker{\pi}$, where $\pi: A \to A/\kappa$ is the canonical projection, by 
    assumption $\pi$ si Rees regular, it follows that $\kappa$ is a Rees congruence by proposition \ref{regPropChar}. \par 
    $2. \implies 1.$\par 
    Let $f: A\to X$ be a homomorphism, then $\ker{f}$ is a Rees congruence, since $A$ is Rees simple, hence by proposition \ref{regChar} 
    we have that $f$ is Rees regular.
\end{proof}
\begin{proposition}
    Consider a Rees short exact sequence 
    \[
        \begin{tikzcd}
            0 & A & B & C & 0
            \arrow[from=1-1, to=1-2]
            \arrow["f", from=1-2, to=1-3]
            \arrow["g", from=1-3, to=1-4]
            \arrow[from=1-4, to=1-5]
        \end{tikzcd}
    \]
    if $B$ is Rees simple, then so are $A$ and $C$.
\end{proposition}
\begin{proof}
    Let $\kappa$ be a congruence on $A$, then we define a congruence $\rho$ on $B$ as $(x,y)\in\rho \iff x=y \lor (x=f(x_0), y=f(y_0) \land (x_0,y_0)\in\kappa)$. 
    It is easy to see, that it is indeed a congruence relation, hence there is some $B'\subseteq B$ such that $\rho = \rho_{B'}$, then 
    $f^{-1}(B')$ is a subact of $A$, since $f(0)=0\in B'$ and if $f(x)\in B'$, then $rf(x)=f(rx)\in B'$. We want to show that 
    $\kappa = \rho_{f^{-1}(B')}$. If $(x,y)\in\kappa$ and $x\neq y$, then $f(x)\neq f(y)$ by injectivity of $f$, hence 
    $(f(x),f(y))\in\rho$, which implies that $x,y\in f^{-1}(B')$. Suppose otherwise, that $f(x),f(y)\in B'$, then $(f(x),f(y))\in\rho$, 
    which either implies $f(x)=f(y) \implies x=y$ or that $(x,y)\in\kappa$, which concludes the proof.\par
    Let $\kappa$ be a congruence on $C$, then $g^{-1}(\kappa) := \{(x,y) : (g(x),g(y))\in\kappa\}$ is clearly a congruence on 
    $B$, so we find $B'\subseteq B$ such that $g^{-1}(\kappa)=\rho_{B'}$. We will show that $\kappa=\rho_{g(B')}$. Let $(x,y)\in\kappa$ be such that $x\neq y$, then we find $x_0,y_0\in B$, where 
    $g(x_0)=x$ and $g(y_0)=y$, then $x_0\neq y_0$ and $(x_0,y_0)\in g^{-1}(\kappa)=\rho_{B'}$, therefore $x_0,y_0\in B' \implies x,y\in g(B')$. Let $(x,y)\in\rho_g(B')$ and $x\neq y$,
    we find $x_0,y_0\in B'$ so that $g(x_0)=x$ and $g(y_0)=y$. We now have have that $(x_0,y_0)\in g^{-1}(\kappa)$, hence $(x,y)\in\kappa$.
\end{proof}
%\chapter{Diagrams of acts}
\begin{proposition}\label{regprop}
    Let $f : A\to B$ and $g: B\to C$ be morphisms of pointed acts such that $g\circ f$ is Rees regular.
    \begin{enumerate}
        \item If $g$ is a monomorphism, then $f$ is Rees regular.
        \item If $f$ is an epimorphism, then $g$ is Rees regular.
    \end{enumerate}
\end{proposition}
\begin{proof}[Proof]
    \begin{enumerate}
       \item Let $f(x)=f(y)$, then $(g\circ f)(x)=(g\circ f)(y)$, which implies $x=y$ or 
       $g(f(x))=0=g(f(y))$. In the second case by injectivity of $g$ we have $f(x)=0=f(y)$.
       \item Assume $g(x)=g(y)$ we find $x_0,y_0\in A$ such that $g(f(x_0))=g(f(y_0))$, by regularity
       we have $x_0=y_0 \implies x=y$ or $g(x)=g(f(x_0))= 0 = g(f(y_0))=g(y)$.
    \end{enumerate}
\end{proof}
\begin{lemma}\label{reglemma1}
    Given a commutative diagram of pointed acts whose first row is Rees exact
    \[\begin{tikzcd}
        A & B & C \\
        {A'} & {B'} & {C'}
        \arrow["f", from=1-1, to=1-2]
        \arrow["\alpha", "\circ"{marking}, from=1-1, to=2-1]
        \arrow["g", from=1-2, to=1-3]
        \arrow["\beta", from=1-2, to=2-2]
        \arrow["\gamma", hook, from=1-3, to=2-3]
        \arrow["{f'}"', "\circ"{marking}, from=2-1, to=2-2]
        \arrow["{g'}"', from=2-2, to=2-3]
    \end{tikzcd}\]
    If $f'$ and $\alpha$ are Rees regular and $\gamma$ a monomorphism, then $\beta$ is Rees regular.
\end{lemma}
\begin{proof}[Proof]
    Assume that $\beta(x)=\beta(y)$ and $x\neq y$, then we have $\gamma(g(x))=g'(\beta(x))=g'(\beta(y))=\gamma(g(y))$, which 
    implies $g(x)=g(y)$, by Rees exactness at $B$ we find $x_0,y_0\in A$ such that $f(x_0)=x$ and $f(y_0)=y$, then 
    $f'(\alpha(x_0))=\beta(x)=\beta(y)=f'(\alpha(y_0))$. If $f'(\alpha(x_0)) =0 = f'(\alpha(y_0))$ we would have 
    $\beta(x) = 0 = \beta(y)$ by commutativity, else we have $\alpha(x_0)=\alpha(y_0)$, which by regularity implies
    $x_0=y_0\implies x=y$ or that $\alpha(x_0)=0=\alpha(y_0)$, which would again imply that $\beta(x)=0=\beta(y)$.
\end{proof}
\begin{lemma}\label{reglemma2}
    Consider the following commutative diagram whose second row and third column are exact
    \[\begin{tikzcd}
        & B & C \\
        {A'} & {B'} & {C'} \\
        {A''} & {B''} & {C''}
        \arrow["g", two heads, from=1-2, to=1-3]
        \arrow["\beta", from=1-2, to=2-2]
        \arrow["\gamma", from=1-3, to=2-3]
        \arrow["{f'}", from=2-1, to=2-2]
        \arrow["{\alpha'}", "\circ"{marking}, from=2-1, to=3-1]
        \arrow["{g'}", from=2-2, to=2-3]
        \arrow["{\beta'}", from=2-2, to=3-2]
        \arrow["{\gamma'}", from=2-3, to=3-3]
        \arrow["{f''}", "\circ"{marking}, from=3-1, to=3-2]
        \arrow["{g''}", from=3-2, to=3-3]
    \end{tikzcd}\]
    If $g$ is an epimorphism, $\alpha'$, $f''$ are Rees regular and $\beta'\circ\beta = 0$, then $\beta'$ is Rees regular.
\end{lemma}
\begin{proof}[Proof]
    Suppose $\beta'(x)=\beta'(y)$,$x\neq y$, then $\gamma'(g'(x))=g''(\beta'(x))=g''(\beta'(y)) = \gamma'(g'(y))$ by 
    regularity we either have $g'(x)=g'(y)$ or $g'(x),g'(y)\in\Kernel{\gamma'}$. We will consider both cases separately. \par 
    In the first case, because of Rees exactness 
    we find $x_0,y_0\in A'$ such that $f'(x_0)=x$ and $g'(y_0)=y$, which implies 
    $f''(\alpha'(x_0))=\beta'(x)=\beta'(y)=f''(\alpha'(y_0))$. If $\alpha'(x_0)=\alpha'(y_0)$, then by regularity 
    of $\alpha'$ we would have $\alpha'(x_0)=0=\alpha'(y_0) \implies \beta(x)=0=\beta(y)$, since $x_0\neq y_0$.
    Otherwise we have $\beta'(x)=f''(\alpha'(x_0))=0=f''(\alpha'(y_0))=\beta'(y)$.\par
    In the second case, that is if $g'(x),g'(y)\in\Kernel{\gamma'}=\Image{\gamma}$, we find $x',y'\in C$ so that 
    $\gamma(x')=g'(x)$ and $\gamma(y')=g'(y)$. Since $g$ is an epimorphism we find $x'_0,y'_0\in B$ so that 
    $g'(x) = \gamma(g(x'_0))= g'(\beta(x'_0))$ and $g'(y)=\gamma(g(y'_0))=g'(\beta(y'_0))$. If, say 
    $x = \beta(x'_0)$ we would have $\beta'(y)=\beta'(x)=0$ and similarly if $y=\beta(y'_0)$. Otherwise we have $g'(x)=0=g'(y)$, but we have already resolved this case above. This completes the proof.
\end{proof}
The idea behind the lemmata above is to use them for larger diagrams to show Rees regularity whenever necessary and then using
proposition $\ref{regularityProp}$ we can do a regular diagram chase to verify Rees exactness.\par 
The following three results, i.e. the four lemma, five lemma and splitting lemma for acts already appear in \cite{Jafari19}. 
The splitting lemma in particular is first stated in \cite{Chen02}. We give more concise
proofs that utilize the characterization in proposition $\ref{regularityProp}$.
\begin{lemma}[Four lemma]
    Consider the following commutative diagram, whose rows are Rees exact
    \[\begin{tikzcd}
        A & B & C & D \\
        {A'} & {B'} & {C'} & {D'}
        \arrow["f", from=1-1, to=1-2]
        \arrow["r", from=1-1, to=2-1]
        \arrow["g", from=1-2, to=1-3]
        \arrow["s", from=1-2, to=2-2]
        \arrow["h", from=1-3, to=1-4]
        \arrow["t", from=1-3, to=2-3]
        \arrow["u", from=1-4, to=2-4]
        \arrow["{f'}", from=2-1, to=2-2]
        \arrow["{g'}", from=2-2, to=2-3]
        \arrow["{h'}", from=2-3, to=2-4]
    \end{tikzcd}\]
    then
    \begin{enumerate}
        \item If $s,u$ are monomorphisms and $r$ is an epimorphism, then $t$ is a monomorphism.
        \item If $r,t$ are epimorphisms and $u$ is a monomorphim, then $s$ is an epimorphism.
    \end{enumerate}
\end{lemma}
\begin{proof}[Proof]
    To prove the first part we first apply lemma \ref{reglemma1} to see that $t$ is Rees regular. Let $t(x)=0$, then 
    $u(h(x)) = h'(t(x))=0 \implies h(x)=0$, then using Rees exactness we find $x'\in B$ such that $g(x')=x$, then 
    $g'(s(x')) = t(x) = 0$, then we find $z\in A'$ such that $f'(z)=s(x')$, since $r$ is an epimorphism we find $z'\in A$
    such that $r(z')=z$, then $s(x')=f'(z)=f'(r(z'))=s(f(z')) \implies x'=f(z') \implies 0=g(x')=x$. Thus the kernel is 
    trivial, which implies that $t$ is injective by Rees regularity of $t$. \par
    To prove the second part choose $y\in B'$, then by surjectivity of $t$ we find $y'\in C$ such that $g'(y)=t(y')$, then 
    we have $u(h(y')) = h'(t(y'))= h'(g'(y)) = 0 \implies h(y')=0$. By Rees exactness we find $y''\in B$ such that 
    $g(y'')=y'$, then $g'(s(y'')) = t(g(y'')) = t(y') = g'(y)$. If $s(y'')=y$ we are done, otherwise by regularity of $g'$ we 
    have $g'(y)=0$, so by Rees exactness there is some $z\in A'$ such that $f'(z)=y$. Since $r$ is an epimorphism we find $z'\in A$
    so that $r(z')=z$, then $y = f'(r(z')) = s(f(z'))$, hence $y\in\Image{s}$, which concludes the proof.
\end{proof}
\begin{corollary}[Five lemma]
    Consider the following commutative diagram with Rees exact rows
    \[\begin{tikzcd}
            A & B & C & D & E \\
            {A'} & {B'} & {C'} & {D'} & {E'}
            \arrow["f", from=1-1, to=1-2]
            \arrow["r", from=1-1, to=2-1]
            \arrow["g", from=1-2, to=1-3]
            \arrow["s", from=1-2, to=2-2]
            \arrow["h", from=1-3, to=1-4]
            \arrow["t", from=1-3, to=2-3]
            \arrow["k", from=1-4, to=1-5]
            \arrow["u", from=1-4, to=2-4]
            \arrow["v", from=1-5, to=2-5]
            \arrow["{f'}", from=2-1, to=2-2]
            \arrow["{g'}", from=2-2, to=2-3]
            \arrow["{h'}", from=2-3, to=2-4]
            \arrow["{k'}", from=2-4, to=2-5]
    \end{tikzcd}\]
    if $s,u$ are isomorphisms, $r$ is an epimorphism and $v$ is a monomorphism, then $t$ is an isomorphism.
\end{corollary}
\begin{proof}[Proof]
    By applying the first and second part of the four lemma to our diagram we get that $t$ is a monomorphism and an 
    epimorphism, therefore an isomorphism, because the category of pointed acts is balanced.
\end{proof}
\begin{corollary}[Short five lemma]
    Consider the following commutative diagram with Rees exact rows
    \[\begin{tikzcd}
        0 & A & B & C & 0 \\
        0 & {A'} & {B'} & {C'} & 0
        \arrow[from=1-1, to=1-2]
        \arrow["f", from=1-2, to=1-3]
        \arrow["r", from=1-2, to=2-2]
        \arrow["g", from=1-3, to=1-4]
        \arrow["s", from=1-3, to=2-3]
        \arrow[from=1-4, to=1-5]
        \arrow["t", from=1-4, to=2-4]
        \arrow[from=2-1, to=2-2]
        \arrow["{f'}", from=2-2, to=2-3]
        \arrow["{g'}", from=2-3, to=2-4]
        \arrow[from=2-4, to=2-5]
    \end{tikzcd}\]
    \begin{enumerate}
        \item If $r$ and $t$ are monomorphisms, then $s$ is a monomorphism.
        \item If $r$ and $t$ are epimorphisms, then $s$ is an epimorphism.
        \item If $r$ and $t$ are isomorphism, then $s$ is an isomorphism. 
    \end{enumerate}
\end{corollary}
\begin{proof}[Proof]
    \begin{enumerate}
        \item Since $0$ is an epimorphism, $r$ and $t$ are monomorphisms we can apply the first part of the four lemma 
        to get the result.
        \item Similarly since $r$ and $t$ are epimorphisms and $0$ is a monomorphism we can apply the second part of the four lemma to the diagram.
        \item We can either use $1.$ and $2.$ or apply the five lemma.
    \end{enumerate}
\end{proof}

\begin{proposition}[Splitting sequences]
    Let 
    \[\begin{tikzcd}
        0 & A & B & C & 0
        \arrow[from=1-1, to=1-2]
        \arrow["f", from=1-2, to=1-3]
        \arrow["g", from=1-3, to=1-4]
        \arrow[from=1-4, to=1-5]
    \end{tikzcd}\]
    be a Rees short exact sequence, then TFAE:
    \begin{enumerate}
        \item $g$ is a split epimorphism
        \item $f$ is a split monomorphism and $(B\setminus\Image{f})\cup\{0\}$ is a subact of $B$.
        \item There is an isomorphism $\phi: B\to  A\coprod C$ such that $\phi\circ f=\iota_A$ and 
        $g\circ\phi^{-1}=\lambda_C$.
    \end{enumerate}
\end{proposition}
\begin{proof}[Proof]
    Assume that there is a $g':C \to B$ such that $g\circ g' = 1_C$. We define a map $\psi: A\coprod C \to B$ as 
    $\psi(a,0) = f(a)$ and $\psi(0,c) = g'(c)$. This is a homomorphism of acts, because $g'$ and $f$ are. We have that 
    \[
        (\psi\circ \iota_A)(a) = \psi(a,0) = f(a) \implies \psi\circ\iota_A = f
    \]
    \[
        (g\circ\psi)(a,0) = g(f(a)) = 0\hspace{20pt} (g\circ\psi)(0,c) = (g\circ g')(c) = c \implies g\circ\psi = \lambda_C
    \]
    therefore this diagram with short exact rows commutes
    \[\begin{tikzcd}
	0 & A & {A\coprod C} & C & 0 \\
	0 & A & B & C & 0
	\arrow[from=1-1, to=1-2]
	\arrow["{\iota_A}", from=1-2, to=1-3]
	\arrow[equal,from=1-2, to=2-2]
	\arrow["{\lambda_C}", from=1-3, to=1-4]
	\arrow["\psi", from=1-3, to=2-3]
	\arrow[from=1-4, to=1-5]
	\arrow[equal,from=1-4, to=2-4]
	\arrow[from=2-1, to=2-2]
	\arrow["f", from=2-2, to=2-3]
	\arrow["g", shift left, from=2-3, to=2-4]
	\arrow["{g'}", shift left, from=2-4, to=2-3]
	\arrow[from=2-4, to=2-5]
\end{tikzcd}\]
    by the short five lemma $\psi$ is an isomorphism. Then we have that $\iota_A = \psi^{-1}\circ f$ and $g = \lambda_C\circ\psi^{-1}$ 
    by the commutativity of the squares above.\par 
    Assume now that there is an $f': B \to A$ such that $f'\circ f = 1_A$. We define a map $\phi: B \to A\coprod C$ as
    \[
        \phi(b) = \begin{cases}
            (f'(b),0) & \text{if } b\in\Image{f} \\
            (0,g(b)) & \text{if } b\not\in\Image{f}
        \end{cases}
    \]
    We will now use the fact that $(B\setminus\Image{f})\cup\{0\}$ is a subact of $B$ to show it is a homomorphism. First off, if 
    $b\in\Image{f}$, then $rb\in\Image{f}$, hence 
    \[
    \phi(rb) = (f'(rb),0)= r\cdot(f'(b),0) = r\phi(b)
    \]
    Assume now that $b\not\in\Image{f}$ and $rb=0$, then 
    \[
        r\phi(b) = r\cdot(0,g(b)) = (0,g(rb)) = (0,0) = (f'(rb),0) = \phi(rb)
    \]
    If $rb\neq 0$, then $rb\not\in\Image{f}$ and 
    \[
        \phi(rb) = (0,g(rb)) = r\cdot(0,g(b)) = r\phi(b)
    \]
    therefore it is well-defined. We will again verify the commutativity of the following diagram and apply the short five lemma
    \[\begin{tikzcd}
        0 & A & B & C & 0 \\
        0 & A & {A\coprod C} & C & 0
        \arrow[from=1-1, to=1-2]
        \arrow["f", shift left, from=1-2, to=1-3]
        \arrow[equal, from=1-2, to=2-2]
        \arrow["{f'}", shift left, from=1-3, to=1-2]
        \arrow["g", from=1-3, to=1-4]
        \arrow["\phi", from=1-3, to=2-3]
        \arrow[from=1-4, to=1-5]
        \arrow[equal, from=1-4, to=2-4]
        \arrow[from=2-1, to=2-2]
        \arrow["{\iota_A}", from=2-2, to=2-3]
        \arrow["{\lambda_C}", from=2-3, to=2-4]
        \arrow[from=2-4, to=2-5]
    \end{tikzcd}\]
    \[
        (\phi\circ f)(a) = (f'(f(a)),0) = (a,0) = \iota_A(a) \implies \phi\circ f = \iota_A
    \]
    To verify the commutativity of the right triangle we need to consider two cases. Suppose that $b\in\Image{f}$, then 
    \[
        (\lambda_C\circ \phi)(b) = \lambda_C((f'(b),0)) = 0 = g(b)
    \]
    since $\Image{f}\subseteq\Kernel{g}$. Now assume $b\not\in\Image{f}$, then we have 
    \[
        (\lambda_C\circ\phi)(b) = \lambda_C(0,g(b)) = g(b)
    \]
    therefore the right triangle commutes, by the short five lemma we get that $\phi$ is an isomorphism, and we are done.\par 
    We will show that if the third condition holds, then both the first and the second hold. Let $\phi: B \to A\coprod C$ be 
    an isomorphism, such that $g\circ\phi^{-1}=\lambda_C$ and $\phi\circ f = \iota_A$.
    We have that 
    \[
        \lambda_A \circ \phi \circ f = \lambda_A \circ \iota_A = 1_A
    \]
    \[
        g\circ \phi^{-1} \circ \iota_C = \lambda_C\circ\iota_C = 1_C
    \]
    therefore $f$ and $g$ split. It remains to show that $B\setminus\Image{f} \cup\{0\}$ is a subact of $B$. Let 
    $b\not\in\Image{f}$ and choose $r\in S$, then $\phi(b)=(0,c)$, where $c\neq 0$, otherwise, if 
    $\phi(b)=(a,0)$, then $g(b) = (\lambda_C\circ \phi)(b) = 0 \implies b\in\Kernel{g}=\Image{f}$. Let $\phi(b)=(0,c)$, where 
    $c\neq 0$ and assume $rb\neq 0$ (otherwise we are done), then $(0,rc) = \phi(rb) \neq 0$, so $rc\neq 0$, but then 
    $g(rb) = (\lambda_C\circ\phi)(rb) = rc \neq 0$, so $rb\not\in\Image{f}$ as required.
\end{proof}
\begin{lemma}[Snake lemma for acts]
    Consider the following commutative diagram with Rees exact rows
    \[\begin{tikzcd}
        & A & B & C & 0 \\
        0 & {A'} & {B'} & {C'}
        \arrow["\alpha", from=1-2, to=1-3]
        \arrow["f", from=1-2, to=2-2]
        \arrow["\beta", from=1-3, to=1-4]
        \arrow["g", "\circ"{marking}, from=1-3, to=2-3]
        \arrow[from=1-4, to=1-5]
        \arrow["h", from=1-4, to=2-4]
        \arrow[from=2-1, to=2-2]
        \arrow["\gamma", from=2-2, to=2-3]
        \arrow["\delta", from=2-3, to=2-4]
    \end{tikzcd}\]
    such that $g$ is Rees regular, then we have a Rees exact sequence
    \[\begin{tikzcd}[sep=small]
        {\Kernel{f}} & {\Kernel{g}} & {\Kernel{h}} & {\Cokernel{f}} & {\Cokernel{g}} & {\Cokernel{h}}
        \arrow["{\hat{\alpha}}", from=1-1, to=1-2]
        \arrow["{\hat{\beta}}", from=1-2, to=1-3]
        \arrow["\partial", from=1-3, to=1-4]
        \arrow["{\hat{\gamma}}", from=1-4, to=1-5]
        \arrow["{\hat{\delta}}", from=1-5, to=1-6]
    \end{tikzcd}\]
    where $\partial$ is the so called connecting homomorphism. Moreover, if $\alpha$ is a monomorphism, then so is the induced 
    homomorphism $\hat{\alpha}$ and if $\delta$ is an epimorphism, then so is $\hat{\delta}$.
\end{lemma}
\begin{proof}[Proof]
    \begin{enumerate}
        \item The well-definedness of $\hat{\alpha},\hat{\beta}$ and Rees exactness at $\Kernel{g}$.\par
        We define the maps $\hat{\alpha}=\alpha\mid_{\Kernel{f}}$ and $\hat{\beta}=\beta\mid_{\Kernel{g}}$. To show 
        well-definedness let $f(x)=0$, then $0=\gamma(f(x))=g(\alpha(x)) \implies \alpha(x)\in\Kernel{g}$. Similarly if 
        $g(x)=0$ we have $0=\delta(g(x))=h(\beta(x)) \implies \beta(x)\in\Kernel{h}$. The Rees regularity at $\hat{\beta}$ is clear 
        because $\beta(x)=\hat{\beta}(x)=\hat{\beta}(y)=\beta(y)\implies x=y \lor \beta(x)=0=\beta(y)$. Let $\hat{\beta}(x)=0$, then
        we find $x'\in A$ such that $\alpha(x')=x$, then $\gamma(f(x'))=g(\alpha(x'))=g(x)=0$, since $x\in\Kernel{g}$, which implies
        $f(x')=0$,so $x\in\Image{\hat{\alpha}}$, since we have $\hat{\beta}\circ\hat{\alpha}=\beta\circ\alpha = 0$ the Rees exactness at $\Kernel{g}$ follows.
        \item The well-definedness of $\hat{\gamma},\hat{\delta}$ and Rees exactness at $\Cokernel{g}$.\par
        Let $\hat{\gamma}([x])=[\gamma(x)]$ and $\hat{\delta}([x])=[\delta(x)]$. We will show that $\hat{\gamma}$ is well-defined. 
        Let $[x]=[y]$, if $x=y$ we are done, otherwise there are $x_0,y_0\in A$ such that $f(x_0)=x$ and $f(y_0)=y$, then 
        $\gamma(x)=\gamma(f(x_0))=g(\alpha(x_0)) \implies \gamma(x)\in\Image{g}$ and similarly we get $\gamma(y)\in\Image{g}$, thus 
        $\hat{\gamma}([x])=[0]=\hat{\gamma}([y])$. In the same manner we can show that $\hat{\delta}$ is well-defined. Next let us
        see that $\hat{\delta}$ is Rees regular. If $[\delta(x)]=\hat{\delta}([x])=\hat{\delta}([y])=[\delta(y)]$ we either have 
        $\delta(x)=\delta(y) \implies x=y\lor \delta(x)=0=\delta(y)$ by regularity of $\delta$, which clearly implies $\hat{\delta}$ is Rees regular, 
        or else we have $\delta(x),\delta(y)\in\Image{h}$, which means precisely $\hat{\delta}([x])=[0]=\hat{\delta}([y])$. The fact 
        that $\hat{\delta}\circ\hat{\gamma}=0$ is clear from definition. Let $\hat{\delta}([x])=[0]$. If $\delta(x)=0$ by Rees exactness 
        we have $\gamma(x_0)=x$ for some $x_0\in A$, hence $\hat{\gamma}([x_0])=[x]$. Otherwise we have $\delta(x)\in\Image{h}$, so 
        since $\beta$ is epi we have some $x_0\in B$ such that $\delta(x)=h(\beta(x_0))=\delta(g(x_0))$. If $g(x_0)=x$ we have 
        $[x]=[0]$ and trivially it's in the image of $\hat{\gamma}$, else we find by Rees exactness some $z\in A$ s.t. $\gamma(z)=x$, which implies 
        $\hat{\gamma}([z])=[x]$ and we are done.
        \item Construction of $\partial$.\par 
        Let $x\in\Kernel{h}$, then we find $x_0\in B$ such that $\beta(x_0)=x$, since $\delta(g(x_0))=0$ we find 
        necessarily unique $x_0'\in A'$ so that $g(x_0)=\gamma(x_0')$. Let $\partial(x) = [x_0']$.
        \item Well-definedness of $\partial$. \par
        We will verify that $\forall x,y : x=y \implies \partial(x)=\partial(y)$.\par
        Let $x,y\in\Kernel{h},x=y$ we find $x_0,y_0\in B$ such that $\beta(x_0)=x=y=\beta(y_0)$. By regularity of $\beta$ 
        we either have $x_0=y_0 \implies \partial(x)=\partial(y)$ or $\beta(x_0)=0=\beta(y_0)$. In the second case
        by Rees exactness we find $\hat{x_0},\hat{y_0}\in A$ such that $\alpha(\hat{x_0})=x_0$ and $\alpha(\hat{y_0})=y_0$. 
        Continuing with our construction we have 
        $\gamma(x_0')=g(x_0)=\gamma(f(\hat{x_0})) \implies x_0'=f(\hat{x_0})$ and similarly $y_0'=f(\hat{y_0})$,
        which implies that $\partial(x)=[x_0']=[0]=[y_0']=\partial(y)$, therefore $\partial$ is well-defined.
        \item $\partial$ is a homomorphism of acts.\par 
        Let $\partial(rx)=[y_0']$. We have $\beta(y_0)=rx=r\beta(x_0)=\beta(rx_0)$, therefore by regularity of $\beta$ 
        we have $y_0=rx_0 \implies \partial(rx)=r\partial(x)$ or we find $\hat{y_0},z\in A$ such that 
        $\alpha(\hat{y_0})=y_0$ and $\alpha(z)= rx$, then  $\gamma(rx_0')=rg(x_0)=g(\alpha(z))=\gamma(f(z)) \implies rx_0'=f(z)$
        and $y_0' = f(\hat{y_0})$, thus
        \[
            r\partial(x)=r[x_0']=[rx_0']=[f(z)]=[0]=[f(\hat{y_0})]=[y_0']=\partial(rx)
        \] 
        \item Rees regularity of $\partial$.\par
        Let $[x_0']=\partial(x)=\partial(y)=[y_0'] \implies x_0'=y_0' \lor x_0',y_0'\in\Kernel{f}$. In the second case 
        we clearly have $\partial(x)=0=\partial(y)$. Assume that $x_0'=y_0'$, then $g(x_0)=g(y_0)$, by Rees regularity of $g$
        we either have $x_0=y_0 \implies x=\beta(x_0)=\beta(y_0)=y$ or $\gamma(x_0')=g(x_0)=0=g(y_0)=\gamma(y_0') \implies x_0'=0=y_0'$, 
        but then $\partial(x)=0=\partial(y)$.
        \item Rees exactness at $\Kernel{h}$.\par
        Let $[x_0']=\partial(x)=[0]$, that means that $x_0'\in\Image{f}$. Let $f(z)=x_0'$, then we have $g(x_0)=g(\alpha(z))$.
        If $x_0=\alpha(z)$ we have $x=\beta(x_0)=\beta(\alpha(z))=0$ and then clearly $x\in\Image{\hat{\beta}}$,since $\hat{\beta}(0)=x$.
        Otherwise by Rees regularity of $g$ we have $g(x_0)=0$, so $x_0\in\Kernel{g}$ and $\hat{\beta}(x_0)=\beta(x_0)=x$.
        It remains to show that $\partial\circ\hat{\beta}=0$. Choose $x\in\Kernel{g}$, we find $x_0\in B$ such that 
        $\beta(x_0)=\beta(x)$. If $x_0=x$ we have $\gamma(x_0')=g(x_0')=g(x)=0 \implies \partial(\hat{\beta}(x))=0$, else by Rees regularity
        of $\beta$ we have $\beta(x_0)=0$, so there is $z_0\in A$ such that $\alpha(z_0)=x_0$ by Rees exactness. This implies that 
        $\gamma(x_0')=g(x_0)=g(\alpha(z_0))=\gamma(f(z_0))$, hence $\partial(\hat{\beta}(x))=0$. This show Rees exactness at $\Kernel{h}$. 
        \item Rees exactness at $\Cokernel{f}$.\par 
        Note regularity of $\hat{\gamma}$, since $[\gamma(x)]=[\gamma(y)]$ implies $\gamma(x)=\gamma(y)\implies x=y$ or 
        $[\gamma(x)]=[0]=[\gamma(y)]$. To show $\hat{\gamma}\circ\partial = 0$, it is enough to observe that 
        $\hat{\gamma}(\partial(x))=[\gamma(x_0')]=[g(x_0)]=[0]$. Suppose $[\gamma(z_0')]=[0]$. There exists some 
        $z_0\in B$ such that $g(z_0)=\gamma(z_0')$. Let $z = \beta(z_0)$ and notice that 
        $h(z) = h(\beta(z_0))= \delta(g(z_0))=\delta(\gamma(z_0'))=0$, therefore $z\in\Kernel{h}$ and $\partial(z)=[z_0']$ by
        construction.
        \item $\alpha$ mono $\implies\hat{\alpha}$ mono and $\delta$ epi $\implies\hat{\delta}$ epi.\par 
        We have defined $\hat{\alpha}$ as a restriction of $\alpha$, therefore if $\alpha$ is injective, then so is $\hat{\alpha}$. 
        If $\delta$ is an epimorphism, then for each $y\in C'$ we find $x\in B'$ such that $\delta(x)=y$, then
        $\hat{\delta}([x]) = \delta(x) = y$, so $\hat{\delta}$ is an epimorphism.
        
    \end{enumerate}
\end{proof}
\begin{corollary}
    Let $0 \to (A,a) \to (B,b) \to (C,c) \to 0$ be a Rees short exact sequence of chain complexes and $(B,b)$ Rees regular, then 
    we have a Rees long exact sequence of homology acts 
    \[\begin{tikzcd}[sep=small]
        \cdots & {H_{n+1}(A)} & {H_n(A)} & {H_n(B)} & {H_n(C)} & {H_{n-1}(A)} & \cdots
        \arrow[from=1-1, to=1-2]
        \arrow["\partial", from=1-2, to=1-3]
        \arrow["{H_n(f)}", from=1-3, to=1-4]
        \arrow["{H_n(g)}", from=1-4, to=1-5]
        \arrow["\partial", from=1-5, to=1-6]
        \arrow[from=1-6, to=1-7]
    \end{tikzcd}\]
\end{corollary}
\begin{proof}[Proof]
    Since each $b_{n}$ is Rees regular we can apply a part of the Snake lemma to the following diagram
    \[\begin{tikzcd}
        0 & {A_{n+1}} & {B_{n+1}} & {C_{n+1}} & 0 \\
        0 & {A_n} & {B_n} & {C_n} & 0
        \arrow[from=1-1, to=1-2]
        \arrow[from=1-2, to=1-3]
        \arrow["{a_{n+1}}", from=1-2, to=2-2]
        \arrow[from=1-3, to=1-4]
        \arrow["{b_{n+1}}", from=1-3, to=2-3]
        \arrow[from=1-4, to=1-5]
        \arrow["{c_{n+1}}", from=1-4, to=2-4]
        \arrow[from=2-1, to=2-2]
        \arrow[from=2-2, to=2-3]
        \arrow[from=2-3, to=2-4]
        \arrow[from=2-4, to=2-5]
    \end{tikzcd}\]
    yielding a Rees short exact sequence 
    \[\begin{tikzcd}
        {\frac{A_n}{\im(a_{n+1})}} & {\frac{B_n}{\im(b_{n+1})}} & {\frac{C_n}{\im(c_{n+1})}} & 0
        \arrow["{\hat{f}_n}", from=1-1, to=1-2]
        \arrow["{\hat{g}_n}", from=1-2, to=1-3]
        \arrow[from=1-3, to=1-4]
    \end{tikzcd}\]
    Applying a part of the snake lemma to the same diagram shifted down two rows we get a Rees short exact sequence
    \[\begin{tikzcd}
        0 & {\Kernel{a_{n-1}}} & {\Kernel{b_{n-1}}} & {\Kernel{c_{n-1}}}
        \arrow[from=1-1, to=1-2]
        \arrow["{\hat{f}_{n-1}}", from=1-2, to=1-3]
        \arrow["{\hat{g}_{n-1}}", from=1-3, to=1-4]
    \end{tikzcd}\]
    Now we claim that there are induced maps $\overline{a_{n}},\overline{b_{n}},\overline{c_{n}}$, where $\overline{b_{n}}$
    is Rees regular, such that the following diagram commutes 
    \[\begin{tikzcd}
        & {\frac{A_n}{\im(a_{n+1})}} & {\frac{A_n}{\im(a_{n+1})}} & {\frac{A_n}{\im(a_{n+1})}} & 0 \\
        0 & {\Kernel{a_{n-1}}} & {\Kernel{b_{n-1}}} & {\Kernel{c_{n-1}}}
        \arrow["{\hat{f}_{n+1}}", from=1-2, to=1-3]
        \arrow["{\overline{a_{n}}}", from=1-2, to=2-2]
        \arrow["{\hat{g}_{n+1}}", from=1-3, to=1-4]
        \arrow["{\overline{b_{n}}}", "\circ"{marking}, from=1-3, to=2-3]
        \arrow[from=1-4, to=1-5]
        \arrow["{\overline{c_{n}}}", from=1-4, to=2-4]
        \arrow[from=2-1, to=2-2]
        \arrow["{\hat{f}_{n-1}}", from=2-2, to=2-3]
        \arrow["{\hat{g}_{n-1}}", from=2-3, to=2-4]
    \end{tikzcd}\]
    Indeed, define $\overline{a_n}([x]) = a_n(x)$ (similarly $\overline{b_n}$ and $\overline{c_n}$). This is well-defined, since
    $[x]=[y] \implies (x,y)\in\im(a_{n+1})\subseteq\ker{a_n} \implies a_n(x)=a_n(y)$. The images of the induced 
    maps land in the respective kernels. Since $b_n$ is Rees regular, we have
    $b_n(x)=\overline{b_n}([x])=\overline{b_n}([y])=b_n(y) \implies b_n(x)=0=b_n(y) \lor x=y$, therefore $[x]=[y] \lor \overline{b_n}([x])=\overline{b_n}([y])$
    and $\overline{b_n}$ is Rees regular. The commutativity of the diagram follows directly from the commutativity of the diagram
    we started with.\par 
    We can apply the snake lemma again one last time (notice that $\Kernel{\overline{a_n}} = H_n(A)$ and $\Cokernel{\overline{a_n}} = H_{n-1}(A)$, similarly
    for $\overline{b_n}$ and $\overline{c_n}$) to get 
    \[\begin{tikzcd}[sep=small]
        {H_{n}(A)} & {H_n(B)} & {H_n(C)} & {H_{n-1}(A)} & {H_{n-1}(B)} & {H_{n-1}(C)}
        \arrow["{H_n(f)}", from=1-1, to=1-2]
        \arrow["{H_n(g)}", from=1-2, to=1-3]
        \arrow["\partial", from=1-3, to=1-4]
        \arrow["{H_{n-1}(f)}", from=1-4, to=1-5]
        \arrow["{H_{n-1}(g)}", from=1-5, to=1-6]
    \end{tikzcd}\]
    We obtain the Rees long exact sequence by pasting these sequences together, which completes the proof.
\end{proof}
\begin{corollary}
    Let $0 \to (A,a) \to (B,b) \to (C,c) \to 0$ be a Rees short exact sequence of chain complexes. If two of them are 
    Rees exact, then so is the third.
\end{corollary}
\begin{proof}[Proof]
    The only thing of note is that if $(B,b)$ is Rees regular, then so are $(A,a)$ and $(C,c)$ by proposition \ref{regprop}. and 
    if $(A,a)$ and $(C,c)$ are Rees regular, then so is $(B,b)$ by lemma \ref{reglemma2}. Then the proof is exactly the same as 
    for modules.
\end{proof}
\begin{corollary}[3x3 lemma]
    Consider the following commutative diagram, whose rows are Rees exact
    \[\begin{tikzcd}
        & 0 & 0 & 0 \\
        0 & {A_3} & {B_3} & {C_3} & 0 \\
        0 & {A_2} & {B_2} & {C_2} & 0 \\
        0 & {A_1} & {B_1} & {C_1} & 0 \\
        & 0 & 0 & 0
        \arrow[from=1-2, to=2-2]
        \arrow[from=1-3, to=2-3]
        \arrow[from=1-4, to=2-4]
        \arrow[from=2-1, to=2-2]
        \arrow["{f_3}", from=2-2, to=2-3]
        \arrow["{r_3}", from=2-2, to=3-2]
        \arrow["{g_3}", from=2-3, to=2-4]
        \arrow["{r_2}", from=2-3, to=3-3]
        \arrow[from=2-4, to=2-5]
        \arrow["{r_1}", from=2-4, to=3-4]
        \arrow[from=3-1, to=3-2]
        \arrow["{f_2}", from=3-2, to=3-3]
        \arrow["{s_3}", from=3-2, to=4-2]
        \arrow["{g_2}", from=3-3, to=3-4]
        \arrow["{s_2}", from=3-3, to=4-3]
        \arrow[from=3-4, to=3-5]
        \arrow["{s_1}", from=3-4, to=4-4]
        \arrow[from=4-1, to=4-2]
        \arrow["{f_1}", from=4-2, to=4-3]
        \arrow[from=4-2, to=5-2]
        \arrow["{g_1}", from=4-3, to=4-4]
        \arrow[from=4-3, to=5-3]
        \arrow[from=4-4, to=4-5]
        \arrow[from=4-4, to=5-4]
    \end{tikzcd}\]
    \begin{enumerate}
        \item If the first and the second columns are Rees exact, then so is the third.
        \item If the second and the third columns are Rees exact, then so is the first.
        \item If the first and the third columns are Rees exact and additionally $s_2\circ r_2=0$, then so is the second.
    \end{enumerate}
\end{corollary}
\begin{proof}[Proof]
    For the first part we will have to show, that the sequence 
    \[\begin{tikzcd}
        \cdots & 0 & {C_3} & {C_2} & {C_1} & 0 & \cdots
        \arrow[from=1-1, to=1-2]
        \arrow[from=1-2, to=1-3]
        \arrow["{r_1}", from=1-3, to=1-4]
        \arrow["{s_1}", from=1-4, to=1-5]
        \arrow[from=1-5, to=1-6]
        \arrow[from=1-6, to=1-7]
    \end{tikzcd}\]
    is a chain complex. It suffices to prove that $\im(r_1)\subseteq\ker{s_1}$.
    We have to show $s_1\circ r_1 = 0$, since $g_3$ is epic we have 
    \[
        s_1\circ r_1\circ g_3 = s_1\circ g_2\circ r_2 = g_1\circ s_2\circ s_1 = g_1 \circ 0 = 0 = 0\circ g_3 \implies s_1\circ r_1 = 0
    \]
    Now applying the previous proposition to the evident chain complex yields the statement.\par
    Assume now that the second and the third columns are Rees exact. We again only have to show that 
    $s_3\circ r_3 = 0$. We have 
    \[
    f_1\circ s_3\circ r_3 = s_2 \circ f_2 \circ r_3 = s_2 \circ r_2 \circ f_3 = 0\circ f_3 = 0 = f_1\circ 0 \implies s_3\circ r_3 = 0
    \]
    For the third part we can just directly apply the previous proposition. This completes the proof.
\end{proof}

\chapter*{Conclusion and further generalisations}
\addcontentsline{toc}{chapter}{Conclusion}

    In the thesis we have constructed localisation of pointed acts and we have shown that the localisation functor preserves Rees short exact sequences. We 
have defined Rees simple acts and we have proved some of their basic properties. In the main part of the thesis we have defined chain complexes of pointed acts and 
proved that an analogue of the zigzag lemma holds for their homology acts. We gave a proof of the snake lemma and the nine lemma for acts, using proposition \ref{regularityProp}, we have 
also given a novel proves for the four and the five lemmata using the same proposition. \par 
    Possible ways the work can be extended would be to look at injective resolutions of acts, since they always yield Rees exact complexes, and see, if there is some notion, in which 
different resolutions are similar, i.e. if there is a suitable notions of chain homotopy, or if there is a more fruitful definition for chain complexes of pointed acts altogether. We are also of the 
opinion that the category $(\actcat{S})^{\mathrm{op}}$ might be homological, or at least protomodular, which would imply some of the results in chapter 3. Work can be also done to give a structural characterisation 
of Rees simple acts, which we have failed to find. 

%%% Bibliography
%%% Bibliography (literature used as a source)
%%%
%%% We employ biblatex to construct the bibliography. It processes
%%% citations in the text (e.g., the \cite{...} macro) and looks up
%%% relevant entries in the bibliography.bib file.
%%%
%%% See also biblatex settings in thesis.tex.

%%% Generate the bibliography. Beware that if you cited no works,
%%% the empty list will be omitted completely.

% We let bibliography items stick out of the right margin a little
\def\bibfont{\hfuzz=2pt}

\printbibliography[heading=bibintoc]

%%% If case you prefer to write the bibliography manually (without biblatex),
%%% you can use the following. Please follow the ISO 690 standard and
%%% citation conventions of your field of research.

% \begin{thebibliography}{99}
%
% \bibitem{lamport94}
%   {\sc Lamport,} Leslie.
%   \emph{\LaTeX: A Document Preparation System}.
%   2nd edition.
%   Massachusetts: Addison Wesley, 1994.
%   ISBN 0-201-52983-1.
%
% \end{thebibliography}


%%% Figures used in the thesis (consider if this is needed)
\listoffigures

%%% Tables used in the thesis (consider if this is needed)
%%% In mathematical theses, it could be better to move the list of tables to the beginning of the thesis.
\listoftables

%%% Abbreviations used in the thesis, if any, including their explanation
%%% In mathematical theses, it could be better to move the list of abbreviations to the beginning of the thesis.
\chapwithtoc{List of Abbreviations}

%%% Doctoral theses must contain a list of author's publications
\ifx\ThesisType\TypePhD
\chapwithtoc{List of Publications}
\fi

%%% Attachments to the thesis, if any. Each attachment must be referred to
%%% at least once from the text of the thesis. Attachments are numbered.
%%%
%%% The printed version should preferably contain attachments, which can be
%%% read (additional tables and charts, supplementary text, examples of
%%% program output, etc.). The electronic version is more suited for attachments
%%% which will likely be used in an electronic form rather than read (program
%%% source code, data files, interactive charts, etc.). Electronic attachments
%%% should be uploaded to SIS. Allowed file formats are specified in provision
%%% of the rector no. 72/2017. Exceptions can be approved by faculty's coordinator.
\appendix
\chapter{Attachments}

\section{First Attachment}

\end{document}
