\chapter*{Introduction}
\addcontentsline{toc}{chapter}{Introduction}

Monoids are algebraic structures with a single associative binary operation and an identity element. 
They appear naturally in various areas of mathematics and computer science, such as automata theory, formal languages, and the theory of computation. 
Understanding the structure and properties of monoids can provide insight into these fields.
In this thesis, we explore the concept of monoidal actions, also known as acts, which work as an analogue to modules over rings.\par 

In the first chapter we recall some basic definitions and results from category theory, which we will refer to throughout the thesis.\par  
The first section of the second chapter gives a definition of a pointed act and recalls some basic properties of pointed acts. In the second section we show that the category of pointed acts is complete and 
cocomplete, define localisation of pointed acts and show some properties the localisation has. In the last section we look at a particular 
notion of exactness of pointed acts, define a chain complex of pointed acts and look at a property analogous to Dedekind groups, i.e. Rees simple acts.\par 
In the third chapter we state and prove various diagram lemmata for pointed acts, such as the four, five and the nine lemma, snake lemma and the long homology sequence for
exact chain complexes.\\
\textbf{Previous results}\par 
    In an article by Y. Chen \cite{Chen02} a notion of exactness for pointed acts is defined and conditions are given for when such 
    short exact sequences split. In a PhD Thesis by Jaret Flores \cite{Flores15} admissible morphisms, i.e. morphisms that are injective up to kernel,
    and admissible short exact sequences are defined. We note that the two notions in fact coincide, which follows immediately from proposition \ref{regularityProp}.\par 
    Another article by M. Jafari, et al. \cite{Jafari19} from $2019$ gives a proof of the four and the five lemma and shows, how various classes of 
    pointed acts are inherited through Rees short exact sequences.