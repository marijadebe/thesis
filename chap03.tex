\chapter{Diagrams of acts}
    In this chapter we give state and prove different diagram lemmata for pointed acts. The proposition statements and proofs are 
    ours, with the exception of the four, five lemma and the splitting lemma, which appear in \cite{Jafari19} and \cite{Chen02} respectively, though we supply 
    a different proof.
\begin{proposition}\label{regprop}
    Let $f : A\to B$ and $g: B\to C$ be morphisms of pointed acts such that $g\circ f$ is Rees regular.
    \begin{enumerate}
        \item If $g$ is a monomorphism, then $f$ is Rees regular.
        \item If $f$ is an epimorphism, then $g$ is Rees regular.
    \end{enumerate}
\end{proposition}
\begin{proof}[Proof]
    \begin{enumerate}
       \item Let $f(x)=f(y)$, then $(g\circ f)(x)=(g\circ f)(y)$, which implies $x=y$ or 
       $g(f(x))=0=g(f(y))$. In the second case by injectivity of $g$ we have $f(x)=0=f(y)$.
       \item Assume $g(x)=g(y)$ we find $x_0,y_0\in A$ such that $g(f(x_0))=g(f(y_0))$, by regularity
       we have $x_0=y_0 \implies x=y$ or $g(x)=g(f(x_0))= 0 = g(f(y_0))=g(y)$.
    \end{enumerate}
\end{proof}
\begin{lemma}\label{reglemma1}
    Given a commutative diagram of pointed acts whose first row is Rees exact
    \[\begin{tikzcd}
        A & B & C \\
        {A'} & {B'} & {C'}
        \arrow["f", from=1-1, to=1-2]
        \arrow["\alpha", "\circ"{marking}, from=1-1, to=2-1]
        \arrow["g", from=1-2, to=1-3]
        \arrow["\beta", from=1-2, to=2-2]
        \arrow["\gamma", hook, from=1-3, to=2-3]
        \arrow["{f'}"', "\circ"{marking}, from=2-1, to=2-2]
        \arrow["{g'}"', from=2-2, to=2-3]
    \end{tikzcd}\]
    If $f'$ and $\alpha$ are Rees regular and $\gamma$ a monomorphism, then $\beta$ is Rees regular.
\end{lemma}
\begin{proof}[Proof]
    Assume that $\beta(x)=\beta(y)$ and $x\neq y$, then we have $\gamma(g(x))=g'(\beta(x))=g'(\beta(y))=\gamma(g(y))$, which 
    implies $g(x)=g(y)$, by Rees exactness at $B$ we find $x_0,y_0\in A$ such that $f(x_0)=x$ and $f(y_0)=y$, then 
    $f'(\alpha(x_0))=\beta(x)=\beta(y)=f'(\alpha(y_0))$. If $f'(\alpha(x_0)) =0 = f'(\alpha(y_0))$ we would have 
    $\beta(x) = 0 = \beta(y)$ by commutativity, else we have $\alpha(x_0)=\alpha(y_0)$, which by regularity implies
    $x_0=y_0\implies x=y$ or that $\alpha(x_0)=0=\alpha(y_0)$, which would again imply that $\beta(x)=0=\beta(y)$.
\end{proof}
\begin{lemma}\label{reglemma2}
    Consider the following commutative diagram whose second row and third column are Rees exact
    \[\begin{tikzcd}
        & B & C \\
        {A'} & {B'} & {C'} \\
        {A''} & {B''} & {C''}
        \arrow["g", two heads, from=1-2, to=1-3]
        \arrow["\beta", from=1-2, to=2-2]
        \arrow["\gamma", from=1-3, to=2-3]
        \arrow["{f'}", from=2-1, to=2-2]
        \arrow["{\alpha'}", "\circ"{marking}, from=2-1, to=3-1]
        \arrow["{g'}", from=2-2, to=2-3]
        \arrow["{\beta'}", from=2-2, to=3-2]
        \arrow["{\gamma'}", from=2-3, to=3-3]
        \arrow["{f''}", "\circ"{marking}, from=3-1, to=3-2]
        \arrow["{g''}", from=3-2, to=3-3]
    \end{tikzcd}\]
    If $g$ is an epimorphism, $\alpha'$, $f''$ are Rees regular and $\beta'\circ\beta = 0$, then $\beta'$ is Rees regular.
\end{lemma}
\begin{proof}[Proof]
    Suppose $\beta'(x)=\beta'(y)$, $x\neq y$, then $\gamma'(g'(x))=g''(\beta'(x))=g''(\beta'(y)) = \gamma'(g'(y))$ by 
    regularity we either have $g'(x)=g'(y)$ or $g'(x),g'(y)\in\Kernel{\gamma'}$. We will consider both cases separately. \par 
    In the first case, because of Rees exactness 
    we find $x_0,y_0\in A'$ such that $f'(x_0)=x$ and $g'(y_0)=y$, which implies 
    $f''(\alpha'(x_0))=\beta'(x)=\beta'(y)=f''(\alpha'(y_0))$. If $\alpha'(x_0)=\alpha'(y_0)$, then by regularity 
    of $\alpha'$ we would have $\alpha'(x_0)=0=\alpha'(y_0) \implies \beta(x)=0=\beta(y)$, since $x_0\neq y_0$.
    Otherwise we have $\beta'(x)=f''(\alpha'(x_0))=0=f''(\alpha'(y_0))=\beta'(y)$.\par
    In the second case, that is if $g'(x),g'(y)\in\Kernel{\gamma'}=\Image{\gamma}$, we find $x',y'\in C$ so that 
    $\gamma(x')=g'(x)$ and $\gamma(y')=g'(y)$. Since $g$ is an epimorphism we find $x'_0,y'_0\in B$ so that 
    $g'(x) = \gamma(g(x'_0))= g'(\beta(x'_0))$ and $g'(y)=\gamma(g(y'_0))=g'(\beta(y'_0))$. If, say 
    $x = \beta(x'_0)$ we would have $\beta'(y)=\beta'(x)=0$ and similarly if $y=\beta(y'_0)$. Otherwise we have $g'(x)=0=g'(y)$, but we have already resolved this case above. This completes the proof.
\end{proof}
The idea behind the lemmata above is to use them for larger diagrams to show Rees regularity wherever necessary and then using
proposition $\ref{regularityProp}$ we can do a regular diagram chase to verify Rees exactness.\par 
The following three results, i.e. the four lemma, five lemma and splitting lemma for acts already appear in \cite{Jafari19}. 
The splitting lemma in particular is first stated in \cite{Chen02}. We give more concise
proofs that utilize the characterization in proposition $\ref{regularityProp}$.
\begin{lemma}[Four lemma]
    Consider the following commutative diagram, whose rows are Rees exact
    \[\begin{tikzcd}
        A & B & C & D \\
        {A'} & {B'} & {C'} & {D'}
        \arrow["f", from=1-1, to=1-2]
        \arrow["r", from=1-1, to=2-1]
        \arrow["g", from=1-2, to=1-3]
        \arrow["s", from=1-2, to=2-2]
        \arrow["h", from=1-3, to=1-4]
        \arrow["t", from=1-3, to=2-3]
        \arrow["u", from=1-4, to=2-4]
        \arrow["{f'}", from=2-1, to=2-2]
        \arrow["{g'}", from=2-2, to=2-3]
        \arrow["{h'}", from=2-3, to=2-4]
    \end{tikzcd}\]
    then
    \begin{enumerate}
        \item If $s,u$ are monomorphisms and $r$ is an epimorphism, then $t$ is a monomorphism.
        \item If $r,t$ are epimorphisms and $u$ is a monomorphim, then $s$ is an epimorphism.
    \end{enumerate}
\end{lemma}
\begin{proof}[Proof]
    To prove the first part we first apply lemma \ref{reglemma1} to see that $t$ is Rees regular. Let $t(x)=0$, then 
    $u(h(x)) = h'(t(x))=0 \implies h(x)=0$, then using Rees exactness we find $x'\in B$ such that $g(x')=x$, then 
    $g'(s(x')) = t(x) = 0$, then we find $z\in A'$ such that $f'(z)=s(x')$, since $r$ is an epimorphism we find $z'\in A$
    such that $r(z')=z$, then $s(x')=f'(z)=f'(r(z'))=s(f(z')) \implies x'=f(z') \implies 0=g(x')=x$. Thus the kernel is 
    trivial, which implies that $t$ is injective by Rees regularity of $t$. \par
    To prove the second part choose $y\in B'$, then by surjectivity of $t$ we find $y'\in C$ such that $g'(y)=t(y')$, then 
    we have $u(h(y')) = h'(t(y'))= h'(g'(y)) = 0 \implies h(y')=0$. By Rees exactness we find $y''\in B$ such that 
    $g(y'')=y'$, then $g'(s(y'')) = t(g(y'')) = t(y') = g'(y)$. If $s(y'')=y$ we are done, otherwise by regularity of $g'$ we 
    have $g'(y)=0$, so by Rees exactness there is some $z\in A'$ such that $f'(z)=y$. Since $r$ is an epimorphism we find $z'\in A$
    so that $r(z')=z$, then $y = f'(r(z')) = s(f(z'))$, hence $y\in\Image{s}$, which concludes the proof.
\end{proof}
\begin{corollary}[Five lemma]
    Consider the following commutative diagram with Rees exact rows
    \[\begin{tikzcd}
            A & B & C & D & E \\
            {A'} & {B'} & {C'} & {D'} & {E'}
            \arrow["f", from=1-1, to=1-2]
            \arrow["r", from=1-1, to=2-1]
            \arrow["g", from=1-2, to=1-3]
            \arrow["s", from=1-2, to=2-2]
            \arrow["h", from=1-3, to=1-4]
            \arrow["t", from=1-3, to=2-3]
            \arrow["k", from=1-4, to=1-5]
            \arrow["u", from=1-4, to=2-4]
            \arrow["v", from=1-5, to=2-5]
            \arrow["{f'}", from=2-1, to=2-2]
            \arrow["{g'}", from=2-2, to=2-3]
            \arrow["{h'}", from=2-3, to=2-4]
            \arrow["{k'}", from=2-4, to=2-5]
    \end{tikzcd}\]
    if $s,u$ are isomorphisms, $r$ is an epimorphism and $v$ is a monomorphism, then $t$ is an isomorphism.
\end{corollary}
\begin{proof}[Proof]
    By applying the first and second part of the four lemma to our diagram we get that $t$ is a monomorphism and an 
    epimorphism, therefore an isomorphism, because the category of pointed acts is balanced.
\end{proof}
\begin{corollary}[Short five lemma]
    Consider the following commutative diagram with Rees exact rows
    \[\begin{tikzcd}
        0 & A & B & C & 0 \\
        0 & {A'} & {B'} & {C'} & 0
        \arrow[from=1-1, to=1-2]
        \arrow["f", from=1-2, to=1-3]
        \arrow["r", from=1-2, to=2-2]
        \arrow["g", from=1-3, to=1-4]
        \arrow["s", from=1-3, to=2-3]
        \arrow[from=1-4, to=1-5]
        \arrow["t", from=1-4, to=2-4]
        \arrow[from=2-1, to=2-2]
        \arrow["{f'}", from=2-2, to=2-3]
        \arrow["{g'}", from=2-3, to=2-4]
        \arrow[from=2-4, to=2-5]
    \end{tikzcd}\]
    \begin{enumerate}
        \item If $r$ and $t$ are monomorphisms, then $s$ is a monomorphism.
        \item If $r$ and $t$ are epimorphisms, then $s$ is an epimorphism.
        \item If $r$ and $t$ are isomorphism, then $s$ is an isomorphism. 
    \end{enumerate}
\end{corollary}
\begin{proof}[Proof]
    \begin{enumerate}
        \item Since $0$ is an epimorphism, $r$ and $t$ are monomorphisms we can apply the first part of the four lemma 
        to get the result.
        \item Similarly since $r$ and $t$ are epimorphisms and $0$ is a monomorphism we can apply the second part of the four lemma to the diagram.
        \item We can either use $1.$ and $2.$ or apply the five lemma.
    \end{enumerate}
\end{proof}

\begin{proposition}[Splitting sequences]
    Let 
    \[\begin{tikzcd}
        0 & A & B & C & 0
        \arrow[from=1-1, to=1-2]
        \arrow["f", from=1-2, to=1-3]
        \arrow["g", from=1-3, to=1-4]
        \arrow[from=1-4, to=1-5]
    \end{tikzcd}\]
    be a Rees short exact sequence, then TFAE:
    \begin{enumerate}
        \item $g$ is a split epimorphism
        \item $f$ is a split monomorphism and $(B\setminus\Image{f})\cup\{0\}$ is a subact of $B$.
        \item There is an isomorphism $\phi: B\to  A\coprod C$ such that $\phi\circ f=\iota_A$ and 
        $g\circ\phi^{-1}=\lambda_C$.
    \end{enumerate}
\end{proposition}
\begin{proof}[Proof]First we verify, that $1.$ implies $3.$
    Assume that there is a $g':C \to B$ such that $g\circ g' = 1_C$. We define a map $\psi: A\coprod C \to B$ as 
    $\psi(a,0) = f(a)$ and $\psi(0,c) = g'(c)$. This is a homomorphism of acts, because $g'$ and $f$ are. We have that 
    \[
        (\psi\circ \iota_A)(a) = \psi(a,0) = f(a) \implies \psi\circ\iota_A = f
    \]
    \[
        (g\circ\psi)(a,0) = g(f(a)) = 0\hspace{20pt} (g\circ\psi)(0,c) = (g\circ g')(c) = c \implies g\circ\psi = \lambda_C
    \]
    therefore this diagram with short exact rows commutes
    \[\begin{tikzcd}
	0 & A & {A\coprod C} & C & 0 \\
	0 & A & B & C & 0
	\arrow[from=1-1, to=1-2]
	\arrow["{\iota_A}", from=1-2, to=1-3]
	\arrow[equal,from=1-2, to=2-2]
	\arrow["{\lambda_C}", from=1-3, to=1-4]
	\arrow["\psi", from=1-3, to=2-3]
	\arrow[from=1-4, to=1-5]
	\arrow[equal,from=1-4, to=2-4]
	\arrow[from=2-1, to=2-2]
	\arrow["f", from=2-2, to=2-3]
	\arrow["g", shift left, from=2-3, to=2-4]
	\arrow["{g'}", shift left, from=2-4, to=2-3]
	\arrow[from=2-4, to=2-5]
\end{tikzcd}\]
    by the short five lemma $\psi$ is an isomorphism. Then we have that $\iota_A = \psi^{-1}\circ f$ and $g = \lambda_C\circ\psi^{-1}$ 
    by the commutativity of the squares above.\par 
    Now we verify that $2.$ implies $3.$ Assume that there is an $f': B \to A$ such that $f'\circ f = 1_A$. We define a map $\phi: B \to A\coprod C$ as
    \[
        \phi(b) = \begin{cases}
            (f'(b),0) & \text{if } b\in\Image{f} \\
            (0,g(b)) & \text{if } b\not\in\Image{f}
        \end{cases}
    \]
    We will now use the fact that $(B\setminus\Image{f})\cup\{0\}$ is a subact of $B$ to show it is a homomorphism. First off, if 
    $b\in\Image{f}$, then $rb\in\Image{f}$, hence 
    \[
    \phi(rb) = (f'(rb),0)= r\cdot(f'(b),0) = r\phi(b)
    \]
    Assume now that $b\not\in\Image{f}$ and $rb=0$, then 
    \[
        r\phi(b) = r\cdot(0,g(b)) = (0,g(rb)) = (0,0) = (f'(rb),0) = \phi(rb)
    \]
    If $rb\neq 0$, then $rb\not\in\Image{f}$ and 
    \[
        \phi(rb) = (0,g(rb)) = r\cdot(0,g(b)) = r\phi(b)
    \]
    therefore it is well-defined. We will again verify the commutativity of the following diagram and apply the short five lemma
    \[\begin{tikzcd}
        0 & A & B & C & 0 \\
        0 & A & {A\coprod C} & C & 0
        \arrow[from=1-1, to=1-2]
        \arrow["f", shift left, from=1-2, to=1-3]
        \arrow[equal, from=1-2, to=2-2]
        \arrow["{f'}", shift left, from=1-3, to=1-2]
        \arrow["g", from=1-3, to=1-4]
        \arrow["\phi", from=1-3, to=2-3]
        \arrow[from=1-4, to=1-5]
        \arrow[equal, from=1-4, to=2-4]
        \arrow[from=2-1, to=2-2]
        \arrow["{\iota_A}", from=2-2, to=2-3]
        \arrow["{\lambda_C}", from=2-3, to=2-4]
        \arrow[from=2-4, to=2-5]
    \end{tikzcd}\]
    \[
        (\phi\circ f)(a) = (f'(f(a)),0) = (a,0) = \iota_A(a) \implies \phi\circ f = \iota_A
    \]
    To verify the commutativity of the right triangle we need to consider two cases. Suppose that $b\in\Image{f}$, then 
    \[
        (\lambda_C\circ \phi)(b) = \lambda_C((f'(b),0)) = 0 = g(b)
    \]
    since $\Image{f}\subseteq\Kernel{g}$. Now assume $b\not\in\Image{f}$, then we have 
    \[
        (\lambda_C\circ\phi)(b) = \lambda_C(0,g(b)) = g(b)
    \]
    therefore the right triangle commutes, by the short five lemma we get that $\phi$ is an isomorphism, and we are done.\par 
    We will show that if the third condition holds, then both the first and the second hold. Let $\phi: B \to A\coprod C$ be 
    an isomorphism, such that $g\circ\phi^{-1}=\lambda_C$ and $\phi\circ f = \iota_A$.
    We have that 
    \[
        \lambda_A \circ \phi \circ f = \lambda_A \circ \iota_A = 1_A
    \]
    \[
        g\circ \phi^{-1} \circ \iota_C = \lambda_C\circ\iota_C = 1_C
    \]
    therefore $f$ and $g$ split. It remains to show that $B\setminus\Image{f} \cup\{0\}$ is a subact of $B$. Let 
    $b\not\in\Image{f}$ and choose $r\in S$, then $\phi(b)=(0,c)$, where $c\neq 0$, otherwise, if 
    $\phi(b)=(a,0)$, then $g(b) = (\lambda_C\circ \phi)(b) = 0 \implies b\in\Kernel{g}=\Image{f}$. Let $\phi(b)=(0,c)$, where 
    $c\neq 0$ and assume $rb\neq 0$ (otherwise we are done), then $(0,rc) = \phi(rb) \neq 0$, so $rc\neq 0$, but then 
    $g(rb) = (\lambda_C\circ\phi)(rb) = rc \neq 0$, so $rb\not\in\Image{f}$ as required.
\end{proof}
\begin{lemma}[Snake lemma for acts]
    Consider the following commutative diagram with Rees exact rows
    \[\begin{tikzcd}
        & A & B & C & 0 \\
        0 & {A'} & {B'} & {C'}
        \arrow["\alpha", from=1-2, to=1-3]
        \arrow["f", from=1-2, to=2-2]
        \arrow["\beta", from=1-3, to=1-4]
        \arrow["g", "\circ"{marking}, from=1-3, to=2-3]
        \arrow[from=1-4, to=1-5]
        \arrow["h", from=1-4, to=2-4]
        \arrow[from=2-1, to=2-2]
        \arrow["\gamma", from=2-2, to=2-3]
        \arrow["\delta", from=2-3, to=2-4]
    \end{tikzcd}\]
    such that $g$ is Rees regular, then we have a Rees exact sequence
    \[\begin{tikzcd}[sep=small]
        {\Kernel{f}} & {\Kernel{g}} & {\Kernel{h}} & {\Cokernel{f}} & {\Cokernel{g}} & {\Cokernel{h}}
        \arrow["{\hat{\alpha}}", from=1-1, to=1-2]
        \arrow["{\hat{\beta}}", from=1-2, to=1-3]
        \arrow["\partial", from=1-3, to=1-4]
        \arrow["{\hat{\gamma}}", from=1-4, to=1-5]
        \arrow["{\hat{\delta}}", from=1-5, to=1-6]
    \end{tikzcd}\]
    where $\partial$ is the so called connecting homomorphism. Moreover, if $\alpha$ is a monomorphism, then so is the induced 
    homomorphism $\hat{\alpha}$ and if $\delta$ is an epimorphism, then so is $\hat{\delta}$.
\end{lemma}
\begin{proof}[Proof]
    \begin{enumerate}
        \item The well-definedness of $\hat{\alpha},\hat{\beta}$ and Rees exactness at $\Kernel{g}$.\par
        We define the maps $\hat{\alpha}=\alpha\mid_{\Kernel{f}}$ and $\hat{\beta}=\beta\mid_{\Kernel{g}}$. To show 
        well-definedness let $f(x)=0$, then $0=\gamma(f(x))=g(\alpha(x)) \implies \alpha(x)\in\Kernel{g}$. Similarly if 
        $g(x)=0$ we have $0=\delta(g(x))=h(\beta(x)) \implies \beta(x)\in\Kernel{h}$. The Rees regularity at $\hat{\beta}$ is clear 
        because $\beta(x)=\hat{\beta}(x)=\hat{\beta}(y)=\beta(y)\implies x=y \lor \beta(x)=0=\beta(y)$. Let $\hat{\beta}(x)=0$, then
        we find $x'\in A$ such that $\alpha(x')=x$, then $\gamma(f(x'))=g(\alpha(x'))=g(x)=0$, since $x\in\Kernel{g}$, which implies
        $f(x')=0$,so $x\in\Image{\hat{\alpha}}$, since we have $\hat{\beta}\circ\hat{\alpha}=\beta\circ\alpha = 0$ the Rees exactness at $\Kernel{g}$ follows.
        \item The well-definedness of $\hat{\gamma},\hat{\delta}$ and Rees exactness at $\Cokernel{g}$.\par
        Let $\hat{\gamma}([x])=[\gamma(x)]$ and $\hat{\delta}([x])=[\delta(x)]$. We will show that $\hat{\gamma}$ is well-defined. 
        Let $[x]=[y]$, if $x=y$ we are done, otherwise there are $x_0,y_0\in A$ such that $f(x_0)=x$ and $f(y_0)=y$, then 
        $\gamma(x)=\gamma(f(x_0))=g(\alpha(x_0)) \implies \gamma(x)\in\Image{g}$ and similarly we get $\gamma(y)\in\Image{g}$, thus 
        $\hat{\gamma}([x])=[0]=\hat{\gamma}([y])$. In the same manner we can show that $\hat{\delta}$ is well-defined. Next let us
        see that $\hat{\delta}$ is Rees regular. If $[\delta(x)]=\hat{\delta}([x])=\hat{\delta}([y])=[\delta(y)]$ we either have 
        $\delta(x)=\delta(y) \implies x=y\lor \delta(x)=0=\delta(y)$ by regularity of $\delta$, which clearly implies $\hat{\delta}$ is Rees regular, 
        or else we have $\delta(x),\delta(y)\in\Image{h}$, which means precisely $\hat{\delta}([x])=[0]=\hat{\delta}([y])$. The fact 
        that $\hat{\delta}\circ\hat{\gamma}=0$ is clear from definition. Let $\hat{\delta}([x])=[0]$. If $\delta(x)=0$ by Rees exactness 
        we have $\gamma(x_0)=x$ for some $x_0\in A$, hence $\hat{\gamma}([x_0])=[x]$. Otherwise we have $\delta(x)\in\Image{h}$, so 
        since $\beta$ is epi we have some $x_0\in B$ such that $\delta(x)=h(\beta(x_0))=\delta(g(x_0))$. If $g(x_0)=x$ we have 
        $[x]=[0]$ and trivially it's in the image of $\hat{\gamma}$, else we find by Rees exactness some $z\in A$ s.t. $\gamma(z)=x$, which implies 
        $\hat{\gamma}([z])=[x]$ and we are done.
        \item Construction of $\partial$.\par 
        Let $x\in\Kernel{h}$, then we find $x_0\in B$ such that $\beta(x_0)=x$, since $\delta(g(x_0))=0$ we find 
        necessarily unique $x_0'\in A'$ so that $g(x_0)=\gamma(x_0')$. Let $\partial(x) = [x_0']$. We will fix this notation of the 
        construction of $\partial$ for the rest of the proof.
        \item Well-definedness of $\partial$. \par
        We will verify that $[x_0']$ does not depend on the choice of $x_0$.\par
        Let $x,y\in\Kernel{h},x=y$ we find $x_0,y_0\in B$ such that $\beta(x_0)=x=y=\beta(y_0)$. By regularity of $\beta$ 
        we either have $x_0=y_0 \implies \partial(x)=\partial(y)$ or $\beta(x_0)=0=\beta(y_0)$. In the second case
        by Rees exactness we find $\hat{x_0},\hat{y_0}\in A$ such that $\alpha(\hat{x_0})=x_0$ and $\alpha(\hat{y_0})=y_0$. 
        Continuing with our construction we have 
        $\gamma(x_0')=g(x_0)=\gamma(f(\hat{x_0})) \implies x_0'=f(\hat{x_0})$ and similarly $y_0'=f(\hat{y_0})$,
        which implies that $\partial(x)=[x_0']=[0]=[y_0']=\partial(y)$, therefore $\partial$ is well-defined.
        \item $\partial$ is a homomorphism of acts.\par 
        Let $\partial(rx)=[y_0']$. We have $\beta(y_0)=rx=r\beta(x_0)=\beta(rx_0)$, therefore by regularity of $\beta$ 
        we have $y_0=rx_0 \implies \partial(rx)=r\partial(x)$ or we find $\hat{y_0},z\in A$ such that 
        $\alpha(\hat{y_0})=y_0$ and $\alpha(z)= rx$, then  $\gamma(rx_0')=rg(x_0)=g(\alpha(z))=\gamma(f(z)) \implies rx_0'=f(z)$
        and $y_0' = f(\hat{y_0})$, thus
        \[
            r\partial(x)=r[x_0']=[rx_0']=[f(z)]=[0]=[f(\hat{y_0})]=[y_0']=\partial(rx)
        \] 
        \item Rees regularity of $\partial$.\par
        Let $[x_0']=\partial(x)=\partial(y)=[y_0'] \implies x_0'=y_0' \lor x_0',y_0'\in\Kernel{f}$. In the second case 
        we clearly have $\partial(x)=0=\partial(y)$. Assume that $x_0'=y_0'$, then $g(x_0)=g(y_0)$, by Rees regularity of $g$
        we either have $x_0=y_0 \implies x=\beta(x_0)=\beta(y_0)=y$ or $\gamma(x_0')=g(x_0)=0=g(y_0)=\gamma(y_0') \implies x_0'=0=y_0'$, 
        but then $\partial(x)=0=\partial(y)$.
        \item Rees exactness at $\Kernel{h}$.\par
        Let $[x_0']=\partial(x)=[0]$, that means that $x_0'\in\Image{f}$. Let $f(z)=x_0'$, then we have $g(x_0)=g(\alpha(z))$.
        If $x_0=\alpha(z)$ we have $x=\beta(x_0)=\beta(\alpha(z))=0$ and then clearly $x\in\Image{\hat{\beta}}$,since $\hat{\beta}(0)=x$.
        Otherwise by Rees regularity of $g$ we have $g(x_0)=0$, so $x_0\in\Kernel{g}$ and $\hat{\beta}(x_0)=\beta(x_0)=x$.
        It remains to show that $\partial\circ\hat{\beta}=0$. Choose $x\in\Kernel{g}$, we find $x_0\in B$ such that 
        $\beta(x_0)=\beta(x)$. If $x_0=x$ we have $\gamma(x_0')=g(x_0')=g(x)=0 \implies \partial(\hat{\beta}(x))=0$, else by Rees regularity
        of $\beta$ we have $\beta(x_0)=0$, so there is $z_0\in A$ such that $\alpha(z_0)=x_0$ by Rees exactness. This implies that 
        $\gamma(x_0')=g(x_0)=g(\alpha(z_0))=\gamma(f(z_0))$, hence $\partial(\hat{\beta}(x))=0$. This show Rees exactness at $\Kernel{h}$. 
        \item Rees exactness at $\Cokernel{f}$.\par 
        Note regularity of $\hat{\gamma}$, since $[\gamma(x)]=[\gamma(y)]$ implies $\gamma(x)=\gamma(y)\implies x=y$ or 
        $[\gamma(x)]=[0]=[\gamma(y)]$. To show $\hat{\gamma}\circ\partial = 0$, it is enough to observe that 
        $\hat{\gamma}(\partial(x))=[\gamma(x_0')]=[g(x_0)]=[0]$. Suppose $[\gamma(z_0')]=[0]$. There exists some 
        $z_0\in B$ such that $g(z_0)=\gamma(z_0')$. Let $z = \beta(z_0)$ and notice that 
        $h(z) = h(\beta(z_0))= \delta(g(z_0))=\delta(\gamma(z_0'))=0$, therefore $z\in\Kernel{h}$ and $\partial(z)=[z_0']$ by
        construction.
        \item $\alpha$ mono $\implies\hat{\alpha}$ mono and $\delta$ epi $\implies\hat{\delta}$ epi.\par 
        We have defined $\hat{\alpha}$ as a restriction of $\alpha$, therefore if $\alpha$ is injective, then so is $\hat{\alpha}$. 
        If $\delta$ is an epimorphism, then for each $y\in C'$ we find $x\in B'$ such that $\delta(x)=y$, then
        $\hat{\delta}([x]) = \delta(x) = y$, so $\hat{\delta}$ is an epimorphism.
        
    \end{enumerate}
\end{proof}
\begin{corollary}
    Let $0 \to (A,a) \to (B,b) \to (C,c) \to 0$ be a Rees short exact sequence of chain complexes and $(B,b)$ Rees regular, then 
    we have a Rees long exact sequence of homology acts 
    \[\begin{tikzcd}[sep=small]
        \cdots & {H_{n+1}(A)} & {H_n(A)} & {H_n(B)} & {H_n(C)} & {H_{n-1}(A)} & \cdots
        \arrow[from=1-1, to=1-2]
        \arrow["\partial", from=1-2, to=1-3]
        \arrow["{H_n(f)}", from=1-3, to=1-4]
        \arrow["{H_n(g)}", from=1-4, to=1-5]
        \arrow["\partial", from=1-5, to=1-6]
        \arrow[from=1-6, to=1-7]
    \end{tikzcd}\]
\end{corollary}
\begin{proof}[Proof]
    Since each $b_{n}$ is Rees regular we can apply a part of the Snake lemma to the following diagram
    \[\begin{tikzcd}
        0 & {A_{n+1}} & {B_{n+1}} & {C_{n+1}} & 0 \\
        0 & {A_n} & {B_n} & {C_n} & 0
        \arrow[from=1-1, to=1-2]
        \arrow[from=1-2, to=1-3]
        \arrow["{a_{n+1}}", from=1-2, to=2-2]
        \arrow[from=1-3, to=1-4]
        \arrow["{b_{n+1}}", from=1-3, to=2-3]
        \arrow[from=1-4, to=1-5]
        \arrow["{c_{n+1}}", from=1-4, to=2-4]
        \arrow[from=2-1, to=2-2]
        \arrow[from=2-2, to=2-3]
        \arrow[from=2-3, to=2-4]
        \arrow[from=2-4, to=2-5]
    \end{tikzcd}\]
    yielding a Rees short exact sequence 
    \[\begin{tikzcd}
        {\frac{A_n}{\im(a_{n+1})}} & {\frac{B_n}{\im(b_{n+1})}} & {\frac{C_n}{\im(c_{n+1})}} & 0
        \arrow["{\hat{f}_n}", from=1-1, to=1-2]
        \arrow["{\hat{g}_n}", from=1-2, to=1-3]
        \arrow[from=1-3, to=1-4]
    \end{tikzcd}\]
    Applying a part of the snake lemma to the same diagram shifted down two rows we get a Rees short exact sequence
    \[\begin{tikzcd}
        0 & {\Kernel{a_{n-1}}} & {\Kernel{b_{n-1}}} & {\Kernel{c_{n-1}}}
        \arrow[from=1-1, to=1-2]
        \arrow["{\hat{f}_{n-1}}", from=1-2, to=1-3]
        \arrow["{\hat{g}_{n-1}}", from=1-3, to=1-4]
    \end{tikzcd}\]
    Now we claim that there are induced maps $\overline{a_{n}},\overline{b_{n}},\overline{c_{n}}$, where $\overline{b_{n}}$
    is Rees regular, such that the following diagram commutes 
    \[\begin{tikzcd}
        & {\frac{A_n}{\im(a_{n+1})}} & {\frac{A_n}{\im(a_{n+1})}} & {\frac{A_n}{\im(a_{n+1})}} & 0 \\
        0 & {\Kernel{a_{n-1}}} & {\Kernel{b_{n-1}}} & {\Kernel{c_{n-1}}}
        \arrow["{\hat{f}_{n+1}}", from=1-2, to=1-3]
        \arrow["{\overline{a_{n}}}", from=1-2, to=2-2]
        \arrow["{\hat{g}_{n+1}}", from=1-3, to=1-4]
        \arrow["{\overline{b_{n}}}", "\circ"{marking}, from=1-3, to=2-3]
        \arrow[from=1-4, to=1-5]
        \arrow["{\overline{c_{n}}}", from=1-4, to=2-4]
        \arrow[from=2-1, to=2-2]
        \arrow["{\hat{f}_{n-1}}", from=2-2, to=2-3]
        \arrow["{\hat{g}_{n-1}}", from=2-3, to=2-4]
    \end{tikzcd}\]
    Indeed, define $\overline{a_n}([x]) = a_n(x)$ (similarly $\overline{b_n}$ and $\overline{c_n}$). This is well-defined, since
    $[x]=[y] \implies (x,y)\in\im(a_{n+1})\subseteq\ker{a_n} \implies a_n(x)=a_n(y)$. The images of the induced 
    maps land in the respective kernels. Since $b_n$ is Rees regular, we have
    $b_n(x)=\overline{b_n}([x])=\overline{b_n}([y])=b_n(y) \implies b_n(x)=0=b_n(y) \lor x=y$, therefore $[x]=[y] \lor \overline{b_n}([x])=\overline{b_n}([y])$
    and $\overline{b_n}$ is Rees regular. The commutativity of the diagram follows directly from the commutativity of the diagram
    we started with.\par 
    We can apply the snake lemma again one last time (notice that $\Kernel{\overline{a_n}} = H_n(A)$ and $\Cokernel{\overline{a_n}} = H_{n-1}(A)$, similarly
    for $\overline{b_n}$ and $\overline{c_n}$) to get 
    \[\begin{tikzcd}[sep=small]
        {H_{n}(A)} & {H_n(B)} & {H_n(C)} & {H_{n-1}(A)} & {H_{n-1}(B)} & {H_{n-1}(C)}
        \arrow["{H_n(f)}", from=1-1, to=1-2]
        \arrow["{H_n(g)}", from=1-2, to=1-3]
        \arrow["\partial", from=1-3, to=1-4]
        \arrow["{H_{n-1}(f)}", from=1-4, to=1-5]
        \arrow["{H_{n-1}(g)}", from=1-5, to=1-6]
    \end{tikzcd}\]
    We obtain the Rees long exact sequence by pasting these sequences together, which completes the proof.
\end{proof}
\begin{corollary}
    Let $0 \to (A,a) \to (B,b) \to (C,c) \to 0$ be a Rees short exact sequence of chain complexes. If two of them are 
    Rees exact, then so is the third.
\end{corollary}
\begin{proof}[Proof]
    The only thing of note is that if $(B,b)$ is Rees regular, then so are $(A,a)$ and $(C,c)$ by proposition \ref{regprop}. 
    If $(A,a)$ and $(C,c)$ are Rees regular, then so is $(B,b)$ by lemma \ref{reglemma2}. Then the proof follows easily from the 
    proposition above and the fact that homology of a Rees regular chain complex is zero, iff Rees exactness holds.
\end{proof}
\begin{corollary}[3x3 lemma]
    Consider the following commutative diagram, whose rows are Rees exact
    \[\begin{tikzcd}
        & 0 & 0 & 0 \\
        0 & {A_3} & {B_3} & {C_3} & 0 \\
        0 & {A_2} & {B_2} & {C_2} & 0 \\
        0 & {A_1} & {B_1} & {C_1} & 0 \\
        & 0 & 0 & 0
        \arrow[from=1-2, to=2-2]
        \arrow[from=1-3, to=2-3]
        \arrow[from=1-4, to=2-4]
        \arrow[from=2-1, to=2-2]
        \arrow["{f_3}", from=2-2, to=2-3]
        \arrow["{r_3}", from=2-2, to=3-2]
        \arrow["{g_3}", from=2-3, to=2-4]
        \arrow["{r_2}", from=2-3, to=3-3]
        \arrow[from=2-4, to=2-5]
        \arrow["{r_1}", from=2-4, to=3-4]
        \arrow[from=3-1, to=3-2]
        \arrow["{f_2}", from=3-2, to=3-3]
        \arrow["{s_3}", from=3-2, to=4-2]
        \arrow["{g_2}", from=3-3, to=3-4]
        \arrow["{s_2}", from=3-3, to=4-3]
        \arrow[from=3-4, to=3-5]
        \arrow["{s_1}", from=3-4, to=4-4]
        \arrow[from=4-1, to=4-2]
        \arrow["{f_1}", from=4-2, to=4-3]
        \arrow[from=4-2, to=5-2]
        \arrow["{g_1}", from=4-3, to=4-4]
        \arrow[from=4-3, to=5-3]
        \arrow[from=4-4, to=4-5]
        \arrow[from=4-4, to=5-4]
    \end{tikzcd}\]
    \begin{enumerate}
        \item If the first and second columns are Rees exact, then so is the third.
        \item If the second and third columns are Rees exact, then so is the first.
        \item If the first and third columns are Rees exact and additionally $s_2\circ r_2=0$, then so is the second.
    \end{enumerate}
\end{corollary}
\begin{proof}[Proof]
    For the first part we will have to show, that the sequence 
    \[\begin{tikzcd}
        \cdots & 0 & {C_3} & {C_2} & {C_1} & 0 & \cdots
        \arrow[from=1-1, to=1-2]
        \arrow[from=1-2, to=1-3]
        \arrow["{r_1}", from=1-3, to=1-4]
        \arrow["{s_1}", from=1-4, to=1-5]
        \arrow[from=1-5, to=1-6]
        \arrow[from=1-6, to=1-7]
    \end{tikzcd}\]
    is a chain complex. It suffices to prove that $\im(r_1)\subseteq\ker{s_1}$.
    We have to show $s_1\circ r_1 = 0$, since $g_3$ is epic we have 
    \[
        s_1\circ r_1\circ g_3 = s_1\circ g_2\circ r_2 = g_1\circ s_2\circ s_1 = g_1 \circ 0 = 0 = 0\circ g_3 \implies s_1\circ r_1 = 0
    \]
    Now applying the previous proposition to the chain complex above yields the statement.\par
    Assume now that the second and the third columns are Rees exact. We again only have to show that 
    $s_3\circ r_3 = 0$. We have 
    \[
    f_1\circ s_3\circ r_3 = s_2 \circ f_2 \circ r_3 = s_2 \circ r_2 \circ f_3 = 0\circ f_3 = 0 = f_1\circ 0 \implies s_3\circ r_3 = 0
    \]
    For the third part we can just directly apply the previous proposition. This completes the proof.
\end{proof}
