\chapter{Preliminaries}
    We recall some notions from the theory of categories. This chapters exposition will loosely follow \cite{Leinster14}.
\begin{definition}
    A \textbf{category} $\catc$ consists of
    \begin{enumerate}
        \item a class of objects $\ob{\catc}$
        \item for each $A,B\in\ob{\catc}$ a class of morphisms $\homc{\catc}{A}{B}$ called the hom-set of $A$ and $B$\\
        We will write $f : A\to B$ or
$\begin{tikzcd}
A & B
\arrow["f", from=1-1, to=1-2]
\end{tikzcd}$ for $f\in\homc{\catc}{A}{B}$
        \item for each $A,B,C\in\ob{\catc}$ a class map $\circ$ : $\homc{\catc}{B}{C}\times\homc{\catc}{A}{B}\to \homc{\catc}{A}{C}$ such that
    for morphisms $\begin{tikzcd}
        A & B & C & D
        \arrow["f", from=1-1, to=1-2]
        \arrow["g", from=1-2, to=1-3]
        \arrow["h", from=1-3, to=1-4]
    \end{tikzcd}$ we have $h\circ(g\circ f) = (h\circ g)\circ f$
        \item for each object $A\in\ob{\catc}$ we have a morphism $1_A : A\to A$ satisfying
         for all morphisms $\begin{tikzcd}
            X & A & Y
            \arrow["f", from=1-1, to=1-2]
            \arrow["g", from=1-2, to=1-3]
        \end{tikzcd}$ the identity
        \[1_A \circ f = f \hspace{20pt} g \circ 1_A = g \]
        which we will call the identity morphism. 
    \end{enumerate}
    A category $\catc$ is \textbf{small}, if all hom-sets are sets and $\ob{\catc}$ is a set. \par
    We define the \textbf{opposite category} $\opp{\catc}$ by
    \[\ob{\opp{\catc}} = \ob{\catc}\]
    \[\homc{\opp{\catc}}{A}{B} = \homc{\catc}{B}{A}\]
    For $A,B,C\in\ob{\opp{\catc}}$ and morphisms $\begin{tikzcd}
	A & B & C
	\arrow["f", from=1-1, to=1-2]
	\arrow["g", from=1-2, to=1-3]
\end{tikzcd}$ we'll let $g\circop f = f\circ g$ and define the identity morphisms of $\opp{\catc}$ as the identity morphisms of $\catc$.
\end{definition}
\begin{definition}
        Let $\catc$ and $\catd$ be categories. We will say that $\catc$ is a \textbf{subcategory} of $\catd$, if 
        $\ob{\catc}$ is a subclass of $\ob{\catd}$, for each $A,B\in\ob{\catc}$ the hom-set
        $\homc{\catc}{A}{B}$ is a subclass of $\homc{\catd}{A}{B}$, the identity morphisms in $\catc$ are 
        the identity morphisms in $\catd$ and the composition $\circ$ in $\catc$ is the restriction of composition in $\catd$.
        If $\forall A,B\in\ob{\catc} : \homc{\catc}{A}{B} = \homc{\catd}{A}{B}$
        then we call $\catc$ a \textbf{full subcategory} of $\catd$.    
\end{definition}
\begin{definition}
    Let $\catc$ and $\catd$ be categories. A \textbf{functor} $F: \catc \to \catd$ consists of 
    \begin{enumerate}
        \item a class map $\ob{\catc}\to\ob{\catd}$ where $A \mapsto F(A)$
        \item for each $A,B\in\ob{\catc}$ a class map $\homc{\catc}{A}{B}\to\homc{\catd}{F(A)}{F(B)}$ where $f \mapsto F(f)$
    \end{enumerate}
    such that it preserves composition $F(g\circ f) = F(g)\circ F(f)$ and the identity $F(1_A) = 1_{F(A)}$.\par
    Given a functor $F: \catc\to\catd$ we define the \textbf{opposite functor} $F^{op}: \opp{\catc}\to\opp{\catd}$ as 
    $F^{op}(A) = F(A)$ for $A\in\ob{\opp{\catc}}$ and $F^{op}(f)=F(f)$ for a morphism $f$.\par
    A functor $F:\catc\to\catd$ is \textbf{faithful}, if it is injective on hom-sets.
    More precisely for all $A,B\in\ob{\catc}$ and each pair of morphisms $\begin{tikzcd}
        A & B
        \arrow["g"', shift right, from=1-1, to=1-2]
        \arrow["f", shift left, from=1-1, to=1-2]
    \end{tikzcd}$ if $F(f)=F(g)$ then $f=g$.\\ 
\end{definition}
\begin{definition}
    Let $f: A\to B$ be a morphism in some category, then we call $f$
    \begin{enumerate}
        \item an \textbf{isomorphism}, if there exists $f^{-1} : B\to A$ such that $f\circ f^{-1} = 1_B$ and $f^{-1}\circ f = 1_A$.
        \item a \textbf{split monomorphism}, if there exists $g: B\to A$ such that $g\circ f= 1_A$.
        \item a \textbf{split epimorphism}, if it is a split monomorphism in the opposite category.
        \item a \textbf{monomorphism}, if for all objects $X$ and $g,h: X \to A$ we have $f\circ g =f\circ h \implies g=h$.
        \item an \textbf{epimorphism}, if it is a monomorphism in the opposite category.
        \item a \textbf{bimorphism}, if it is both a monomorphism and an epimorphism.
    \end{enumerate}
    We call a category \textbf{balanced}, if every bimorphism in it is an isomorphism.
\end{definition}
\begin{definition}
    A category $\catc$ is \textbf{pointed}, if there exists an object $0\in\ob{\catc}$ such that for all $A\in\ob{\catc}$ there are 
    unique morphisms $0 \to A$ and $A \to 0$, such an object is necessarily unique up to a unique isomorphism and we call it the \textbf{zero object of $A$}. For objects $A,B$ in a 
    pointed category we define the \textbf{zero morphism} $0: A\to B$ to be the unique morphism, that factors through the zero object.
\end{definition}
\begin{definition}
    Given a small category $\mathbf{J}$ and a category $\catc$ we call any functor $F:\mathbf{J}\to\catc$ a \textbf{diagram in $\catc$}. We call the tuple
    $(A,(f_j : A \to F(j))_{j\in\mathbf{\ob{J}}})$ where $A\in\ob{A}$ a \textbf{cone of the diagram $F$}, if for all 
    $j_1,j_2\in\ob{\mathbf{J}}$ and $g: j_1 \to j_2$ we have $F(g)\circ f_{j_1} = f_{j_2}$.\par
    Let $F: \mathbf{J} \to\catc$ be a diagram and $(U, (f_j : U \to F(j))_{j\in\mathbf{\ob{J}}})$ a cone over $F$. We say the cone $(U, (f_j : U \to F(j))_{j\in\mathbf{\ob{J}}})$
    is \textbf{universal} or a \textbf{limit} of the diagram $F$, if given any other cone $(A,(g_j: A \to F(j))_{j\in\mathbf{\ob{J}}})$ 
    there exists a unique morphism $g: A \to U$ such that $g\circ f_j = g_j$ for all $j\in\ob{J}$.\par
    Given a diagram $F: \mathbf{J}\to\catc$ we define a \textbf{colimit} of $F$ as a limit of $F^{op}$ in $\opp{\catc}$.
\end{definition}
\begin{remark}
    If $(U, (f_j : U \to F(j))_{j\in\mathbf{\ob{J}}})$ and $(V, (g_j : V \to F(j))_{j\in\mathbf{\ob{J}}})$ are universal cones, 
    there exists a unique isomorphism $\phi: U \to V$ such that $g_j\circ\phi=f_j$, we will choose one universal cone from this isomorphism class 
    and call it the limit of diagram $F$ and similarly we can speak of the colimit of a diagram. In most categories there is a very canonical way to make this choice. 
\end{remark}
\begin{definition}
    We call $\mathbf{J}$ a \textbf{discrete} category, if the only morphisms in $\mathbf{J}$ are the identity morphisms.
    Given a diagram $F:\mathbf{J}\to\catc$ of a discrete category, we call its (co)limit the \textbf{(co)product of} $(F(j))_{j\in\mathbf{\ob{J}}}$.
\end{definition}
\begin{definition}
    If $\mathbf{J}$ is a category with two objects and two morphisms
    of the form 
        \[\begin{tikzcd}
        \bullet & \bullet
        \arrow["{j_1}", shift left, from=1-1, to=1-2]
        \arrow["{j_2}"', shift right, from=1-1, to=1-2]
    \end{tikzcd}\]
    we call its (co)limit the \textbf{(co)equaliser of $F(j_1)$ and $F(j_2)$}.
\end{definition}
\begin{definition}
    A category is called (co)complete, if it has all (co)limits.
\end{definition}
\begin{theorem}
    A category is (co)complete, if and only if it has all (co)equalisers and (co)products.
\end{theorem}
\begin{proof}
    See \cite[Proposition 5.1.26]{Leinster14}.
\end{proof}
\begin{definition}
    If $f$ is a morphism in a pointed category, we will call the \textbf{(co)kernel of $f$} the (co)equaliser of $f$ and the zero morphism, if it exists.
\end{definition}
\begin{definition}
    A monomorphism $f$ in a pointed category $\catc$ with kernels and cokernels is called normal, if $f$ is a kernel of its cokernel. An 
    epimorphism $f$ is called conormal, if it is a cokernel of its kernel. The category $\catc$ is then called 
    (co)normal, if every (epi)monomorphism 
    in it is (co)normal.
\end{definition}
\begin{definition}
    A tuple $(\catc,U)$, where $\catc$ is a category and $U:\catc\to\mathbf{Set}$ is a functor will be called a \textbf{concrete category}, if $U$ is faithful.
\end{definition}
    In a concrete category we say that a morphism $f$ is injective/surjective, if $U(f)$ is injective/surjective. It is easy to see 
    that any injective/surjective morphism in a concrete category is a mono/epimorphism.
\iffalse
\section{Algebras}
\begin{definition}
    Let $\Omega$ be a nonempty set and $\tau: \Omega\to\mathbb{N}_0$ a map, which we will call the \textbf{signature of $\Omega$}.
    We will call the tuple $(\Omega,\tau)$ an \textbf{algebra type}.
\end{definition}
\begin{definition}
    Let $(\Omega, \tau)$ be an algebra type, we call the tuple $(A,(F_\omega)_{\omega\in\Omega})$ an \textbf{algebra of type $(\Omega,\tau)$}, if
    $A$ is a set and for each $\omega\in\Omega$ we have $F_\omega : A^{\tau(\omega)}\to A$, we then say that $A$ is a universe of the algebra 
    and $F_{\omega}$ an operation of arity $\tau(\omega)$ on $A$. We will also, for readibility, denote $F_\omega$ as $\omega$ (keeping in mind, that we cannot exactly 
    identify $F_\omega$ with $\omega$).
\end{definition}
\begin{remark}
    We identify $A^0$ with the set $\{\emptyset\}$ and the map $e: A^0 \to A$ with its image $e(\emptyset)$. In case $\Omega$ is finite of cardinality $n$, we will denote 
    it as an ordered $n$-tuple $\Omega=(\omega_1,\dots,\omega_n)$ such that $\tau(\omega_1)\geq\dots\geq\tau(\omega_n)$ and we'll denote its signature as 
    $(\tau(\omega_1),\dots,\tau(\omega_n))$.
\end{remark}
\begin{definition}
    Let $(A,(F_\omega)_{\omega\in\Omega}),(B,(G_\omega)_{\omega\in\Omega})$ be two algebras of type $(\Omega,\tau)$ we say 
    $f:A \to B$ is a \textbf{homomorphism}, if 
    \[
        \forall\omega\in\Omega\,\forall x_1,\dots,x_{\tau(\omega)} : f(F_\omega(x_1,\dots,x_{\tau(\omega)})) = G_\omega(f(x_1),\dots,f(x_{\tau(\omega)})) 
    \]
\end{definition}
\begin{remark}
    Algebras of type $(\Omega,\tau)$ along with their homomorphisms form a category $\mathrm{Alg}_{(\Omega,\tau)}$.
\end{remark}
\begin{definition}
    Let $(A,(F_\omega)_{\omega\in\Omega})$ be an algebra of type $(\Omega,\tau)$ and equivalence relation $\sim$ on $A$ is a \textbf{congruence}
    on $A$, if for each $\omega\in\Omega$ and sequences $x_1,\dots,x_{\tau(\omega)},y_1,\dots,y_{\tau(\omega)}\in A$
    \[
        x_1\sim y_1,\dots, x_{\tau(\omega)}\sim y_{\tau(\omega)} \implies F_\omega(x_1,\dots,x_{\tau(\omega)}) \sim F_\omega(y_1,\dots,y_{\tau(\omega)})
    \]
\end{definition}
    If $f: A \to B$ is a homomorphism of algebras, we define a relation $\ker{f} = \{(x,y)\in A^2 : f(x)=f(y)\}$, such a relation is
    always a congruence on the algebra $A$ and we will call it the kernel congruence of $f$. On every algebra we also have the 
    smallest and the largest congruence $\Delta_A = \{(x,x)\in A^2 : x\in A\}$ (which we will call the diagonal relation) and $A\times A$ respectively. \par 
    The congruences on $A$ form a lattice, whose meet is defined to be the intersection and whose join is the intersection of all congruences 
    that contain the union of a given set of congruences (the least congruence generated by the union).
\begin{definition}
    Let $(A,(F_\omega)_{\omega\in\Omega})$ be an algebra of type $(\Omega,\tau)$ and $\sim$ a congruence on $A$. We define a 
    \textbf{factor algebra} $(A/\sim,(G_\omega)_{\omega\in\Omega})$, where we let
    \[
        G_\omega([x_1]_\sim,\dots,[x_\tau(\omega)]_\sim) = [F_\omega(x_1,\dots,x_{\tau(\omega)})]_\sim
    \] 
    these operations are well defined, since $\sim$ is a congruence on $A$.
\end{definition}
\begin{definition}
    Let $(A,(F_\omega)_{\omega\in\Omega})$,$(B,(G_\omega)_{\omega\in\Omega})$ be algebras of type $(\Omega,\tau)$. We call $B$ a \textbf{subalgebra 
    of $A$}, if $B\subseteq A$ and for each $\omega\in\Omega$ we have $F_\omega\restriction_B = G_\omega$.
\end{definition}
\fi