\chapter*{Conclusion and further generalisations}
\addcontentsline{toc}{chapter}{Conclusion}

    In the thesis we have constructed localisation of pointed acts and we have shown that the localisation functor preserves Rees short exact sequences. We 
have defined Rees simple acts and we have proved some of their basic properties. In the main part of the thesis we have defined chain complexes of pointed acts and 
proved that an analogue of the zigzag lemma holds for their homology acts. We gave a proof of the snake lemma and the nine lemma for acts, using proposition \ref{regularityProp}, we have 
also given a novel proves for the four and the five lemmata using the same proposition. \par 
    Possible ways the work can be extended would be to look at injective resolutions of acts, since they always yield Rees exact complexes, and see, if there is some notion, in which 
different resolutions are similar, i.e. if there is a suitable notions of chain homotopy, or if there is a more fruitful definition for chain complexes of pointed acts altogether. We are also of the 
opinion that the category $(\actcat{S})^{\mathrm{op}}$ might be homological, or at least protomodular, which would imply some of the results in chapter 3. Work can be also done to give a structural characterisation 
of Rees simple acts, which we have failed to find. 